\section{Anonymous and Attributed Submissions}

Starting off with mere support for anonymous submissions has numerous benefits:

\begin{enumerate}

\item The submission can be purged from the test server once the feedback is
given. This means that OnlineTA does not have to offer secure long-term
submission storage capabilities. Submission data is never at rest.

\item We do not need to think about offering a student-friendly means of
offering both attribution authentication and secure long-term storage of
submissions.

\item We can offer access to the service to the general public, which may be
useful for the public relations of the project.

\item An anonymous test server setup can be decorated with an attribution
front-end, enabling a simple solution for attributed submissions. That is, the
front-end is responsible for attribution verification, and then issues an
anonymous test of the submission on the anonymous test server setup. It is the
attribution server that then also makes sure to collect the feedback and offer
secure long-term storage.

\end{enumerate}

Down-sides:

\begin{enumerate}

\item UNIX permissions cannot be easily employed, but can be offered from a uid
pool, which we have to make sure to maintain throughout the lifecycle of a
test.

\end{enumerate}
