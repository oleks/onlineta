% vim: set spell:

\section{Purpose and Environment}

The purpose of the OnlineTA is to aid traditional on-site education at the
Department of Computer Science, University of Copenhagen, Denmark (DIKU).

Denmark is a signatory of the Bologna process, an international European
agreement for the development of a common European Higher Education Area
\cite{ehea1999}. The intent of the agreement is to further international
cooperation, allow students and researchers to move freely across borders, and
to make European universities more attractive worldwide.

To adhere to the agreement, an education at the University of Copenhagen is
split into two main cycles: a 3-year Bachelor education, and a subsequent
2-year Master's education, and as the Danish law \cite{bek814} mandates:

\begin{itemize}

\item \textbf{\textsection\ 6, part 1 \& 2:} A Bachelor and Master's
education is accredited using the European Credit Transfer System (ECTS).

\item \textbf{\textsection\ 6, part 3:} 60 ECTS points corresponds to 1 year's
worth of full time studies. 

\item \textbf{\textsection\ 11, part 1:} A Bachelor education is attributed
180 ECTS points.

\item \textbf{\textsection\ 17, part 1:} A Master's education is attributed 120
ECTS points.

\end{itemize}

The Department of Computer Science is a subordinate of the Faculty of Science
at the University of Copenhagen. As such, it adheres to the faculty-wide block
structure\cite{cnd2004}. A study year is split into 4 blocks, each
corresponding to 9 weeks worth of full time work and 15 ECTS credits. Most
courses require half time work for 9 weeks, yielding 7.5 ETCS credits. Some
courses span multiple blocks, or require full time work, yielding 15 ECTS
credits, or more. Courses aside, students can also take up projects, write a
Bachelor project, or a Thesis, but we will not concern ourselves with such
matters here.

According to Danish law \cite[\textsection 6, parts 1 \& 2]{bek814}, a
Bachelor and Master's education is built up of multiple modules. A module is
a unit of study, providing the student with some qualifications and skills,
yielding a certain number of ECTS points, and completed by passing one or more
exams. A DIKU course roughly corresponds to a module: every course is completed
by passing an exam.

According to Danish law \cite[\textsection 14, parts 1 \& 4]{bek666}, a
student enrolled for a course is automatically admitted to the exam in that
course, unless the student throughout the course does not meet the requirements
beforehand set by the university for being admitted to the exam. According to
Danish law \cite[\textsection 3, parts 2 \& 3]{bek666} this may be
requirements of participation, requirements to submit written assignments
during the course, or similar. The requirements are further specified by the
university curriculum.

The Faculty of Science at the University of Copenhagen has both a shared BSc
and MSc curricula for study programmes\cite{faelles-del-ENG}, and
programme-specific curricula.

According to the shared faculty curricula, the individual courses offered by
the faculty, including the requirements to be met in order to be admitted to
the exam, are described in the University of Copenhagen course catalogue, at
\url{http://kurser.ku.dk}. This excludes the aforementioned projects, Bachelor
project and Thesis, which are described in the faculty curricula itself.

Most courses at DIKU run for the aforementioned 9 weeks, and expect half time
study. The first 7 of the weeks are used for teaching and various mandatory
written assignments, one week is left for exam study, and the last week is left
for the exams themselves. The written assignments serve as the basis for the
judgement of whether the individual student should be admitted to the exam or
not.

The purpose of the OnlineTA is to aid in the judgement of whether a student is
to be admitted to the exam as well as aid in the evaluation of the student's
exam performance itself.

\section{Notes}

Bestået/ikke bestået. Mødepligt.

This is the
main decision to be made in the course of the first 7 weeks, individually for
every student. The purpose of the OnlineTA is to aid in this judgement.

The purpose of the eduction is, in part, to develop competence in computable
processes and structures. This typically involves both theory and practice,
i.e. both reasoning about computable processes and structures, and implementing
them on a computer. The two approaches reinforce each other, providing insight
both into what is reasonable, and what is feasible.

% If we use a reference to CS2013, we have to justify why its relevant to the
% education at DIKU.

Computer Science is not a unimodal science. There are many different topics,
and many different areas of interest, all of which cannot fit into the head of
any one student.

A course spans a period of time, has a curriculum, and demands the student to
complete some sort of individual work. It is the student's individual work that
forms the basis for the evaluation of the student's performance. It is the
accumulative performance of the students in all their courses that form their
degrees.

All courses may either be passed or not passed by a student.

Most courses are graded on a 7-grade scale, but some are merely pass or no pass

 which will be used as the basis for the assessment of a student's work.

A course is a something that spans a 

The education involves the student participating in various courses. A course
spans a period of time, has a particular curriculum, and particular problems
that the student is asked to solve. The performance of the student is then
judged on the basis of their work.

 The system
should aid the teaching staff in their judgements of individual students.

Of particular interest are programming courses, i.e. courses that involve the
writing of programs to be executed by a computer.
