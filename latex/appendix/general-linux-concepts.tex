% vim: set spell:

\chapter{General Linux Concepts}

This chapter covers some general Linux concepts which are not necessarily to
known to the canonical Linux user.

\section{Tasks}

\label{appendix:general-linux-concepts:tasks}

The distinction between a thread and a process in the Linux kernel is somewhat
more subtle than in textbook operating systems practice.

In the Linux kernel, a thread of execution has a thread ID (also called a PID),
a thread group ID, and a parent thread group ID. New threads of execution can
be created using the \cite{man-2-clone} system call\footnote{The more canonical
\cite{man-2-fork} system call is seldom used. Its behaviour can be mimicked by
\cite{man-2-clone}. This is indeed what the standard glibc \texttt{fork()}
function does.}. Depending on the parameters passed, the child can share
various parts of its execution context with its parent. For instance, we may
choose to stay under the same thread group ID, or parent thread group ID, share
open files, memory, etc.

A process is a nonempty set of threads that share the same thread group ID. A
process is identified by its thread group ID. This is what is usually referred
to as the PID in user space. The system calls \texttt{getpid()} and
\texttt{getppid()} return the thread group ID and the parent thread group ID,
respectively.

A thread of execution is also called a task.
