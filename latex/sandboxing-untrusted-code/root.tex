% vim: set spell:

\chapter{Sandboxing untrusted code}

\begin{quotation}

\footnotesize\sffamily\itshape

\begin{flushright}

Going all the way back to early time-sharing systems, we systems people
regarded the users, and any code they wrote, as the mortal enemies of us and
each other. We were like the police force in a violent slum.

\smallbreak

\upshape

--- ROGER NEEDHAM, IEEE Symposium on Security and Privacy (1999)

\end{flushright}

\end{quotation}

Students submit digital files in response to assignments. Some of these files
may specify executable computer programs. The automatic evaluation of student
submissions constitutes the static and dynamic evaluation of such files. Static
evaluation constitutes executing computer programs specified by the teaching
staff, which read and analyze student files. In addition, dynamic evaluation
includes executing the programs submitted by students.

The student programs may misbehave in a myriad of different ways. The programs
of the teaching staff, although more trustworthy, may also misbehave. If
nothing else, they may undermine the misbehaviour of students. The intent of
this chapter is to discuss the means in which we can mitigate for such
misbehaviour for all parties, and ensure fair service.

In the first section we provide a high-level overview of the technologies that
can be used sandboxing. Here we come to the conclusion that operating-system
level virtualization is a best candidate option. The remainder of the chapter
deals with basic principles in the Linux kernel (version 3.14) that provide
various means of virtualizing and limiting system resources.

% vim: set spell:

\section{Technology Overview}

A program is executed within a program execution environment. A sandboxed
execution environment ensures the non-interference of the program with other
programs being executed on the host.

There are two general approaches to providing sandboxed program execution
environments: sandboxing the operating system, or sandboxing within an
operating system. The first two subsections discuss the former, the last
subsection discusses the latter.

\subsection{Dedicated Servers}

We can provide an off-the-shelf operating system sandbox using a dedicated
server for every program. This however, relinquishes remote control of the
execution environment, and may demand physical access to the machine in case of
failure. This is impractical. Also, this is expensive since most computer
systems today are intended as time-sharing systems.

\subsection{Virtual Machines}

The next option is to provide an operating system by means of hardware or
software virtualization. This retains remote control of the execution
environment. However, it imposes huge costs on every execution. An entire
operating system has to boot up before testing can commence.

Alternatively a pool of virtual machines could be kept online, pulling tasks
from a task queue. Such a set up does not always fail fast, again because an
entire operating system may have to be rebooted in case of failure. Combined
with empirical evidence that student programs fail often, this is impractical.

Such high-level virtualization also makes the execution environment hard to
monitor. The overhead of the operating system may dilute the true costs
inherent in executing various programs. For similar reasons, fine-grained
resource limits are often hard to enforce. All this is desirable for the
purposes of evaluating our programs and tuning our sandboxes.

\subsection{Operating system-level virtualization}

Operating system-level virtualization alleviates the need for a separate kernel
for sandboxing program execution environments.

Time-sharing systems have for a long time provided for multiple simultaneous
user space instances on top of a single kernel. Combined with file-system user
permissions and user groups, these provided for the very first sandboxing
capabilities.

Recent developments in modern operating systems have facilitated more
fine-grained sandboxing by virtualizing underlying system resources. Such a
virtualized user space instance is typically called a ``jail'' or a
``container''.

The pitfall of operating system-level virtualization in general, is that we
become more vulnerable to vulnerabilities in the kernel. If a contained program
can utilize a kernel vulnerability, the whole system is under threat.

\subsubsection{FreeBSD Jails}

\subsubsection{Linux Kernel Containment}

\begin{itemize}

\item LXC

\item libvirt-lxc

\item Docker

\end{itemize}



% vim: set spell:

\section{Control Groups}

Control groups (cgroups) provide a mechanism of hierarchically
grouping/partitioning tasks (see also
\referToAppendix{general-linux-concepts:tasks}) and their future
children\cite{cgroups-cgroups-since-linux-3.14}. On their own, cgroups are
perhaps only useful for simple job tracking. The idea, is to have other
subsystems hook into the cgroups functionality and provide accounting or
limiting of various system resources.

The standard cgroup subsystems include subsystems to monitor and limit memory,
CPU time, I/O, and device activity. Most modern Linux distributions come with
control groups and many of these canonical subsystems enabled. The system's
\texttt{/proc/config.gz} can reveal the setup on your system\cite{man-5-proc}.
If \texttt{CONFIG\_CGROUPS} is enabled, you have cgroups support.

The variables related to various subsystems are explored further in the
following subsections. First however, we discuss how cgroups can be accessed
and manipulated from user space in general.

\subsection{Managing cgroups}

Cgroups are managed via a pseudo-filesystem: cgroups reside in memory, but can
be manipulated through the virtual file system.  \texttt{cgroup} is therefore
an inherent file system type on systems that have the cgroups functionality
enabled.

\subsubsection{Cgroups, subsystems, and hierarchies}

A cgroup is an association of a set of tasks with a set of preferences for one
or more subsystems.  A hierarchy is a set of cgroups arranged in a tree; every
task in the system is contained in exactly one cgroup of a hierarchy; all
cgroups in a hierarchy, associate their tasks with the same set of preferences.

We say that a hierarchy is ``associated'' with a set of subsystems, and that a
task is ``attached'' to a cgroup.

\subsubsection{Mounting}

When mounting a cgroup file system, we create a new hierarchy. The set of
subsystems to associate with the hierarchy is listed as mount options:

\begin{lstlisting}
$ mount -t cgroup -o cpu,cpuacct cgroup ./cgroup/cpu,cpuacct
\end{lstlisting}

This associates the \texttt{cpu} and \texttt{cpuacct} subsystems with a new
hierarchy, and mounts the hiearchy under the target
\texttt{./cgroup/cpu,cpuacct}, unless one of the subsystems is busy.

A subsystem is busy e.g. if it is associated with a hierarchy having tasks
attached.  Since a hierarchy is automatically attached to all tasks in the
system, this effectively means that a subsystem may be associated with at most
one hierarchy. If a hierarchy assosciated with the exact same set of subsystems
already exists however, it will be reused for the new mount.

What hierarchies already exist, and what subsystems they are associated with,
depends on the system at hand. The system's \texttt{/proc/mounts} can reveal
how this is setup on your system\cite{man-5-proc}.

\subsubsection{Tasks and nested cgroups}

After a hierarchy is successfully mounted, we see a range of files, and perhaps
folders, below our target. We monitor/modify the preferences of a cgroup by
monitoring/modifying these files. We create/remove child cgroups by
creating/removing subdirectories.

With a few exception at the root of the hierarchy, all cgroups contain the same
files, created when the cgrpup is created. Some files are common to all
hierarchies, others are due to the associated subsystems. Two files are of
particular interest:

\begin{description}[\setleftmargin{0.2in}\breaklabel\setlabelstyle{\tt}]

\item [cgroup.procs] Lists the set of thread group IDs in the current cgroup.
Appending a thread group ID to this file moves all the threads in the thread
group into this cgroup.

\item [tasks] Lists the set of thread IDs in the current cgroup. Appending a
thread ID to this file moves the thread into this cgroup.

\end{description}

\subsection{memory}

\subsection{cpuacct}

\subsection{cpu}

\subsection{cpuset}

\subsection{devices}

\subsection{blkio}


% vim: set spell:

\section{Namespaces}

The purpose of a Linux namespace is to abstract over a global resource, and
make it appear to processes within the namespace, as though they have their own
isolated instance of the global resource. Various types of namespaces allow to
abstract over various global resources.

A task can be associated with a namespace using an \cite{man-2-unshare},
\cite{man-2-setns}, or \cite{man-2-clone} system call. The first disassociates
the process from a namespace, associating it with a new namespace. The second
reassociates the process with an existing namespace. The last is the general
system call for task creation, allowing to create a task in a new namespace.

The namespaces that a task is associated with are identified in the
\cite{man-5-proc} pseudo file system. The general pattern of the file names is
\texttt{/proc/[pid]/ns/[nstype]}, where \texttt{[pid]} is the thread group
identifier of the task (see also
\referToAppendix{general-linux-concepts:tasks}), and \texttt{[nstype]} is one
of a range of supported namespace types.

We discuss some of these types in the following sections. Support for more
types may come in the future, as containers demand more resource isolation.
Furthermore, not all of these types are necessarily enabled on your system. The
user namespace is frequently omitted by many distributions as it opens up a
large part of the kernel, previously not available to the non-privileged user.
Some believe it requires a lot more testing before being enabled by default.

To our knowledge, enabling a namespace type requires compiling your own kernel.
You can check which namespaces are enabled on your system, by reading the
\texttt{/proc/config.gz} file, or listing the files in a process namespace
subdirectory. For instance,

\begin{lstlisting}
$ ls /proc/$$/ns
ipc  mnt  net  pid  uts
\end{lstlisting}

\subsection{mnt}

The mount(mnt) namespace abstracts over the mount points of a system. This
allows for processes in different namespaces to have different views of the
file system. Within a container, we can unmount points that are perhaps needed
by the host, but not by the container, and would perhaps make the host
vulnerable, if the container had access to them.

\subsubsection{Pivot root}

One particularly useful application of mount namespaces is pivoting the file
system root to some other point in the file system using
\cite{man-2-pivot-root}. Pivoting the root in a container does not affect the
host, or other containers. At the same time, pivoting the root moves all the
dependencies on the old root, to a new root within the container.

This allows us to subsequently unmount the old root, provided that the new root
does not depend on this mount point. This can be achieved by having the new
root mounted as a \texttt{tmpfs}, or perhaps a read-only \texttt{squashfs}.
This hides the original root file system in a matter similar to
\cite{man-2-chroot}, but makes reestablishing the old root slightly more
cumbersome, since the old root first has to be properly remounted first.

\subsection{uts}

The UNIX Time-sharing System(UTS) namespace abstracts over the host- and domain
name of a system. This allows each container to retain a personal host- and
domain name, perhaps different from the underlying host.

\subsection{ipc}

\subsection{pid}

\subsection{net}

\subsection{usr}


% vim: set spell:

\section{Resource Limits}

\label{section:sandboxing:resource-limits}

The system call \cite{man-2-getrlimit}, and its siblings, \texttt{setrlimit(2)}
and \texttt{prlimit(2)}, can be used to manage per-user soft and hard limits on
various resources. In general, a process is warned upon reaching a soft limit,
and killed, or prohibited in acquiring more of that resource, upon reaching a
hard limit.

An unprivileged user can freely change their soft limit to any value between 0
and the hard limit, or irreversibly lower their hard limit. A privileged user
can freely change either value. Limits are enforced per user session, and are
enforced throughout the lifetime of a user session.

We discuss some interesting
possible limits below. Some further technical details are discussed in
\referToAppendix{resource-limits}.

\begin{description}[\setleftmargin{0.2in}\breaklabel\setlabelstyle{\tt}]

\item[RLIMIT\_NPROC] A limit on the number of tasks a user can create per session.

\item[RLIMIT\_CPU] A limit on the total CPU time per process.

\item[RLIMIT\_CORE] A limit on the size of the core file.

\end{description}

\texttt{RLIMIT\_NPROC}, combined with \texttt{RLIMIT\_CPU} and a limit on the
wall-clock time, can be used to mitigate for fork bombs, busy, and long-running
processes. A fork bomb is a process that recursively creates new tasks in
attempt to cause a denial-of-service. A limit on the wall-clock time ensures to
kill off long-running processes which consume little to no CPU time, e.g.
interactive processes waiting unduly for user input.

Putting a limit on the core file size can be a good idea to prevent students
from inspecting us inspecting the runtime of their programs. This may reveal
certain aspects of the system, which they in turn may use to exploit it.

There are other interesting limits, but combined with control groups, we have
found these to be of little use.


% vim: set spell:

\section{Linux Security Modules}

Linux Security Modules (LSM) is an access control framework, providing a
mechanism for various security checks to be hooked by various kernel
extensions\cite{wright-et-al-2002,lsm.txt}. The main use of LSM is the
implementation of mandatory access controls.

\subsection{SELinux}

\subsection{AppArmor}

\subsection{Capabilities}



% vim: set spell:

\section{Seccomp}

Student programs may often be expected to use but a handful of system calls.
Whitelisting the syscalls that students may use provides both for making sure
that they use the intended tools, and that student programs cannot abuse
obscure, or even vulnerable parts of the Linux kernel.

The secure computing (seccomp) part of the Linux kernel is provides for such
syscall filtering\cite{seccomp-filter.txt}.

Seccomp was originally conceived by Andrea Arcangeli\cite{arcangeli-2005} while
working on the cpushare project, a service that would let you to contribute
your idle CPU, for those in need of CPU time\cite{lwn.net-2005}. This requires
executing arbitrary code on your computer, and so limiting the allowed syscalls
to read/write/exit/sigreturn was a simple sandboxing idea. The Chromium team
has since worked on extending this functionality\cite{tinnes-2012} with an
approach inspired by Berkley Packet Filters\cite{drewry-2012}.

Seccomp filters can be applied using the \cite{man-2-prctl} system call. We can
use this to specify a filter program, that given a particular syscall will
reply whether the syscall should be executed or not. Some sample filter
programs can be found in \cite{linux-kernel-v3.14.2}, under
\texttt{./samples/seccomp/}.


