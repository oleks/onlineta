% vim: set spell:

\chapter{Introduction}

\begin{quotation}

\footnotesize\sffamily\itshape

\begin{flushright}

In any teaching of the application of computers it is essential to have the
students do practical programming problems and to grade their results. Such
grading should consider both the formal correctness and the performance of the
programs and tends to become difficult and time consuming as soon as the
teaching is beyond the most elementary level.  The possibility of using the
computer to help in this task therefore soon suggests itself.

\smallbreak

\upshape

--- PETER NAUR, {\itshape BIT 4} (1964)

\end{flushright}

\end{quotation}

\section{Reader Expectations}

The reader is assumed to be familiar with the concept of a university, i.e. an
institution of higher education and research, aimed at educating scholars and
professionals, granting them degrees, signifying their accomplishments.

The reader is assumed to be familiar with Computer Science, i.e. the study of
computable processes and structures, with the aid of of computers. Preferably,
the reader should hold a Computer Science degree, be, or have been enrolled in
a Computer Science university programme.

\section{Dictionary and Grammar}

Unless otherwise stated, the reader

\section{Legal Disclaimer}

This report references certain legal documents, including Danish laws and
university curricula. As the author is not trained in law, there is no claim as
to the legal soundness of the claims and references made in this report.

A solid attempt has been made at retaining the formulation of the referenced
material, and referencing the most current legal documents, unless this was
hindered by other legal references.

For instance, the shared section of the BSc and MSc curricula for study
programmes at the Faculty of Science, University of
Copenhagen\cite{faelles-del-2013} is based on Ministerial Order no. 819 as of
June 29, 2010 \cite{bek814}. This document is outdated and has been updated
twice, most recently by Ministerial Order no.  1520 as of December 16,
2013\cite{bek1520}. In this particular case, it is the faculty curricula that
was been deemed to mandate the relevant law to reference.
