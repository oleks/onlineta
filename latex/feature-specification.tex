% vim: set spell:

\chapter{Feature Specification}

% simplify teachers' assistants to just assistants.

% What is a course? It takes time.

\section{User Roles}

% (online TA is a bad name, since the service does more than a usual TA would,
% or at least, the online TA has more rights than a physical TA would.)

For an online TA there are two principal user roles: \emph{system
administrators}, \emph{stakeholders} (students, teachers, and the general
public). (TODO: is the general public really a stakeholder?) 

% is the contents not visible to the rest of the world?

% some students may be reluctant to participate in public fora.

System administrators have complete control of the servers running the service,
including the mechanisms that enforce the security policies of the service.
(They can, in principle, change anything.) System administrators are also
responsible for monitoring the service for abnormal or abusive behaviour.

% (Ignoring or limiting system administrators would require ignoring or
% limiting anyone with the root access to the machine. This seems rather
% pointless.)

Stakeholders access the system through a secure online interface. They set up,
enroll participants, and participate in courses --- where a course is a
collection of programming assignments for the students to solve.

% enroll students by KU-email.

There are four principle course-specific stakeholder roles: \emph{instructors},
\emph{assistants}, \emph{students}, and \emph{non-participants}. That is, for
each course, every stakeholder has exactly one of these four roles.

Instructors are the principle course administrators. They can enroll other
participants as either instructors, assistants, or students. Their primary role
is to define the elements of a course.

Assistants are second-level course administrators. They can enroll other
participants as either assistants or students. Their primary role is to assist
the instructors in defining the elements of a course, and to provide feedback
to students.

Students are the basic users of a course. They cannot enroll other
participants, or tamper with the elements of a course. Their primary role is to
submit solutions to the programming assignments of a course. Student
submissions are individual or group-based, and so not visible to other students
in general. Student submissions are visible to instructors and assistants for
the sake of evaluation.

Non-participants can see the elements of a course, but they cannot participate
in the course in any way. The reasoning here is that there is no implicit trust
relationship among the participants of a course, that does not exist among all
the stakeholders of an institution. Providing access to non-participants allows
them to get a feel for the course, which is useful for both encouraging
participation and course evaluation.

\begin{itemize}

\item Only instructors and assistants can define the elements of a course.

\item Only students can make submissions to programming assignments.

\item Course contents is open for everyone in the system to see.

\end{itemize}

% When a teacher hires an assistant, the teacher sets up a trust relationship
% with the assistant. The assistant can do anything a teacher can, except kick
% teachers off the course.

% In principle we can let students set up their own courses, but this may incur
% additional complexities wrt. monitoring that don't seem to be worth dealing
% with at the moment.

% anybody can view the elements of a course, but only enrolled students can
% submit solutions to programming assignments.

% what can the instructors and assistants see, what can the students, and the
% general public see?

