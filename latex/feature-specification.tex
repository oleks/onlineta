% vim: set spell:

\chapter{Feature Specification}

% simplify teachers' assistants to just assistants.

% What is a course? It takes time.

\section{User Roles}

(online TA is a bad name, since the service does more than a usual TA would, or
at least, the online TA has more rights than a physical TA would.)

For an online TA there are two principal user roles: \emph{system
administrators} and \emph{users}.

System administrators have complete control of the servers running the service,
including the mechanisms that enforce the security policies of the service.
(They can, in principle, change anything.) System administrators are also
responsible for monitoring the service for abnormal or abusive behaviour.

(Ignoring or limiting system administrators would require ignoring or limiting
anyone with the root access to the machine. This seems rather pointless.)

Users access the system through a secure online interface. They set up, hold,
enroll to, and participate in \emph{courses} --- a collection of programming
assignments for the enrollees to solve.

There are three principle course-specific user roles: \emph{instructors},
\emph{assistants}, and \emph{students}. Instructors and assistants set up
assignments for the students to solve, as well as keep an eye on student
progress.

Teachers set up, and hold courses. Students enroll to and follow courses. A
course is a collection of programming assignments for the enrolled students to
solve. Some students may be enrolled on a course as teacher's assistants. Their
task is to help the teachers in holding the course.

In what follows, we will distinguish between 

Users use the web application. Users can set up courses and be enrolled for
courses. A user that sets up a course, is trivially enrolled for that course.

Users have additional course roles. A user may be a \emph{teacher}, an
\emph{assistant}, or a \emph{student}. A user that sets up a course is
automatically enrolled as a teacher on that course. A teacher can enroll any
other user as either a teacher, an assistant, or a student.

A user cannot change their own course role. Only a supperior

which yields them a
teacher on that course. They can then enroll other users as either
\emph{teachers}, \emph{assistants}, or \emph{students}.

When a teacher hires an assistant, the teacher sets up a trust relationship
with the assistant. The assistant can do anything a teacher can, except kick
teachers off the course.

In principle we can let students set up their own courses, but this may incur
additional complexities wrt. monitoring that don't seem to be worth dealing
with at the moment.
