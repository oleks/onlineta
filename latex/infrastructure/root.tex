% vim: set spell:

\chapter{Infrastructure}

\begin{quotation}

\footnotesize\sffamily\itshape

\begin{flushright}

\selectlanguage{russian}

Тяжело в учении, легко в бою; легко в учении, тяжело в бою.

\selectlanguage{english}

\smallbreak

(Tough in training, easy in battle; easy in training, tough in battle.)

\smallbreak

\upshape

--- ALEXANDER SUVOROV, Generalissimo of the Russian Empire (1729--1800)

\end{flushright}

\end{quotation}

This chapter provides a high level overview of the chosen system architecture.
We justify our choices, as well as discusses some of their benefits and
downsides.

Our current approach keeps students and staff close to the command line (as
opposed to e.g. a web-based interface). The reason is part idealogical and part
practical. We believe that students become better programmers when kept close
to the command line. At the same time, it is easier to implement an
infrastructure where the integrity and the authenticity of the data can be
ensured. A myriad of tools exist,

Overall, there are three types of servers involved: a key server, a Git server,
and a test server. The communications between them and the students and
teaching staff are roughly illustrated by \referToFigure{infrastructure}. Each
type of server may in reality be a set of interconnected servers balancing
loads among each other, but fulfilling together the role of a single server
type.



\includeFigure[0.75]{infrastructure}{Illustration of some of the system
architecture. All communication happens over secure channels. Both students and
staff interact with the key server and Git server, but students have
considerably less permissive rights on each of the servers}

% vim: set spell:

\section{Key Server}

The purpose of the key server is to provide a centralized registry of the
public keys of students and staff. It can serve as a certificate authority in a
formal or semi-formal sense (see also
\referToSection{gitolite-ssh-certificate-authority}).

\subsection{Serving Public Keys}

\subsection{Staff Key Registration}

\subsection{Student Key Registration}


% vim: set spell:

\section{Git Server}

The purpose of the Git server is to serve as a general purpose data store for
both course content, assignments, and student submissions. It serves as a
gateway between students and staff, allowing for staff to publish course
content and assignments, and for students to make submissions.

\subsection{Why Git?}

Git is a popular\cite{ohloh-2014}, free, and open source distributed version
control and source code management system\cite{git-2014}. Although perhaps not
the ideal system for all intents and purposes, it is an excellent example that
has cemented itself in both the open source community, academia and
industry\cite{gitprojects-2014}.

Version control and code review are some of the Core- Tier1 and Tier2 elements
in \cite{cs-curricula-2013}, meaning that they are highly suggested topics for
any undergraduate Computer Science programme.

We hypothesize that using Git for programming assignments can spur the learning
of some of the workflow of modern software development. Ideally, students
collaborate on assignments, while the teaching staff offer code review, as if
it were a real software development project.

\subsection{Course as a Repository}

A Git server manages Git repositories. We chose to let a course be represented
by a Git repository.

A Git repository has one or more branches. We choose to let one branch - the
master branch - be used for the distribution of course content and assignments
by teaching staff. To make submissions, students create branches in their name
and push their changes to these branches onto the server.

In such an infrastructure it is important that students are not allowed to push
to the master branch, or to other student branches. At the same time, teaching
staff should be allowed to push to both the master (to provide content and
assignments) and student branches (to provide feedback). Last but not least, we
would like to let everyone see that course content and assignments, but
prohibit them in seeing student submissions, or pushing to any of the branches.

Such fine-grained authentication and authorisation can be achieved through
OpenSSH and Git hooks.

\subsection{OpenSSH}

OpenSSH is a set of open source computer programs, allowing users to use a
secure encrypted connection over an insecure network to log in and execute
commands on a remote machine.

One of the authentication methods supported by OpenSSH is public key
cryptography. This cryptographic scheme requires two separate keys: a private
key and a public key. The public key is used to encrypt data or verify a
digital signature, while the private key is used to decrypt data or generate a
digital signature.

\subsection{Git Hooks}

Git hooks is a Git mechanism for executing custom scripts when important events
happen. From the point of view of a server the important events include anyone
reading from or writing to a repository.

\subsection{Gitolite}

Gitolite is an access control layer on top of Git\cite{gitolite-com-2014a}.
Gitolite leverages the features of OpenSSH and Git hooks, as discussed above,
to provide fine-grained authentication and
authorisation\cite{gitolite-com-2014b}.

Gitolite is used in multiple communities with high-stakes projects, such as
Fedora, KDE, Gentoo, and kernel.org\cite{gitolite-com-2014c}. Among the reasons
for choosing gitolite, kernel.org lists\cite{kernel-org-2014} ``well maintained
and supported code base'', ``responsive development'', ``broad and diverse
install base'', and ``had undergone an external code
review''\cite{gitolite-google-group-2011}.

There are also other tools out there, such as  Gerrit\footnote{See also
\url{https://code.google.com/p/gerrit/}.} and Stash\footnote{See also
\url{https://www.atlassian.com/software/stash}.}. Both of these provide a lot
more than a simple access control layer.

In conclusion, we chose to use gitolite ahead of both using other tools, and
implementing our own solution.

\subsection{Installation}

Pull the most recent version of git from your package manager.

Create a git user with shell /bin/git-shell (or similar).

http://git-scm.com/docs/git-shell

Make sure he's passwordless.

Set up .ssh/authorized\_keys check permissions.

http://stackoverflow.com/questions/6377009/adding-public-key-to-ssh-authorized-keys-does-not-log-me-in-automatically

Permissions: http://www.openssh.com/faq.html\#3.14




% vim: set spell:

\section{Encrypted and Signed Git Repositories}

Git comes with a built-in a method to digitally sign tags (versions) and
commits. The intent is to allow us to ensure that particular tags (versions) or
commits come from particular individuals\cite{gerwitz-2013}.

We could use this to ensure that a particular submission indeed comes from a
particular student, but we cannot expect the students to be proficient enough
in Git to use tags and signed commits. Instead, we consider Git repositories
where commits are always, seamlessly signed.

At the core of Git is a key-value store\cite{chacon-2009}. All
version-controlled data, is retained in so-called ``Git objects'', addressed by
the SHA-1 hash of their contents. We ``commit'' data to, and ``checkout'' data
from this data store.

Git retains diffs of files (compared to their earlier versions), rather than
files themselves. A file in a local Git repository is therefore saved twice,
once as a regular file on the file system, and once as a sequence of diffs,
retained as Git objects. When performing a Git push operation to a remote Git
repository, only Git objects are ever sent along.

The \texttt{filter} Git attribute \cite{man-5-gitattributes} allows us to
filter the data before it is committed to, and filter it when it is checked out
of the data store. This is achieved by two commands, ``smudge'' and ``clean'',
respectively. The names imply that the intended use is to retain only ``clean''
data in the data store.

We can abuse the system a bit, taking inspiration from its own
documentation\cite{man-5-gitattributes}. We sign the data on commit and verify
the signature upon checkout. To do this, we append the signature to the data on
commit, and parse out the signature on checkout. We can also specify that a
filter is required, i.e.  must exit with a zero exit code, for a commit or
checkout to succeed. This way, we can fail if we can't verify the signature.

Similarly, we can encrypt the data that goes into the data store, and decrypt
the data when it leaves the data store. This way, we can make sure that student
data stays safe, even if Git objects are ever leaked from our Git server.

\subsection{Keys}

GNU Privacy Guard (GPG) is a suite of cryptographic tools. It supports, among
other things, private/public key cryptography, and boasts tools for trustworthy
distribution of public keys. GnuPG is often used to identify physical
individuals on the Internet. Indeed, the built-in Git tag and commit signing
features expect users to retain a GPG private/public key pair, and securely
distribute their public keys to all interested parties.

OpenSSL is an open-source implementation of the SSL and TLS protocols, used for
secure communication on the Internet. Unlike, GnuPG, OpenSSL is not intended
for the identification of physical individuals. OpenSSL however, is often used
to identify users on servers. For instance, OpenSSL is an underlies OpenSSH,
which we used to identify users on our Git server (see also
\referToSection{openssh}).

Both GnuPG and OpenSSL can be used to implement encrypted and signed Git
repositories. GnuPG however is more conventional for the identification of
individuals, and it is compatible with OpenSSH\cite{man-1-gpg-agent}. OpenSSL
is also very badly documented\footnote{\cite{man-1-openssl} does not even
mention that it can be used to encrypt, decrypt and sign data.} and has
recently suffered a severe drop in popularity due to a major
vulnerability\cite{cvedetails-com-2014c}.

Students can make due with securely informing us of their public GPG key, and
use this to both interact with our Git server, encrypt, and sign their
submissions.

We have not found a way to make public key distribution any more simple for
students, than using some kind of a key server, where keys are validated with
the participation from the teaching staff. One easier approach might be to make
use of their online university accounts and have them supply their keys on
their own in a way similar to Github. We have not had time to look into this.

\subsection{Collaboration}

It is easy enough to encrypt the data for one user, but this defeats many of
the useful collaboration features of Git.

One way in which we can provide for collaborative, encrypted Git repositories
is to encrypt the data for multiple recipients. This is straight-forward using
GPG tools\cite{man-1-gpg2}. The pitfall of this is that new collaborators
cannot see the work done before they joined the project, unless we resort to
rewriting history.

For an implementation of this, see the attached \texttt{./src/git-init.sh}.


% vim: set spell:

\section{Test Server}

The purpose of the of the test server is to test student submissions.

