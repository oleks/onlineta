% vim: set spell:

\section{Git Server}

The purpose of the Git server is to serve as a general purpose data store for
both course content, assignments, and student submissions. It serves as a
gateway between students and staff, allowing for staff to publish course
content and assignments, and for students to make submissions.

\subsection{Why Git?}

Git is a popular\cite{ohloh-2014}, free, and open source distributed version
control and source code management system\cite{git-2014}. Although perhaps not
the ideal system for all intents and purposes, it is an excellent example that
has cemented itself in both the open source community, academia and
industry\cite{gitprojects-2014}.

Version control and code review are some of the Core- Tier1 and Tier2 elements
in \cite{cs-curricula-2013}, meaning that they are highly suggested topics for
any undergraduate Computer Science programme.

We hypothesize that using Git for programming assignments can spur the learning
of some of the workflow of modern software development. Ideally, students
collaborate on assignments, while the teaching staff offer code review, as if
it were a real software development project.

\subsection{Course as a Repository}

A Git server manages Git repositories. We chose to let a course be represented
by a Git repository.

A Git repository has one or more branches. We choose to let one branch - the
master branch - be used for the distribution of course content and assignments
by teaching staff. To make submissions, students create branches in their name
and push their changes to these branches onto the server.

In such an infrastructure it is important that students are not allowed to push
to the master branch, or to other student branches. At the same time, teaching
staff should be allowed to push to both the master (to provide content and
assignments) and student branches (to provide feedback). Last but not least, we
would like to let everyone see that course content and assignments, but
prohibit them in seeing student submissions, or pushing to any of the branches.

Such fine-grained authentication and authorisation can be achieved through
OpenSSH and Git hooks.

\subsection{OpenSSH}

OpenSSH is a set of open source computer programs, allowing users to use a
secure encrypted connection over an insecure network to log in and execute
commands on a remote machine.

One of the authentication methods supported by OpenSSH is public key
cryptography. This cryptographic scheme requires two separate keys: a private
key and a public key. The public key is used to encrypt data or verify a
digital signature, while the private key is used to decrypt data or generate a
digital signature.

\subsection{Git Hooks}

Git hooks is a Git mechanism for executing custom scripts when important events
happen. From the point of view of a server the important events include anyone
reading from or writing to a repository.

\subsection{Gitolite}

Gitolite is an access control layer on top of Git\cite{gitolite-com-2014a}.
Gitolite leverages the features of OpenSSH and Git hooks, as discussed above,
to provide fine-grained authentication and
authorisation\cite{gitolite-com-2014b}.

Gitolite is used in multiple communities with high-stakes projects, such as
Fedora, KDE, Gentoo, and kernel.org\cite{gitolite-com-2014c}. Among the reasons
for choosing gitolite, kernel.org lists\cite{kernel-org-2014} ``well maintained
and supported code base'', ``responsive development'', ``broad and diverse
install base'', and ``had undergone an external code
review''\cite{gitolite-google-group-2011}.

There are also other tools out there, such as  Gerrit\footnote{See also
\url{https://code.google.com/p/gerrit/}.} and Stash\footnote{See also
\url{https://www.atlassian.com/software/stash}.}. Both of these provide a lot
more than a simple access control layer.

In conclusion, we chose to use gitolite ahead of both using other tools, and
implementing our own solution.

\subsection{Installation}

Pull the most recent version of git from your package manager.

Create a git user with shell /bin/git-shell (or similar).

http://git-scm.com/docs/git-shell

Make sure he's passwordless.

Set up .ssh/authorized\_keys check permissions.

http://stackoverflow.com/questions/6377009/adding-public-key-to-ssh-authorized-keys-does-not-log-me-in-automatically

Permissions: http://www.openssh.com/faq.html\#3.14


