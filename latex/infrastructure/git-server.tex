% vim: set spell:

\section{Git Server}

The purpose of the Git server is to serve as a general purpose data store for
both course content, assignments, and student submissions. It serves as a
gateway between students and teaching staff, allowing for teaching staff to
publish course content and assignments, and for students to make submissions.

\subsection{Why Git?}

Git is a popular\cite{ohloh-2014}, free, and open source distributed version
control and source code management system\cite{git-2014}. Although perhaps not
the ideal system for all intents and purposes, it is an excellent example that
has cemented itself in both the open source community, academia and
industry\cite{gitprojects-2014}.

Version control and code review are some of the Core- Tier1 and Tier2 elements
in \cite{cs-curricula-2013}. They are highly suggested topics for any
undergraduate Computer Science programme.

We hypothesize that using Git for programming assignments can spur the learning
of some of the workflow of modern software development. Ideally, students
collaborate on assignments, while teaching staff offer code reviews, all as if
it were a real software development project.

\subsection{Course as a Repository}

A Git server manages Git repositories. We choose to let a course be represented
by a Git repository.

A Git repository has one or more branches. We choose to let one branch - the
master branch - be used for the distribution of course content and assignments
by teaching staff. To make submissions, students create branches in their name
and push their changes to these branches onto the server.

Assessment of a student submission is provided in a special subdirectory on
their private branch. All assessment is bound to particular commit by the
student to a student branch.

In such an infrastructure it is important that students are not allowed to push
to the master branch, or to other student branches. At the same time, teaching
staff should be allowed to push to both the master (to provide content and
assignments) and student branches (to provide feedback). Last but not least, we
would like to let everyone see course content and assignments, but prohibit
them in seeing student submissions, or pushing to any of the branches.

Such fine-grained authentication and authorisation can be achieved through
OpenSSH and Git hooks.

\subsection{OpenSSH}

OpenSSH is a free (as in free speech) version of the SSH connectivity
tools\cite{openssh-com-2014}.  The tools provide for secure encrypted
communication between untrusted hosts over an insecure network\cite{man-1-ssh}.
They include tools for user authentication, remote command execution, file
management, etc.

An OpenSSH host maintains a private/public key pair used to identify the host.
Upon connection, the host offers its public key to the client, in hope that the
client will accept it and (securely) proceed with authentication with the host.
If authenticated, the client is mapped to a particular user on the host.  After
some session preparation, the client, as that user, can start a session, i.e.
request a shell or the execution of a command.

One of the authentication methods supported by OpenSSH is using public key
cryptography. The idea is that each client creates a private/public key pair,
and informs the host of the public key over some otherwise secure channel, e.g.
using a trusted keyserver.

For any user on the host, a file can be created, e.g.
\texttt{\textasciitilde/.ssh/authorized\_keys}, listing the public keys of
those private/public key pairs that may be used to authenticate as that user.
The format of this file\cite{man-8-sshd}, allows to specify additional options
for each key. The options can be used to e.g. set a session-specific
environment variable, or replace the command executed once the user is
authenticated. The original command is then saved as the environment variable
\texttt{SSH\_ORIGINAL\_COMMAND}.

When using a Git server with OpenSSH, Git operations on the client, will
attempt execute Git operations on the host. Per-key options can be used to make
their execution dependent on the key used for authentication, e.g. performing
authorisation.

\subsection{Git Hooks}

Git hooks is a Git mechanism for executing custom scripts when important events
happen\cite{man-5-githooks}. The scripts can control in how far certain Git
operations succeed. A Git hook is an adequately named executables placed in a
special subdirectory in the local Git repository. Git hooks are not part of the
version-controlled code base.

For instance, the \texttt{update} hook is executed whenever the client attempts
to push something to a branch. The client has already been authenticated, but
no changes have yet been made. The hook is passed adequate arguments to
identify the branch or tag being updated and the update taking place. If this
hook exits with a non-zero exit value, the update will duly fail.

\subsection{Users}

When using a Git server with OpenSSH, clients must be mapped to users on the
host. There are at least two options for the mapping: each client gets their
own user, or all clients map to the same user. The first option has a higher
administration costs, but gives perhaps more fine grained access control.

The second option is generally more popular because of less cluttering of the
UTS namespace. Additional tools, like gitolite, are instead used to provide a
fine-grained access control layer. We too, have chosen this option.

\subsection{Gitolite}

Gitolite is an access control layer on top of Git\cite{gitolite-com-2014a}.
Gitolite leverages the features of OpenSSH and Git hooks, as discussed above,
to provide fine-grained authentication and
authorisation\cite{gitolite-com-2014b}.

Gitolite is used in multiple communities with high-stakes projects, such as
Fedora, KDE, Gentoo, and kernel.org\cite{gitolite-com-2014c}. Among the reasons
for choosing gitolite, kernel.org lists\cite{kernel-org-2014} ``well maintained
and supported code base'', ``responsive development'', ``broad and diverse
install base'', and ``had undergone an external code
review''\cite{gitolite-google-group-2011}.

There are also other tools out there, such as  Gerrit\footnote{See also
\url{https://code.google.com/p/gerrit/}.} and Stash\footnote{See also
\url{https://www.atlassian.com/software/stash}.}. Both of these provide a lot
more than a simple access control layer.

In conclusion, we chose to use gitolite ahead of both using other tools, and
implementing our own solution.

\subsubsection{Installation}

Gitolite is installed on a per-user basis. Meaning that we should create and
log in as some designated Git user to set up gitolite, or change ownership
accordingly after install. As an extra security assurance, the gitolite
installation does not require a privileged user, so long as Git, OpenSSH, and
Perl are already installed.

The code is distributed under a GNU General Public License, and is available at
\url{git://github.com/sitaramc/gitolite}. We may wish to check out the latest
tag (version), after verifying that it indeed was signed by Sitaram Chamarty
(the original developer of Gitolite)\footnote{See also
\url{http://git-scm.com/book/en/Git-Basics-Tagging}, if you are unfamiliar with
Git's tagging mechanism.}. To the best of our knowledge, his public GPG key is:

\begin{lstlisting}
pub   4096R/088237A5 2011-10-25
      Key fingerprint =
        560A DA64 7542 816F 412E  5891 A442 9085 0882 37A5
uid       Sitaram Chamarty (work email) <sitaram@atc.tcs.com>
uid       Sitaram Chamarty <sitaramc@gmail.com>
sub   4096R/8AC76EFB 2011-10-25
\end{lstlisting}

Once cloned and compiled, gitolite setup requires the administrator's public
SSH key to be provided in some accessible file:

\begin{lstlisting}
./gitolite setup -pk admin.pub
\end{lstlisting}

This initializes a git repository \texttt{gitolite-admin.git}, which
\texttt{admin} has complete control over. This repository serves as the primary
administrative interface for the gitolite access control layer.

\begin{lstlisting}

\end{lstlisting}

\subsubsection{Administration}

Administration of the gitolite happens through a special Git repository. There
are three important elements to this repository:

\begin{enumerate}

\item The \texttt{./conf/gitolite.conf} configuration file. This file defines
the users, repositories, and the users' permissions in these repositories.

\item The \texttt{./keys} subdirectory which contains the public keys of all
users of the system. The names of the key files designate the names of the
users\cite{gitolite-com-2014d}.

\item The \texttt{post-update} Git hook on the server side, parsing the above
config and file and keys subdirectory and making adequate changes to the
server repositories.

\end{enumerate}

It is important that access to this repository is safely guarded as it gives
complete control over the users and repositories on the Git server.

\subsubsection{Permissions}

Permissions in gitolite are granted on a per repository basis. Every time a
user attempts to perform a read or write operation on a repository, the user's
action is matched against a series of rules. If none of the rules match, the
user operation is denied.

Permissions may be granted to the entire repository or just to a particular
branch, tag, or even subfolder. Users may be granted, read, write, read-write,
and even forced write permissions (more on this in the next section).  There
are some even more fine grained permissions\cite{gitolite-com-2014e}, but we
will not be concerned with them here.

\subsection{Attack surface}

\subsubsection{Login shell}

It is often important with Git servers to disallow clients' shell requests.
This is typically achieved by setting the user's login shell to something
non-permissive, e.g. a \cite{man-1-git-shell}.

This set up is perhaps a bit superfluous, as gitolite disables interactive
shell login via the authorized keys file. Never-the-less it is a good extra
level of security, as the login shell of any user can only be modified by a
privileged user, which the \texttt{git} user is not.

\subsubsection{Session preparation dialog}

When a client is authenticated with OpenSSH, but before a user session starts,
the client and the host enter into a session preparation dialog.

The client can request a pseudo-tty (e.g. interactive shell), forwarding X11
connections (e.g. remote desktop), forwarding TCP connections (e.g. virtual
private networking), or forwarding the authentication agent connection over the
secure channel (e.g. using the secure connection to establish other secure
connections).

All these options open up the attack surface of our Git server. Fortunately,
all of these session dialog options can be disabled for any key in the
authorized keys file\cite{man-8-sshd}. By default, gitolite disables all of
these options for all keys.

\subsubsection{Forced push and rewriting history}\label{section:forced-push}

Git has a, somewhat controversial\cite{torvalds-2007,hamano-2009,rego-2013},
forced push feature. This bypasses the check that the remote ref being updated
should be an ancestor of the local ref used to overwrite
it\cite{man-1-git-push}. Meaning that the branch being updated should be the
strict base of the update.

Forced push is dangerous because it incautiously overwrites history and can
thereby inhibit assessment or even modify student records.

This is mitigated for by gitolite permissions. Students are simply not allowed
to perform a forced push\footnote{As an experimental bug, teaching staff are
still allowed to perform a forced push.}. This means that students cannot e.g.
ammend to a commit that they have already pushed to the server. The students
are encouraged to use \cite{man-1-git-revert} instead.

\subsubsection{Git, OpenSSH, and Perl}

Despite it's popularity, relatively few vulnerabilities have ever been found
outside of the Git development team\cite{cvedetails-com-2014a}. The security of
Git (out of the box) however, depends on the sensibility of the developers
involved. Impersonification and private key leaks are not always well guarded
against\cite{gerwitz-2013}, especially with the advent of modern Git hosting
services\cite{homakov-2012,huang-2013,homakov-2014}.

Likewise, OpenSSH has had relatively few vulnerabilities discovered outside of
the OpenSSH development team\cite{cvedetails-com-2014b}. However, the
underlying OpenSSL has been a lot less fortunate\cite{cvedetails-com-2014c}.

Perl has been only a bit less fortunate\cite{cvedetails-com-2014d}.

All of the above are popular pieces of software on public facing web servers.
Their security is therefore a matter of grave public concern. The referenced
material does not cover all of the underlying libraries of this software.

\subsection{Discussion}

\subsubsection{Pull requests}

Our model of a student submission being a Git push to a student branch is not
an accurate model of modern software development. In modern software
development, a developer may work on their own branch (as our students would),
and then make a ``pull request'' to merge their changes into the master branch.
(Alternatively, a developer might work on in their own repository, and then
make a pull request for their changes to be merged into the main project
repository\cite{bird-et-al-2009}.)

Such pull requests make little sense in education where all students are
working on the same problem --- a scenario you'd often go to great lengths to
avoid in industry. Instead, students are always allowed to submit what they
have to their own branch. Code reviews are then done of snapshots of the
student branch, e.g. automatically every time they push, or by a human at a
nominal point in time.

Alternatively, we could have chosen to have students make a pull request to a
special ``submission branch'', with the other branch being a ``draft branch''.
This would demand a more complicated set up of the Git server, perhaps using
Gerrit or Stash, as mentioned above. The pull request could then be accepted
if the code passed automatic code review.

Unfortunately, it is sometimes instructional to give credit for an attempt at
solving the problem. There may even come a situation where the student has made
it to submit some basic working code in the submission branch but has a more
comprehensive (non-working) solution in their draft branch. It would seem that
this would gravely complicate matters for the subsequent human code review. In
our model, the commit and test history of a branch is sufficient to reveal when
the code had last worked.

\subsubsection{Responding to students}

Responding to students via a subdirectory in their private branch means that
the students have to pull from their branch before they can make a subsequent
submission (the race condition aside). This is good because it encourages
students to read feedback and not to push in the blind. This is bad because it
might inhibit quick (re)submissions (made within minutes): as practice shows,
this is frequent close to a deadline.

An alternative could be to distribute feedback in a separate private student
branch, which is not writable by students. This is easy to set up in gitolite,
but is more permissive of students pushing in the blind, ignoring all feedback.
It also adds to the complexity of the student's view of the system: some
students may fail to realise that feedback is being given at all.

As another alternative, feedback could be provided interactively, as part of a
Git push operation. For instance, in a \texttt{post-receive} or a
\texttt{post-update} Git hook. These hooks lets us run custom scripts after the
real work of a Git push is done. The benefit is that the connection to the
client is not closed until these scripts end, and standard output and error are
redirected to the client\cite{man-5-githooks}. Although this allows to present
the test results immediately, it is unclear where the test results should be
persisted.

\subsubsection{Scalability}

We dedicate a Git branch to every student. To our knowledge, there is no
practical limit on the number of branches in a Git repository. If there is a
limit, it has to do with underlying file system limits, as every branch
requires a separate file in a particular subdirectory. This limit should be in
the manner of millions, and so not applicable in a course, department, or even
a university context.

Each time a client performs a Git operation, a connection to the Git server is
established. The server performs one of a limited set of (presumably, finite)
operations in a separate user session.

To our knowledge, there are no practical limits the number of simultaneous
connections to a Linux server, or simultaneous user sessions on a Linux server.
Again if there are limits, it would be close to underlying file system limits,
as connections and user sessions are typically file system bound on a Linux
machine.

The Git server should safely scale to a course, or a department, provided
sufficient memory, CPU, and disk resources and speeds. On a university scale,
it might be advisable to use a Git server per department, as OpenSSH operations
are fairly CPU expensive.
