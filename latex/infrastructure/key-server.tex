% vim: set spell:

\section{Key Server}

The purpose of the key server is to provide a centralized registry of the
public keys of students and staff. The key server is not intended to be a
general-purpose public key server. Instead, it's intended to serve as a
certificate authority in a formal or semi-formal sense (see also
\referToSection{gitolite-ssh-certificate-authority}).

\subsection{Student Key Registration}

An important role of the key server is to provide the means of registering
student public keys, and validating that the keys belong to the said students.

In a campus-based teaching environment, the teaching institution can serve as a
validating authority. It is however, important that the registration and
validation procedures do not inhibit teaching. A registration bot may be set up
to aid in the matter. Let the bot be an agent with some designated
private/public key pair.

A teaching institution typically designates every student with a unique
identifier. In an introductory lecture, the students may be presented with a
fresh secret key, and asked to cite this secret key in an email to the bot,
together with their unique identifier and public key.

The message may be additionally encrypted with bot's public key, and signed by
the student's private key.  This ensures that other students (and outsiders)
cannot peek at a student's identity, and that a student provides a valid public
key. A script can be provided to assist students in the matter.

The bot may first check the student identifier against a list of course
participants, provided by the teaching staff. If authorised, the bot associates
the student identifier with the email address and public key on the key server.
We can now engage in secure communication with the student.

In the unlikely event of impersonification, a member of the teaching staff can
go in and override this association with proper student credentials, or delete
the record entirely. We trust that a campus-based teaching institution is
capable of validating student identities on site. After all, if students are
not technically able to participate in a course, they can be expected to
eventually contact the teaching staff.

The key server can retain the students' public keys across courses, and
throughout their education. It would be a great contribution to the overall web
of trust on the Internet, if the bot also signed the public keys and published
them to a general-purpose public key server. Having obtained appropriate
permission from the students, of course.

\subsection{Staff Key Registration}

Staff key registration can be left to more manual means. It is important that
staff public keys are validated in person.

To be continued..

\subsection{Serving Public Keys}

The public key registry can serve as a basis for secure communication between
the students and staff in general. The public key registry should therefore be
made available on the teaching institution intranet, or even to the general
public.

To be continued..

\subsection{Discussion}

To be continued..
