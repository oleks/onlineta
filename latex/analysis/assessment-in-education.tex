% vim: set spell:

\section{Assessment in Education}

\label{section:analysis-assessment-in-education}

% Let us assume that a principal goal of education is to bring about learning
% with understanding\cite{formative-vs-summative}.

Assessment, or evaluation, in education is the practice of obtaining
information about students' knowledge, attitudes, and
skills\cite{assessment-and-burnout}.  The purpose of assessment may be
manifold: provide feedback or certification, perform selection or comparison,
improve learning processes, combinations of the aforementioned, and so
on\cite{first-ten-years}.

Assessment in education is primarily of students and their learning, or of
teachers and their teaching. The primary intent of the former is to improve
learning --- the latter --- to improve teaching. In this work we will primarily
concern ourselves with the former, so assessment for us will be getting to know
our students and the quality of their learning\cite{ramsden1992}.

\subsection{Categorising Assessment}

There are two principal of kinds of assessment: formative and summative. The
definition of each kind varies somewhat in educational research\cite{bloom1971,
sadler1989, formative-vs-summative}, and their mutual compatibility is
questionable\cite{butler1988}. The intent of this work is not to advise on the
matter, but to aid in performing assessment, regardless of the flavour.

Let us therefore adopt a primitive distinction, which still supports the
purposes of our further analysis:

\begin{description}[\setleftmargin{60pt}\setlabelstyle{\bf}]

\item[Formative] A student's strengths and weaknesses are documented in
free-text form. Formative assessments are qualitative and non-standard\-ised:
they are aimed at measuring the quality of a student's learning, rather than
whether they live up to some standard criteria. 

\item[Summative] A student is ranked on some well-defined scale, at some
well-defined intervals, based on some well-defined criteria. Summative
assessments are often compoundable and comparable. They may allow to deduce
holistic summative assessments of students, or student groups, quantitatively
measure student progress, etc.

\end{description}

Formative assessment necessitates the ability to perform personalised
assessments, whereas summative assessment demands the ability to specify
standards and perform standardised assessments.

There are other forms of assessment: diagnostic assessment, self-assessment,
peer-assessment, etc \cite{bull-mckenna-2004,topping-1998}.  These forms of
assessment vary in terms of the formative/summative dimensions, but primarily
differ in terms of when, by whom, and of whom the assessment is made.

\subsection{Feedback}

Feedback is information about the difference between the reference level and
the actual level of some parameter which is used to remedy the difference in
some way\cite{ramaprasad1989}.

Feedback is an important bi-product of assessment in
education\cite{black-william-1998}. Ideally, feedback informs the student of
the quality of their work, outlines key errors, provides corrective guidance,
and encourages further student learning. To be so, it is important that
feedback is understandable, timely, and acted upon by
students\cite{gibbs-simpson-2004}.

These requirements are an active area of educational research, and one aiding
approach is to use computer-assisted assessment.

\subsection{Computer-Assisted Assessment}

Computer-assisted assessment is the form of assessment performed with the
assistance of computers\cite{conole-warburton-2005}. The benefit of using
computers is ideally, fast, highly-available, consistent, and unbiased
assessment\cite{ala-mutka-2005}. The requirement is that the perceived student
performance can be encoded in some useful digital format.

This requirement however, has proven evasive. Free form performances, such as
essays or oral presentations, are still hard to assess
automatically\cite{valenti-et-al-2003}.  On the other hand, it is questionable
in how far easily assessable performances, such as, multiple-choice
questionnaires, are appropriate for assessment in higher
education\cite{conole-warburton-2005}.

We hypothesise, without further proof, that in how far computers can assist in
assessment, depends on how rigorous the student performance can be expected to
be.  We formalise this notion in the following sections.

\subsection{Assignment}

An assignment is a request for someone to perform a particular job.  An
assignment in education is a request for a student to make a particular
performance, and often, to provide a record thereof. One purpose of an
assignment is to provide basis for an assessment. The request therefore, often
includes a specification of what the assessment will be based on, and in what
time frame the assignment should be completed in order to be assessed.

\subsection{Submission}

A submission is a record of student performance, submitted for the purposes of
assessment. A digital submission is a digital encoding of such a record.
Digital submissions are amenable to assessment with the assistance of
computers, and thus of most interest to us.

We say that a structure is \emph{rigorous} if we can devise effective
procedures to extract the individual elements of the structure, preferably
within a finite number of steps. In how far computers can assist in assessment
depends on how rigorously a submission can be expected to be structured.

With the advent of modern computer technology, submissions can also be expected
to be structured with the assistance of computers. Such use of computers can
give some rigor to the structure of submissions.

For instance, in a multiple-choice test, a computer may present the student
with the questions and options. The student may then respond to the computer
using toggles, and have the computer encode the options thereby chosen in a
tableau of integers. An assessment then constitutes merely comparing against a
reference tableau --- something computers are notoriously good at.

If instead a student is asked to write an essay, modern computers can assist
with little more than dictionaries, thesauri, and mark up. From the point of
view of a computer, an essay is often little more than a stream of words and
punctuation. Much of the structuring is conducted directly by the student.

An essay can of course be assumed to be written in a natural language. Natural
languages however, often lack in rigor. Vagueness and ambiguity flourishes in
natural languages. They are at best, somewhat rigorous. The extent of this
``somewhat'' is the subject matter of much research in natural language
processing and automated essay assessment\cite{valenti-et-al-2003}.

Beyond where computers can help, it is the question of how rigorous one can
expect the students to be. In some disciplines, such as Computer Science, high
rigor can often be expected in certain types of assignments, such as
programming assignments. We explore this in the following sections.

% the goal of higher education?

% validity vs. reliability.

% force students to see all feedback?

% Computer-aided assessment


% https://en.wikipedia.org/wiki/Educational_assessment
% https://en.wikipedia.org/wiki/Educational_technology
% https://en.wikipedia.org/wiki/Philosophy_of_education
% https://en.wikipedia.org/wiki/Instructional_theory
% https://en.wikipedia.org/wiki/Educational_theory

% https://en.wikipedia.org/wiki/Computer_aided_assessment
% https://en.wikipedia.org/wiki/E-assessment
% https://en.wikipedia.org/wiki/QTI

% http://www.imsglobal.org/question/index.html

% https://www.brookes.ac.uk/services/ocsld/resources/briefing_papers/p_p_assessment.pdf

% http://www.maasfep.org/spring13/Assessment%26ClassroomLearning_Black%26William.pdf

% http://files.eric.ed.gov/fulltext/ED214952.pdf

%http://www.google.dk/books?hl=en&lr=&id=ZQpUAQAAQBAJ&oi=fnd&pg=PP1&dq=Assessing+student+learning+in+higher+education&ots=iyEc1Evj2D&sig=ZlsuHVag2xcoTDJ2infrzGBbZoE&redir_esc=y#v=onepage&q=Assessing%20student%20learning%20in%20higher%20education&f=false

% https://en.wikipedia.org/wiki/Standards_for_Educational_and_Psychological_Testing

% http://www.ed.ac.uk/schools-departments/institute-academic-development/learning-teaching/staff/advice/researching/publications/experience-of-learning

% assignment text
% tool for testing a submission, providing some automated feedback
% tool for transforming the automated feedback to formal summative and formative feedback.
% summative comments are compoundable.

% per student attempt per exercise we save, their submission, and our summative
% and formative feedback for it.

% the summative feedback, if any, per assignment is defined when the assignment
% is defined, also a flag is given whether the summative feedback should be
% visible to the student or not.

% The feedback kinds is typically something like a scale, or multiple scales.

% ## To define an assignment:
% 1. Provide an assignment text.
% 2. Specify who may submit the assignment -- see group of assignments.
% 3. Specify the sequence of summative feedback kinds to be yielded for every
%    attempt to solve the assignment.
% 4. Specify a means of compounding the summative feedback of all attempts into
%    one holistic summative feedback on an assignment for a submitter.
% 5. Specify which of the summative feedbacks should be visible to the submitter.
% 6. Specify a test tool, providing some raw feedback for an attempt.
% 7. Specify a feedback transformation tool, yielding the said summative
%    and formative feedbacks.
% 8. Specify whether groups are supported.

% The summative feedback is compoundable. The formative feedback is not.

% The summative feedback may be compounded into a progress report - the
% progress report may or may not be made visible to the submitters.

% In all assignments, for all submitters, one type of summative feedback is
% visible to all submitters -- whether or not they have submitted.

% ## A group of assignments is:
% 1. A sequence of assignments.
% 2. A specification of the group of people that may submit the attempts to
%    solve the assignment.

% ## A progress report may be generated for a group of assignments.
% 1. A progress report is individual per pupil. (how about groups?)
% 2. Compound all the holistic summative feedbacks for all assignments into one
%    holistic summative feedback for the group of assignments for the pupil.

% http://psycnet.apa.org/psycinfo/1963-07136-000

% http://www.youtube.com/watch?v=jTH3ob1IRFo

% There is a range of summative data ready for collection when using
% computer-aided assessment. This may be useful in assessment in general.

% Use of computers blurs the definitions of summative and formative feedback
% further. \cite[262--265]{bloom1971}.
