% vim: set spell:

\section{Assessment in Education}

% Let us assume that a principal goal of education is to bring about learning
% with understanding\cite{formative-vs-summative}.

Assessment, or evaluation, in education is the practice of obtaining
information about students' knowledge, attitudes, and
skills\cite{assessment-and-burnout}.  The purpose of assessment may be
manifold: provide feedback or certification, perform selection or comparison,
improve learning processes, combinations of the aforementioned, and so
on\cite{first-ten-years}.

Assessment in education is primarily of students and their learning, or of
teachers and their teaching. The former may later be used to improve learning,
the latter, to improve teaching. In this work we will primarily concern
ourselves with the former, so assessment for us will be getting to know our
students and the quality of their learning\cite{ramsden1992}.

\subsection{Formative and Summative Assessment}

There are two principal of kinds of assessment: formative and summative. The
definition of each kind varies somewhat in educational research\cite{bloom1971,
sadler1989, formative-vs-summative}, and their mutual compatibility is
questionable\cite{butler1988}. The intent of this work is not to advise on the
matter, but to aid in performing assessments, regardless of the flavours.

Let us therefore adopt a primitive distinction, which still supports the
purposes of our further analysis:

\begin{description}[\setleftmargin{60pt}\setlabelstyle{\bf}]

\item[Formative] A student's strengths and weaknesses are documented in
free-text form. Formative assessments are qualitative and non-standard\-ised:
they are aimed at measuring the quality of a student's learning, rather than
whether they live up to some standard criteria. 

\item[Summative] A student is ranked on some well-defined scale, at some
well-defined intervals, based on some well-defined criteria. Summative
assessments are often compoundable and comparable. They may allow to deduce
holistic summative assessments of students, or student groups, quantitatively
measure student progress, etc.

\end{description}

Formative assessment necessitates the ability to perform personalised
assessments, whereas summative assessment demands the ability to specify
standards and perform standardised assessments.

There are other forms of assessment, such as diagnostic assessment,
self-assessment, peer-assessment, etc. These forms of assessment still vary
along the formative/summative scales, but differ in terms of when, by whom, and
of whom the assessment is made.

\subsection{Feedback}

Feedback is information about the difference between the reference level and
the actual level of some parameter which is used to remedy the difference in
some way\cite{ramaprasad1989}.

Feedback is an important bi-product of assessment in
education\cite{black-william-1998}. Ideally, feedback informs the student of
the quality of their work, outlines key errors, provides corrective guidance,
and encourages further student learning. To be so, it is important that
feedback is understandable, timely, and acted upon by
students\cite{gibbs-simpson-2004}.

These requirements are an active area of educational research, and one aiding
approach is to use computer-assisted assessment.

\subsection{Computer-Assisted Assessment}

Computer-assisted assessment is the form of assessment performed with the
assistance of computers\cite{conole-warburton-2005}. The benefit of using
computers is often fast, consistent, unbiased assessment, provided that the
perceived student performance can be encoded in some useful digital format.

This requirement however, has proven evasive. Anything that is in free-speech
or free-text form, is still not very useful\cite{valenti-et-al-2003}. On the
other hand, it is questionable in-how-far the obviously useful forms, such as,
multiple-choice questionnaires, are appropriate for assessment in higher
education\cite{conole-warburton-2005}.

Let a submission be a digital encoding of perceived student performance. Being
digital, a submission is always done in some particular data format. A data
format is an organisation of the data such that there exists an effective
procedure that can recognise the encoded data, preferably in a finite number of
steps.

The recognition of the encoded data constitutes the construction of a
representation of the structure of the data. Such a representation will signify
blobs within the data and their mutual relationships. The blobs of data may
themselves be encoded using some other data format.

We say that the structuredness of data is inverse proportional to the maximum
size of the largest blob of data allowed by its data format. For instance, a
text file is not very well-structured since it is composed of a series of blobs
roughly 4096 bytes each. An array of 64-bit integers, on the other hand, is
fairly well-structured, having a maximum blob size of 8 bytes.

We hypothesize, that computers are best-suited for dealing with highly
structured data.

Research into computer-assisted assessment provides some ground for this
hypothesis. Multiple-choice questionaires are easy to assess: they can be
encoded as a tableau of low-range integers. Free-text submissions are hard to
assess: they can seldom be encoded as anything better than one large blob of
data. The general approach in research in this area, seems to be to attempt to
impose some structure on the otherwise plain data blob which is the
submission\cite{valenti-et-al-2003}.

% the goal of higher education?

% validity vs. reliability.

% force students to see all feedback?

% Computer-aided assessment


% https://en.wikipedia.org/wiki/Educational_assessment
% https://en.wikipedia.org/wiki/Educational_technology
% https://en.wikipedia.org/wiki/Philosophy_of_education
% https://en.wikipedia.org/wiki/Instructional_theory
% https://en.wikipedia.org/wiki/Educational_theory

% https://en.wikipedia.org/wiki/Computer_aided_assessment
% https://en.wikipedia.org/wiki/E-assessment
% https://en.wikipedia.org/wiki/QTI

% http://www.imsglobal.org/question/index.html

% https://www.brookes.ac.uk/services/ocsld/resources/briefing_papers/p_p_assessment.pdf

% http://www.maasfep.org/spring13/Assessment%26ClassroomLearning_Black%26William.pdf

% http://files.eric.ed.gov/fulltext/ED214952.pdf

%http://www.google.dk/books?hl=en&lr=&id=ZQpUAQAAQBAJ&oi=fnd&pg=PP1&dq=Assessing+student+learning+in+higher+education&ots=iyEc1Evj2D&sig=ZlsuHVag2xcoTDJ2infrzGBbZoE&redir_esc=y#v=onepage&q=Assessing%20student%20learning%20in%20higher%20education&f=false

% https://en.wikipedia.org/wiki/Standards_for_Educational_and_Psychological_Testing

% http://www.ed.ac.uk/schools-departments/institute-academic-development/learning-teaching/staff/advice/researching/publications/experience-of-learning

% assignment text
% tool for testing a submission, providing some automated feedback
% tool for transforming the automated feedback to formal summative and formative feedback.
% summative comments are compoundable.

% per student attempt per exercise we save, their submission, and our summative
% and formative feedback for it.

% the summative feedback, if any, per assignment is defined when the assignment
% is defined, also a flag is given whether the summative feedback should be
% visible to the student or not.

% The feedback kinds is typically something like a scale, or multiple scales.

% ## To define an assignment:
% 1. Provide an assignment text.
% 2. Specify who may submit the assignment -- see group of assignments.
% 3. Specify the sequence of summative feedback kinds to be yielded for every
%    attempt to solve the assignment.
% 4. Specify a means of compounding the summative feedback of all attempts into
%    one holistic summative feedback on an assignment for a submitter.
% 5. Specify which of the summative feedbacks should be visible to the submitter.
% 6. Specify a test tool, providing some raw feedback for an attempt.
% 7. Specify a feedback transformation tool, yielding the said summative
%    and formative feedbacks.
% 8. Specify whether groups are supported.

% The summative feedback is compoundable. The formative feedback is not.

% The summative feedback may be compounded into a progress report - the
% progress report may or may not be made visible to the submitters.

% In all assignments, for all submitters, one type of summative feedback is
% visible to all submitters -- whether or not they have submitted.

% ## A group of assignments is:
% 1. A sequence of assignments.
% 2. A specification of the group of people that may submit the attempts to
%    solve the assignment.

% ## A progress report may be generated for a group of assignments.
% 1. A progress report is individual per pupil. (how about groups?)
% 2. Compound all the holistic summative feedbacks for all assignments into one
%    holistic summative feedback for the group of assignments for the pupil.

% http://psycnet.apa.org/psycinfo/1963-07136-000

% http://www.youtube.com/watch?v=jTH3ob1IRFo

% There is a range of summative data ready for collection when using
% computer-aided assessment. This may be useful in assessment in general.

% Use of computers blurs the definitions of summative and formative feedback
% further. \cite[262--265]{bloom1971}.
