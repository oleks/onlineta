% vim: set spell:

\section{Roles in Assessment}

There are two principal roles in assessment: the assessing and the assessed.
It is the assessing that define the form of assessment and perform the
assessment itself. It is a matter of ethical concern that the assessed are
sufficiently informed of their assessment and concede to it.

In education, the assessing are often also involved in the role of teaching,
and the assessed are often also involved in the role of learning.

Those involved in the role of assessing, but not in teaching, are involved in
either censure of the assessment, or are interested merely in the summative
purposes of an assessment. Those involved in the role of assessed, but not in
the role of learning, are of little interest to us.

\subsection{Teaching}

Teaching is the expediting of learning. Students learn on their own, but
teachers provide the conditions for effective learning\cite{skinner-1965}.

Teachers exert authority over students.

\subsubsection{Teaching assistants}

Teaching assistants assist in teaching responsibilities. They are teaching
subordinates of teachers. They exert some authority over students, but are
often limited in their authority when it comes to important summative
assessments. The result is that teaching assistants perform much of the
formative assessment, and provide guiding remarks either for the purposes of
feedback or to ease important summative assessments for the teachers.

Teaching assistants come about as a scaling mechanism. Once class size exceeds 

% Lectures

\subsection{Learning}

Learning is the gaining of knowledge or skills. It is a qualitative change of a
person's view of the world. Individuals engage in learning in hope of being
somehow enlightened or trained for solving particular kinds of problems.

\subsection{Censure and Summative Assessments}

Censure is a process of quality assurance of assessment. A censor's
participation in an assessment in an assessment varies from mere observation to
avid participation.

\subsection{General Public}

The general public includes those who are ultimately interested in the outcomes
of education and the quality of assessment therein. This includes both
perspective students, future employers, the politically concious, etc.

The general public may be interested in open access to the elements of
education and assessment for the purposes of assessing the quality of
education. The intent may be to see if the education lives up to social
expectations, demands of the labour market, political promises, etc.

Privacy and anonymity is a matter of grave public concern. If open access is
given, it should only identify those who may reasonably be held responsible for
an eventual lack of quality in education. Also, issues of copyright have to be
taken into account.

As the general public would assess education, and not students, students should
not be personally identifiable by the general public. In how far teachers and
teaching assistants may reasonably be held responsible by the general public
for the quality of education may be a matter of university policy (as their
employer). It is important that the assessed are sufficiently informed of their
assessment and have conceded to it.

As students typically own the content they produce, individual student work or
commentary should not be made available to the general public. In how far
teachers and teaching assistants own the content they produce may again be a
matter of university policy (as their employer). It is important that the
owners of content have command over its reproduction.
