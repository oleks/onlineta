% vim: set spell:

\documentclass[a4paper]{article}

\usepackage[a4paper]{geometry}

\usepackage[fancy]{template}
\usepackage{survival-pack}

\setup{%
  subject={Master Project},%
  assignment={Project Description: Online TA},%
  date={February 4, 2014.}%
}
\setupLocation[short=DIKU]{Datalogisk institut, Copenhagen University}
\setupAuthor[addendum={\email{oleks@diku.dk}}]{Oleksandr Shturmov}

\begin{document}

\maketitle
\thispagestyle{first} % no fancy header on first page (just a fancy footer).

We would like to specify and implement an online system for automatically
evaluating submissions to assignments. The prime focus is on programming
assignments, the submissions to which tend to lend their way to automated
evaluation.

The system is meant as an aid to the traditional on-site education at the
Department of Computer Science, University of Copenhagen. The intent of the
system is to aid teachers and their assistants in some of the evaluations they
have to make in the course of a programming course. It would be fair to say
that we would like to design and implement an online teacher's assistant, or
online TA.

The fact that the producers and the consumers of the learning content will be
Computer Science teachers and students, allows us to disregard many of the
problems that would otherwise be worth dealing with, e.g.  graphical user
interfaces. In contrast, the evaluation of submissions to programming
assignments opens the system up to challenging problems and threats. The
automatic evaluation of such submissions typically involves running static or
dynamic analyses of arbitrary student programs.

We would like to ensure that all evaluations run in isolated environments, and
all students get a fair share of system resources. Student programs may run for
too long, use too many resources, fail in unpredictable ways, or even
deliberately attempt to compromise the system. The analyses themselves,
specified by the teachers and their assistants, may behave likewise, except the
lack of incentive to deliberately compromise the system.

It is these problems and threats which are the focus of this project. In the
following sections we further specify the scope of the project, and present the
project-specific learning objectives.

\bigbreak

\textbf{Keywords:} online, automatic evaluation, security engineering,
arbitrary program execution, learning management system.

% vim: set spell:

\section*{Scope}

The implemented system should be capable of handling an on-site programming
course at DIKU. This demands a moderate-sized web-based platform, capable of
handling a couple hundred simultaneous submissions.

As we are dealing with Computer Science teachers and students, a command-line
interface with the web-application would suffice. The design however, should
support the latter addition of a graphical user interface, especially for
students.

The programs submitted by students and the analyses specified by teachers and
their assistants may, for all intents and purposes, be regarded as arbitrary
programs that may or may not adhere to some posed requirements. It may however
be assumed that the teachers and assistants have no incentive to deliberately
compromise the system. Furthermore, we would like to stay programming language
agnostic, so we constrain ourselves to the operating system level.

The system should enforce read and write permissions, disallowing students to
pose or change assignments, submit on other students behalf, view or change the
private information of teachers, their assistants, or other students, etc.

The system should stay programming language agnostic and allow the use of
various existing tools for the static and dynamic analyses of student programs,
e.g.  compilers or linters. The use of tools installed natively on the system
poses its own security risks.

These are some the basic building blocks of a learning management
system\cite{lms}.



%% vim: set spell:

\section{An online system}

We would like an online system where students can make their submissions at
their leisure. This demands a moderate-sized web-based platform, capable of
handling many simultaneous submissions. As we are dealing with Computer Science
teachers and students, a command-line interface with the web-application would
suffice. This does not alleviate the system from many of the usual threats and
problems inherent with having a publicly facing web-application.



%% vim: set spell:

\section{Arbitrary program execution}

We would like to stay programming language agnostic, so we constrain ourselves
to the operating system level.

The programs submitted by students and the analyses specified by teachers and
their assistants may, for all intents and purposes, be regarded as arbitrary
programs that may or may not adhere to some posed requirements. It may however
be assumed that the teachers and assistants have no incentive to deliberately
compromise the system.

We would like to ensure that all evaluations run in isolated environments, and
all students get a fair share of resources. This can be challenging to do when
student programs and the analyses are arbitrary programs.



%% vim: set spell:

\section{A basic learning management system}

The system should enforce read and write permissions, disallowing students to
pose or change assignments, submit on other students behalf, view or change the
private information of teachers, their assistants, or other students, etc. The
system should stay programming language agnostic and allow the use of various
existing tools for the static and dynamic analyses of student programs, e.g.
compilers or linters. The use of tools installed natively on the system poses
its own security risks.

These are some the basic building blocks of a learning management
system\cite{lms}.



% vim: set spell:

\section*{Learning objectives}

The following is a list of project-specific learning objectives:

\begin{enumerate}

\item Perform a security analysis, and identify safe security policies for

\begin{enumerate}

\item an online system;

\item arbitrary program execution; and

\item a basic learning management system.

\end{enumerate}

\item List, explain, and compare possible measures for enforcing the security
policies above.

\item Design and implement a system which enforces the security policies above.

\item Verify that the system mediates the threats identified in the analysis
above, using e.g.:

\begin{enumerate}

\item unit-testing;

\item property-based testing;

\item stress testing (e.g. many simultaneous submissions); and

\item penetration testing.

\end{enumerate}

\end{enumerate}



\begin{thebibliography}{9} % 9 if < 10 references, 99 if < 100 references, etc.

\bibitem[Wheeler, 2007]{why-oss-fs}

David A. Wheeler. \emph{Why Open Source Software / Free Software (OSS/FS,
FLOSS, or FOSS)? Look at the Numbers!}. Revised as of April 16, 2007.
\url{https://web.archive.org/web/20140210042356/http://www.dwheeler.com/oss_fs_why.html}.

\end{thebibliography}



\end{document}

