% vim: set spell:

\documentclass[a4paper]{article}

\usepackage[a4paper]{geometry}

\usepackage[fancy]{template}
\usepackage{survival-pack}

\setup{%
  subject={Master Project},%
  assignment={Project Description: Online TA},%
  date={February 4, 2014.}%
}
\setupLocation[short=DIKU]{Datalogisk institut, Copenhagen University}
\setupAuthor[addendum={\email{oleks@diku.dk}}]{Oleksandr Shturmov}

\begin{document}

\maketitle
\thispagestyle{first} % no fancy header on first page (just a fancy footer).

We would like to specify and implement an online system for automatically
evaluating submissions to assignments. The prime focus is on programming
assignments, the submissions to which tend to lend their way to automated
evaluation.

The system is meant as an aid to the traditional on-site education at the
Department of Computer Science, University of Copenhagen. The intent of the
system is to aid teachers and their assistants in some of the evaluations they
have to make in the course of a programming course. It would be fair to say
that we would like to design and implement an online teacher's assistant, or
online TA.

The fact that the producers and the consumers of the learning content will be
Computer Science teachers and students, allows us to disregard many of the
problems that would otherwise be worth dealing with, e.g.  graphical user
interfaces. In contrast, the evaluation of submissions to programming
assignments opens the system up to challenging problems and threats. The
automatic evaluation of such submissions typically involves running static or
dynamic analyses of arbitrary student programs.

We would like to ensure that all evaluations run in isolated environments, and
all students get a fair share of system resources. Student programs may run for
too long, use too many resources, fail in unpredictable ways, or even
deliberately attempt to compromise the system. The analyses themselves,
specified by the teachers and their assistants, may behave likewise, except the
lack of incentive to deliberately compromise the system.

It is these problems and threats which are the focus of this project. In the
following sections we further specify the scope of the project, and present the
project-specific learning objectives.

\bigbreak

\textbf{Keywords:} online, automatic evaluation, security engineering,
arbitrary program execution, learning management system.

% vim: set spell:

\section*{Scope}

The implemented system should be capable of handling an on-site programming
course at DIKU. This demands a moderate-sized web-based platform, capable of
handling a couple hundred simultaneous submissions.

As we are dealing with Computer Science teachers and students, a command-line
interface with the web-application would suffice. The design however, should
support the latter addition of a graphical user interface, especially for
students.

The programs submitted by students and the analyses specified by teachers and
their assistants may, for all intents and purposes, be regarded as arbitrary
programs that may or may not adhere to some posed requirements. It may however
be assumed that the teachers and assistants have no incentive to deliberately
compromise the system. Furthermore, we would like to stay programming language
agnostic, so we constrain ourselves to the operating system level.

The system should enforce read and write permissions, disallowing students to
pose or change assignments, submit on other students behalf, view or change the
private information of teachers, their assistants, or other students, etc.

The system should stay programming language agnostic and allow the use of
various existing tools for the static and dynamic analyses of student programs,
e.g.  compilers or linters. The use of tools installed natively on the system
poses its own security risks.

These are some the basic building blocks of a learning management
system\cite{lms}.



%% vim: set spell:

\section{An online system}

We would like an online system where students can make their submissions at
their leisure. This demands a moderate-sized web-based platform, capable of
handling many simultaneous submissions. As we are dealing with Computer Science
teachers and students, a command-line interface with the web-application would
suffice. This does not alleviate the system from many of the usual threats and
problems inherent with having a publicly facing web-application.



%% vim: set spell:

\section{Arbitrary program execution}

We would like to stay programming language agnostic, so we constrain ourselves
to the operating system level.

The programs submitted by students and the analyses specified by teachers and
their assistants may, for all intents and purposes, be regarded as arbitrary
programs that may or may not adhere to some posed requirements. It may however
be assumed that the teachers and assistants have no incentive to deliberately
compromise the system.

We would like to ensure that all evaluations run in isolated environments, and
all students get a fair share of resources. This can be challenging to do when
student programs and the analyses are arbitrary programs.



%% vim: set spell:

\section{A basic learning management system}

The system should enforce read and write permissions, disallowing students to
pose or change assignments, submit on other students behalf, view or change the
private information of teachers, their assistants, or other students, etc. The
system should stay programming language agnostic and allow the use of various
existing tools for the static and dynamic analyses of student programs, e.g.
compilers or linters. The use of tools installed natively on the system poses
its own security risks.

These are some the basic building blocks of a learning management
system\cite{lms}.



% vim: set spell:

\section*{Learning objectives}

The following is a list of project-specific learning objectives:

\begin{enumerate}

\item Perform a security analysis, and identify safe security policies for

\begin{enumerate}

\item an online system;

\item arbitrary program execution; and

\item a basic learning management system.

\end{enumerate}

\item List, explain, and compare possible measures for enforcing the security
policies above.

\item Design and implement a system which enforces the security policies above.

\item Verify that the system mediates the threats identified in the analysis
above, using e.g.:

\begin{enumerate}

\item unit-testing;

\item property-based testing;

\item stress testing (e.g. many simultaneous submissions); and

\item penetration testing.

\end{enumerate}

\end{enumerate}



% vim: set spell:

\begin{thebibliography}{9} % 9 if < 10 references, 99 if < 100 references, etc.

\bibitem[Wheeler, 2007]{why-oss-fs}

David A. Wheeler. \emph{Why Open Source Software / Free Software (OSS/FS,
FLOSS, or FOSS)? Look at the Numbers!}. Revised as of April 16, 2007. Retrieved
from \url{http://www.dwheeler.com/oss_fs_why.html} on March 7, 2014.

Archived by WebCite\textsuperscript{\textregistered}\ at
\url{http://www.webcitation.org/6NtSnvALX}.

\bibitem[CS2013]{cs2013}

ACM/IEEE-CS Joint Task Force on Computing Curricula. \emph{Computer Science
Curricula 2013}. December 2013. ACM Press and IEEE Computer Society Press.
Retrieved from
\url{http://ai.stanford.edu/users/sahami/CS2013/final-draft/CS2013-final-report.pdf}
on March 7, 2014.

Archived by WebCite\textsuperscript{\textregistered}\ at
\url{http://www.webcitation.org/6NtTXPH2s}.

\bibitem[EHEA (1999)]{ehea1999}

Joint declaration of the European Ministers of Education. \emph{The Bologna
Declaration}. June 19, 1999. The European Higher Education Area. Retrieved from
\url{http://www.ehea.info/Uploads/Declarations/BOLOGNA_DECLARATION1.pdf} on
March 9, 2014.

Archived by WebCite\textsuperscript{\textregistered}\ at
\url{http://www.webcitation.org/6Nwr8h8l7}.

\bibitem[CND (2004)]{cnd2004}

Sebastian Horst and Carl Winsløw. \emph{Undervisning i blokstruktur -
potentialer og risici}. April 1, 2004. DidakTips 5. Center for Natufagenes
Didaktik.  University of Copenhagen. Retrieved from
\url{http://www.ind.ku.dk/publikationer/didaktips/didaktips5/5.undervisningiblokstruktur-potentialerogrisicimedomslag.pdf}
on March 9, 2014.

Archived by WebCite\textsuperscript{\textregistered}\ at
\url{http://www.webcitation.org/6NwjYWlBi}.

\bibitem[BEK 814]{bek814}

Ministeriet for Forskning, Innovation og Videregående Uddannelser.
\emph{Bekendtgørelse om bachelor- og kandidatuddannelser ved universiteterne
(uddannelsesbekendtgørelsen)}. Ministerial Order no. 814 of December 19,
2013.  Retrieved from
\url{https://www.retsinformation.dk/Forms/R0710.aspx?id=160853} on March 9,
2014.

Archived by WebCite\textsuperscript{\textregistered}\ at
\url{http://www.webcitation.org/6NyGHXc4N}.

\bibitem[BEK 1520]{bek1520}

Ministeriet for Forskning, Innovation og Videregående Uddannelser.
\emph{Bekendtgørelse om bachelor- og kandidatuddannelser ved universiteterne
(uddannelsesbekendtgørelsen)}. Ministerial Order no. 1520 of December 19,
2013.  Retrieved from
\url{https://www.retsinformation.dk/Forms/R0710.aspx?id=160853} on March 9,
2014.

Archived by WebCite\textsuperscript{\textregistered}\ at
\url{http://www.webcitation.org/6Nx4DxLGL}.

\bibitem[BEK 666]{bek666}

Ministeriet for Forskning, Innovation og Videregående Uddannelser.
\emph{Bekendtgørelse om eksamen og censur ved universitetsuddannelser
(eksamensbekendtgørelsen)}. Ministerial Order no. 666 of June 24, 2012.
Retrieved from \url{https://www.retsinformation.dk/Forms/R0710.aspx?id=142560}
on March 10, 2014.

Archived by WebCite\textsuperscript{\textregistered}\ at
\url{http://www.webcitation.org/6NyI0DlqJ}.

\bibitem[BEK 250]{bek250}

Ministeriet for Forskning, Innovation og Videregående Uddannelser.
\emph{Bekendtgørelse om karakterskala og anden bedømmelse ved
universitetsuddannelser (karakterbekendtgørelsen)}. Ministerial Order no. 250
of March 15, 2007.  Retrieved from
\url{https://www.retsinformation.dk/Forms/R0710.aspx?id=29307} on March 11,
2014.

Archived by WebCite\textsuperscript{\textregistered}\ at
\url{http://www.webcitation.org/6NzVOdWtv}.

\bibitem[Curricula (2013)]{faelles-del-ENG}

\emph{The shared section of the BSc and MSc curricula for study programmes at
the Faculty of Science, University of Copenhagen}. September 2013. Retrieved
from
\url{http://www.science.ku.dk/studerende/studieordninger/faelles_sto/Faelles-del-ENG-2013-web.pdf/}
on March 10, 2014.

Archived by WebCite\textsuperscript{\textregistered}\ at
\url{http://www.webcitation.org/6NyEpPK67}.

\bibitem[Bradfoot \& Black (2004)]{first-ten-years}

Patricia Broadfoot and Paul Black. \emph{Redefining assessment? The first ten
years of assessment in education}. 2004. Assessment in Education: Principles,
Policy \& Practice, Vol. 11, No. 1, pp. 7--26. DOI:
10.1080/0969594042000208976. Retrieved from
\url{https://cmap.helsinki.fi/rid=1G5ND18R4-1QLJN7R-1SB} on March 16, 2014.

Archived by WebCite\textsuperscript{\textregistered}\ at
\url{http://www.webcitation.org/6O7OhF8LW}.

\bibitem[Pishghadam et al. (2014)]{assessment-and-burnout}

Reza Pishghadam, Bob Adamson, Shaghayegh Shayesteh Sadafian, and Flora L. F.
Kan. \emph{Conceptions of assessment and teacher burnout}. 2014. Assessment in
Education: Principles, Policy \& Practice, Vol. 21, No. 1, pp. 34--51. DOI:
10.1080/0969594X.2013.817382.

\bibitem[Harlen \& James (1997)]{formative-vs-summative}

Wynne Harlen and Mary James. \emph{Assessment and Learning: differences and
relationships between formative and summative assessment}. 1997. Assessment in
Education: Principles, Policy \& Practice, Vol. 4, No. 3, pp. 365--379. DOI:
10.1080/0969594970040304.

\bibitem[Butler (1988)]{butler1988}

Ruth Butler. \emph{Enhancing and understanding intrinsic motivation: the
effects of of task-involving and ego-involving evaluation on interest and
performance}. February 1988. British Journal of Educational Psychology, Vol.
58, No. 1, pp. 1--14.

\bibitem[Sadler (1989)]{sadler1989}

D.Royce Sadler. \emph{Formative assessment and the design of instructional
systems}. June 1989. Instructional Science, Vol. 18, No. 2, pp. 119--144.
Kluwer Academic Publishers. DOI: 10.1007/BF00117714

\bibitem[Bloom et al. (1971)]{bloom1971}

Benjamin S. Bloom, J. Thomas Hastings, and George F. Madaus. \emph{Handbook on
Formative and Summative Evaluation of Student Learning.} 1971. McGraw-Hili,
Inc. United States. Library of Congress Catalog Card Number 75129488. ISBN
0070061149.

\bibitem[Ramaprasad (1989)]{ramaprasad1989}

Arkalgud Rapaprasad. \emph{On the definition of feedback}. January 1989.
Behavioural Science, Vol. 28, No. 1, pp. 8--13.

\bibitem[Black \& William (1998)]{black-william-1998}

Paul Black and Dylan William. \emph{Assessment and Classroom Learning}. 1998.
Assessment in Education: Principles, Policy \& Practice, Vol. 5, No. 1, pp.
7--74. DOI: 10.1080/0969595980050102.

\bibitem[Gibbs \& Simpson (2004)]{gibbs-simpson-2004}

Graham Gibbs and Claire Simpson. \emph{Conditions Under Which Assessment
Supports Students' Learning}. May 2004. Learning and Teaching in Higher
Education, Issue 1. Retrieved from
\url{http://www2.glos.ac.uk/offload/tli/lets/lathe/issue1/issue1.pdf} on March
22, 2014.

Archived by WebCite\textsuperscript{\textregistered}\ at
\url{http://www.webcitation.org/6OGkhvGyE}.

\bibitem[Ramsden (1992)]{ramsden-1992}

Paul Ramsden. \emph{Learning to Teach in Higher Education}. 1992. Routledge.
ISBN 0-415-06415-5.

\bibitem[Conole \& Warburton (2005)]{conole-warburton-2005}

Gr\'ainne Conole and Bill Warburton. \emph{A review of computer-assisted
assessment}. March 2005. The Journal of the Association for Learning Technology
(ALT-J), Research in Learning Technology, Vol. 14, No. 1, pp. 17--31. Retrieved
from
\url{http://www.researchinlearningtechnology.net/index.php/rlt/article/download/10970/12674}
on March 23, 2014.

Archived by WebCite\textsuperscript{\textregistered}\ at
\url{http://www.webcitation.org/6OIAQzo2d}.

\bibitem[Valenti et al. (2003)]{valenti-et-al-2003}

Salvatore Valenti, Francesca Neri, and Alessandro Cucchiarelli. \emph{An
Overview of Current Research on Automated Essay Grading}. January 2003. Journal
of Information Technology Education, Vol. 2, No. 1, pp. 319--330.

\bibitem[Ala-Mutka (2005)]{ala-mutka-2005}

Kirsti M. Ala-Mutka. \emph{A Survey of Automated Assessment Approaches for
Programming Assignments}. June 2005. Computer Science Education, Vol. 15, No.
2, pp. 83--102.

\bibitem[Bull \& McKenna (2004)]{bull-mckenna-2004}

Joanna Bull and Colleen McKenna. \emph{Blueprint for Computer-Assisted
Assessment}. 2004. Taylor \& Francis e-Library. Master e-book ISBN:
0-203-46468-0.

\bibitem[Topping (1998)]{topping-1998}

Keith Topping. \emph{Peer Assessment between Students in Colleges and
Universities}. Autumn 1998. Review of Educational Research, Vol. 68, No. 3, pp.
249--276.

\bibitem[Carter et al. (2003)]{carter-et-al-2003}

Janet Carter, John English, Kirsti Ala-Mutka, Martin Dick, William Fone, Ursula
Fuller, Judy Sheard. \emph{How Shall We Assess This?} December 2003. Working
group reports from ITiCSE on Innovation and technology in computer science
education (ITiCSE-WGR '03), David Finkel (Ed.), pp. 107--123.
DOI: 10.1145/960492.960539.

\bibitem[CS Curricula 2013]{cs-curricula-2013}

The Joint Task Force on Computing Curricula. Association for Computing
Machinery (ACM). IEEE Computer Society. \emph{Computer Science Curricula 2013:
Curriculum Guidelines for Undergraduate Degree Programs in Computer Science}.
December 20, 2013. Retrieved from
\url{http://www.acm.org/education/CS2013-final-report.pdf} on March 29, 2014.

Archived by WebCite\textsuperscript{\textregistered}\ at
\url{http://www.webcitation.org/6ORGeUYy5}.

\bibitem[Skinner (1965)]{skinner-1965}

Burrhus Frederic Skinner. Harvard University. \emph{Reflections on a Decade of
Teaching Machines}. 1965. In Teaching Machines and Programmed Learning, Vol. 2,
Robert Glaser (Ed.), pp. 5--20. National Education Association of the United
States. LCCN: 60-15721.

\bibitem[Kember (1997)]{kember-1997}

David Kember. Hong Kong Polytechnic University. \emph{A Reconceptualisation of
the Research into University Academics' Conceptions of Teaching}. 1997.
Learning and Instruction, Vol. 7, No. 3, pp. 225--275.

\bibitem[Sclater \& Howie (2003)]{sclater-howie-2003}

Niall Sclater and Karen Howie. Center for Educational Systems, University of
Strathclyde, Scotland, UK. \emph{User requirements of the ``ultimate'' online
assessment engine}. April 2003. Computers \& Education, Vol. 40, No. 3, pp.
285--306. DOI: 10.1016/S0360-1315(02)00132-X.

\bibitem[Ohloh (2014)]{ohloh-2014}

Ohloh. Black Duck Software, Inc. Tools. Compare Repositories. Retreived from
\url{http://www.ohloh.net/repositories/compare} on April 12, 2014.

\bibitem[Git (2014)]{git-2014}

Git. \emph{Git and Software Freedom Conservancy.} Retrieved from
\url{http://git-scm.com/sfc} on April 12. 2014.

Archived by WebCite\textsuperscript{\textregistered}\ at
\url{http://www.webcitation.org/6OmayfnSi}.

\bibitem[GitProjects (2014)]{gitprojects-2014}

GitProjects. \emph{Projects that use Git for their source code management}. Git
Wiki. Last updated on April 5, 2014. Retrieved from
\url{https://git.wiki.kernel.org/index.php/Main_Page} on April 12, 2014.

Archived by WebCite\textsuperscript{\textregistered}\ at
\url{http://www.webcitation.org/6OmbdAIQF}.

\bibitem[gitolite.com (2014a)]{gitolite-com-2014a}

\emph{Hosting git repositories}. Retrieved from
\url{http://gitolite.com/gitolite/index.html} on Arpil 13, 2014.

Archived by WebCite\textsuperscript{\textregistered}\ at
\url{http://www.webcitation.org/6OnsR9f7R}.

\bibitem[gitolite.com (2014b)]{gitolite-com-2014b}

\emph{authentication and authorization in gitolite}. Retrieved from
\url{http://gitolite.com/gitolite/how.html} on Arpil 13, 2014.

Archived by WebCite\textsuperscript{\textregistered}\ at
\url{http://www.webcitation.org/6OoAvdLeK}.

\bibitem[gitolite.com (2014c)]{gitolite-com-2014c}

\emph{who uses gitolite}. Retrieved from
\url{http://gitolite.com/gitolite/who.html} on Arpil 13, 2014.

Archived by WebCite\textsuperscript{\textregistered}\ at
\url{http://www.webcitation.org/6OoB1QVpA}.

\bibitem[gitolite.com (2014d)]{gitolite-com-2014d}

\emph{adding and removing users}. Retrieved from
\url{http://gitolite.com/gitolite/users.html} on Arpil 18, 2014.

Archived by WebCite\textsuperscript{\textregistered}\ at
\url{http://www.webcitation.org/6OvHM2Z2h}.

\bibitem[gitolite.com (2014e)]{gitolite-com-2014e}

\emph{different types of write operations}. Retrieved from
\url{http://gitolite.com/gitolite/write-types.html} on Arpil 18, 2014.

Archived by WebCite\textsuperscript{\textregistered}\ at
\url{http://www.webcitation.org/6OvIKAXXm}.

\bibitem[kernel.org (2014)]{kernel-org-2014}

\emph{How does kernel.org provide its users access to the git trees?}
Frequently asked questions. The Linux Kernel Archives. Retrieved from
\url{https://www.kernel.org/faq.html#whygitolite} on April 13, 2014.

Archived by WebCite\textsuperscript{\textregistered}\ at
\url{http://www.webcitation.org/6OoBgEPwG}.

\bibitem[gitolite Google Group (2011)]{gitolite-google-group-2011}

Dan Carpenter and Sitaram Chamarty. \emph{security audit of Gitolite}.
Discussion on the gitolite Google Group.  First post made on September 30,
2011. Last post made on October 2, 2011. Retrieved from
\url{https://groups.google.com/d/topic/gitolite/jcUkIFKxbQ8/discussion} on
April 13, 2014.

Archived by WebCite\textsuperscript{\textregistered}\ at
\url{http://www.webcitation.org/6OoCHMGop}.

\bibitem[openssh.com (2014)]{openssh-com-2014}

OpenBSD. OpenSSH. Retrieved from \url{http://www.openssh.com/} on April 16,
2014.

Archived by WebCite\textsuperscript{\textregistered}\ at
\url{http://www.webcitation.org/6OsQXs51M}.

\bibitem[openssl.org (2014)]{openssl-org-2014}

OpenSSL. Cryptography and SSL/TLS Toolkit. Retrieved from
\url{https://www.openssl.org/} on April 20, 2014.

Archived by WebCite\textsuperscript{\textregistered}\ at
\url{http://www.webcitation.org/6OyQjYQnz}.


\bibitem[ssh(1)]{man-1-ssh}

BSD. \emph{SSH(1)}. ssh — OpenSSH SSH client (remote login program). BSD
General Commands Manual. Published April 6, 2014.  Retrieved from
\url{http://man7.org/linux/man-pages/man1/ssh.1.html} on April 16, 2014.

Archived by WebCite\textsuperscript{\textregistered}\ at
\url{http://www.webcitation.org/6OsTPwEgi}.

\bibitem[sshd(8)]{man-8-sshd}

BSD. \emph{SSHD(8)}. sshd --- OpenSSH SSH daemon. BSD System Manager's Manual.
Published on April 6, 2014. Retrieved from
\url{http://man7.org/linux/man-pages/man8/sshd.8.html} on April 16, 2014.

Archived by WebCite\textsuperscript{\textregistered}\ at
\url{http://www.webcitation.org/6OsTLo4Ce}.

\bibitem[git-shell(1)]{man-1-git-shell}

Git 1.9.rc1. \emph{GIT-SHELL(1)}. git-shell - Restricted login shell for
Git-only SSH access. Git Manual. Published January 30, 2014. Retrieved from
\url{http://man7.org/linux/man-pages/man1/git-shell.1.html} on Arpil 16, 2014.

Archived by WebCite\textsuperscript{\textregistered}\ at
\url{http://www.webcitation.org/6OsZP6SgC}.

\bibitem[git-push(1)]{man-1-git-push}

Git 1.9.rc1. \emph{GIT-PUSH(1)}. git-push - Update remote refs along with
associated objects. Git Manual. Published January 30, 2014. Retrieved from
\url{http://man7.org/linux/man-pages/man1/git-push.1.html} on Arpil 16, 2014.

Archived by WebCite\textsuperscript{\textregistered}\ at
\url{http://www.webcitation.org/6Ot2V2qp1}.

\bibitem[git-revert(1)]{man-1-git-revert}

Git 1.9.rc1. \emph{GIT-REVERT(1). git-revert - Revert some existing commits.}
Git Manual. Published January 30, 2014. Retrieved from
\url{http://man7.org/linux/man-pages/man1/git-revert.1.html} on Arpil 17, 2014.

Archived by WebCite\textsuperscript{\textregistered}\ at
\url{http://www.webcitation.org/6Ot6LflKS}.

\bibitem[git-hooks(5)]{man-5-githooks}

Git 1.9.rc1. \emph{GITHOOKS(15). githooks - Hooks used by Git.} Git Manual.
Published January 30, 2014. Retrieved from
\url{http://man7.org/linux/man-pages/man5/githooks.5.html} on Arpil 17, 2014.

Archived by WebCite\textsuperscript{\textregistered}\ at
\url{http://www.webcitation.org/6OtvjunoB}.

\bibitem[clone(2)]{man-2-clone}

Linux. \emph{CLONE(2). \texttt{clone}, \texttt{\_\_clone2} - create a child
process}. Linux Programmer's Manual. Published February 27, 2014. Retrieved
from \url{http://man7.org/linux/man-pages/man2/clone.2.html} on April 21, 2014.

Archived by WebCite\textsuperscript{\textregistered}\ at
\url{http://www.webcitation.org/6P04xS6aU}.

\bibitem[unshare(2)]{man-2-unshare}

Linux. \emph{UNSHARE(2). \texttt{unshare} - disassociate parts of the process
execution context}. Linux Programmer's Manual. Published April 17, 2013.
Retrieved from \url{http://man7.org/linux/man-pages/man2/unshare.2.html} on May
8, 2014.

Archived by WebCite\textsuperscript{\textregistered}\ at
\url{http://www.webcitation.org/6PPQVEMzp}.

\bibitem[setns(2)]{man-2-setns}

Linux. \emph{SETNS(2). \texttt{setns} - reassociate thread with a namespace}.
Linux Programmer's Manual. Published January 1, 2013.  Retrieved from
\url{http://man7.org/linux/man-pages/man2/setns.2.html} on May 8, 2014.

Archived by WebCite\textsuperscript{\textregistered}\ at
\url{http://www.webcitation.org/6PPQyqMeY}.

\bibitem[fork(2)]{man-2-fork}

Linux. \emph{FORK(2). \texttt{fork} - create a child process}. Linux
Programmer's Manual. Published March 12, 2014. Retrieved from
\url{http://man7.org/linux/man-pages/man2/fork.2.html} on April 21, 2014.

Archived by WebCite\textsuperscript{\textregistered}\ at
\url{http://www.webcitation.org/6P065nmSn}.

\bibitem[proc(5)]{man-5-proc}

Linux. \emph{PROC(2). \texttt{proc} - process information pseudo-filesystem}.
Linux Programmer's Manual. Published April 12, 2014. Retrieved from
\url{http://man7.org/linux/man-pages/man5/proc.5.html} on April 21, 2014.

Archived by WebCite\textsuperscript{\textregistered}\ at
\url{http://www.webcitation.org/6P0AnUQsS}.

\bibitem[pivot\_root(2)]{man-2-pivot-root}

Linux. \emph{PIVOT\_ROOT(2). \texttt{pivot\_root} - change the root
filesystem}.  Linux Programmer's Manual. Published June 13, 2013. Retrieved
from \url{http://man7.org/linux/man-pages/man5/pivot_root.2.html} on May 12,
2014.

Archived by WebCite\textsuperscript{\textregistered}\ at
\url{http://www.webcitation.org/6PVmriIoe}.

\bibitem[chroot(2)]{man-2-chroot}

Linux. \emph{CHROOT(2). \texttt{chroot} - change root directory}.  Linux
Programmer's Manual. Published September 20, 2010. Retrieved from
\url{http://man7.org/linux/man-pages/man5/chroot.2.html} on May 12, 2014.

Archived by WebCite\textsuperscript{\textregistered}\ at
\url{http://www.webcitation.org/6PVn2lIwz}.

\bibitem[getrlimit(2)]{man-2-getrlimit}

Linux. \emph{GETRLIMIT(2). \texttt{getrlimit}, \texttt{setrlimit},
\texttt{prlimit} - get/set resource limits}.  Linux Programmer's Manual.
Published on January 22, 2014. Retrieved from
\url{http://man7.org/linux/man-pages/man5/getrlimit.2.html} on May 13, 2014.

Archived by WebCite\textsuperscript{\textregistered}\ at
\url{http://www.webcitation.org/6PXC3h35y}.

\bibitem[jail(8)]{man-8-jail}

FreeBSD 9.2. \emph{JAIL(8). \texttt{jail} - manage system jails}. FreeBSD
System Manager's Manual. Published on October 12, 2013. Retrieved from
\url{http://www.freebsd.org/cgi/man.cgi?query=jail\&sektion=8} on May 19, 2014.

Archived by WebCite\textsuperscript{\textregistered}\ at
\url{http://www.webcitation.org/6PgdGfWV3}.

\bibitem[pam\_limits]{pam-limits}

Cristian Gafton. \emph{6.15. \texttt{pam\_limits} - limit resources.} The
Linux-PAM System Administrators' Guide. Version 1.1.2, 31. August 2010.
Retrieved from
\url{http://www.linux-pam.org/Linux-PAM-html/sag-pam_limits.html} on May 13,
2014.

Archived by WebCite\textsuperscript{\textregistered}\ at
\url{http://www.webcitation.org/6PXDQwwd6}.

\bibitem[Bird et al. (2009]{bird-et-al-2009}

Christian Bird, Peter C. Rigby, Earl T. Barr, David J. Hamilton, Daniel M.
German, and Prem Devanbu. \emph{The promises and perils of mining git}.
Proceedings of the 2009 6th IEEE International Working Conference on Mining
Software Repositories, pp. 1--10. IEEE Computer Society Washington, DC, USA.
ISBN 978-1-4244-3493-0.

\bibitem[Torvalds (2007)]{torvalds-2007}

Linus Torvalds. \emph{Re: Modify/edit old commit messages}. Retrieved from
\url{http://www.gelato.unsw.edu.au/archives/git/0702/38650.html} on April 17,
2014.

Archived by WebCite\textsuperscript{\textregistered}\ at
\url{http://www.webcitation.org/6Ot2yqQYR}.

\bibitem[Hamano (2009)]{hamano-2009}

Junio C Hamano. In response to ``\emph{How do I push amended commit to the remote
git repo?}'' StackOverflow. Posted on January 11, 2009, under the username
``gitster''. Edited by Gottlieb Notschnabel on September 3, 2013. Retrieved
from \url{http://stackoverflow.com/a/432518/108100} on April 17, 2014.

Archived by WebCite\textsuperscript{\textregistered}\ at
\url{http://www.webcitation.org/6Ot4dU85N}.

Revision history retrieved from
\url{http://stackoverflow.com/posts/432518/revisions} on April 17, 2014.

Archived by WebCite\textsuperscript{\textregistered}\ at
\url{http://www.webcitation.org/6Ot4hGKpY}.

\bibitem[Rego (2013)]{rego-2013}

Cauê C. M. Rego. In response to ``\emph{How to properly force a Git Push?}''
StackOverflow. Posted on May 22, 2013, under the username ``Cawas''.  Retrieved
from \url{http://stackoverflow.com/a/16702355/108100} on April 17, 2014.

Archived by WebCite\textsuperscript{\textregistered}\ at
\url{http://www.webcitation.org/6Ot4qY1Y2}.

\bibitem[cvedetails.com (2014a)]{cvedetails-com-2014a}

CVE Details. The ultimate security vulnerability datasource. \emph{GIT:
Security Vulnerabilities}. Retrieved from
\url{http://www.cvedetails.com/vulnerability-list/vendor_id-4008/GIT.html} on
April 18, 2014.

Archived by WebCite\textsuperscript{\textregistered}\ at
\url{http://www.webcitation.org/6OvKd3HTW}.

\bibitem[cvedetails.com (2014b)]{cvedetails-com-2014b}

CVE Details. The ultimate security vulnerability datasource. \emph{Openssh:
Security Vulnerabilities}. Retrieved from
\url{http://www.cvedetails.com/vulnerability-list/vendor_id-7161/Openssh.html},
on April 18, 2014.

Archived by WebCite\textsuperscript{\textregistered}\ at
\url{http://www.webcitation.org/6OvM6eL7m}.

\bibitem[cvedetails.com (2014c)]{cvedetails-com-2014c}

CVE Details. The ultimate security vulnerability datasource. \emph{Openssl:
Security Vulnerabilities}. Retrieved from
\url{http://www.cvedetails.com/vulnerability-list/vendor_id-217/Openssl.html},
on April 18, 2014.

Archived by WebCite\textsuperscript{\textregistered}\ at
\url{http://www.webcitation.org/6OvMlTlsh}.

\bibitem[cvedetails.com (2014d)]{cvedetails-com-2014d}

CVE Details. The ultimate security vulnerability datasource. \emph{Perl:
Security Vulnerabilities}. Retrieved from
\url{http://www.cvedetails.com/vulnerability-list/vendor_id-1885/Perl.html}, on
April 18, 2014.

Archived by WebCite\textsuperscript{\textregistered}\ at
\url{http://www.webcitation.org/6OvMGTWjE}.

\bibitem[Huang (2013)]{huang-2013}

Jay Huang. \emph{Pushing code to GitHub as Linus Torvalds}. Posted on December
16, 2013. Retrieved from
\url{http://www.jayhuang.org/blog/pushing-code-to-github-as-linus-torvalds/} on
April 18, 2014.

Archived by WebCite\textsuperscript{\textregistered}\ at
\url{http://www.webcitation.org/6OvK7dQBu}.

\bibitem[Gerwitzh (2013)]{gerwitz-2013}

Mike Gerwitz. \emph{A Git Horror Story: Repository Integrity With Signed
Commits}. Last edited on November 10, 2013. Retrieved from
\url{http://mikegerwitz.com/papers/git-horror-story} on April 18, 2014.

Archived by WebCite\textsuperscript{\textregistered}\ at
\url{http://www.webcitation.org/6OvKR8Hzi}.

\bibitem[Homakov (2012)]{homakov-2012}

Egor Homakov. \emph{Hacking rails/rails repo}. Published on March 4, 2012.
Retrieved from \url{http://homakov.blogspot.dk/2012/03/how-to.html} on April
18, 2014.

Archived by WebCite\textsuperscript{\textregistered}\ at
\url{http://www.webcitation.org/6OvLP3zYf}.

\bibitem[Homakov (2014)]{homakov-2014}

Egor Homakov. \emph{How I hacked GitHub again}. Published on February 7, 2014.
Retrieved from
\url{http://homakov.blogspot.dk/2014/02/how-i-hacked-github-again.html} on
April 18, 2014.

Archived by WebCite\textsuperscript{\textregistered}\ at
\url{http://www.webcitation.org/6OvKurm25}.

\bibitem[Stefan (2013)]{stefan-2013}

Deian Stefan. \emph{cjail - a sandbox utility for Arch Linux}. Stanford
Computer Security Lab. Last updated April 13, 2013. Source code retrieved from
\url{git@github.com:scslab/cjail.git} on April 14, 2014.

\bibitem[DOMJudge (2014)]{domjudge-2014}

DOMJudge team. \emph{DOMjudge Administrator's Manual}. Last published on May 9
2014. Retrieved from \url{http://www.domjudge.org/docs/admin-manual.pdf} on May
19, 2014.

Archived by WebCite\textsuperscript{\textregistered}\ at
\url{http://www.webcitation.org/6PgRY124e}.

\bibitem[Hutchings (2011)]{hutchings-2011}

Ben Hutchings. \emph{Accepted linux-2.6 3.0.0\textasciitilde
rc1-1\textasciitilde experimental.1 (source all amd64)}. Email sent to the
Debian Mailing Lists on June 1, 2011.  Retrieved from
\url{http://packages.qa.debian.org/l/linux-2.6/news/20110601T223515Z.html} on
April 25, 2014.

Archived by WebCite\textsuperscript{\textregistered}\ at
\url{http://www.webcitation.org/6P7kSQe9C}.

\bibitem[Wright et al. (2002)]{wright-et-al-2002}

Chris Wright, Crispin Cowan, Stephen Smalley, James Morris, and Greg
Kroah-Hartman. \emph{Linux Security Modules: General Security Support for the
Linux Kernel}. August 2002. In Proceedings of the 11th USENIX Security
Symposium. USENIX Association. San Francisco, California, United States.
Retrieved from
\url{https://www.usenix.org/legacy/event/sec02/full_papers/wright/wright.pdf}
on April 19, 2014.

Archived by WebCite\textsuperscript{\textregistered}\ at
\url{http://www.webcitation.org/6OxFtc1KV}.

\bibitem[Turner et al. (2010)]{turner-et-al-2010}

Paul Turner, Bharata B Rao, and Nikhil Rao. \emph{CPU bandwidth control for
CFS}. July 2010. In Linux Symposium, Vol 10, pp. 245--254. Ottawa, Ontoria,
Canada. Retrieved from
\url{https://www.kernel.org/doc/ols/2010/ols2010-pages-245-254.pdf} on April
28, 2014.

Archived by WebCite\textsuperscript{\textregistered}\ at
\url{http://www.webcitation.org/6P9ouxLJJ}.

\bibitem[Kerrisk (2013]{kerrisk-2013}

Michael Kerrisk. \emph{Namespaces in operation, part 1: namespaces overview}.
January 4, 2013. LWN.net. Retrieved from \url{http://lwn.net/Articles/531114/}
on May 12, 2014.

Archived by WebCite\textsuperscript{\textregistered}\ at
\url{http://www.webcitation.org/6PWKMBrvj}.

\bibitem[Kerrisk (2012)]{kerrisk-2012}

Michael Kerrisk. \emph{LCE: The failure of operating systems and how we can fix
it}. November 14, 2012. LWN.net. Retrieved from
\url{http://lwn.net/Articles/524952/} on May 19, 2014.

Archived by WebCite\textsuperscript{\textregistered}\ at
\url{http://www.webcitation.org/6PgfCc67y}.

\bibitem[Arch Linux Bug 36969]{arch-linux-userns}

Arch Linux. \emph{FS\#36969 - [linux] 3.13 add CONFIG\_USER\_NS}. Arch Linux
Bugtracker. Bug filed on September 17, 2013. Retrieved from
\url{https://bugs.archlinux.org/task/36969} on May 12, 2014.

Archived by WebCite\textsuperscript{\textregistered}\ at
\url{http://www.webcitation.org/6PWL6yV41}.

\bibitem[Fedora Bug 917708]{fedora-userns}

Fedora. \emph{Bug 917708 - Re-enable CONFIG\_USER\_NS}. Red Hat Bugzilla. Bug
filed on March 4, 2013. Retrieved from
\url{https://bugzilla.redhat.com/show_bug.cgi?id=917708} on May 12, 2014.

Archived by WebCite\textsuperscript{\textregistered}\ at
\url{http://www.webcitation.org/6PWMQA0oI}.

\bibitem[Veldhuizen (2003)]{veldhuizen-2003}

Todd L. Veldhuizen. \emph{C++ Templates are Turing Complete}. 2003. Indiana
University of Computer Science. Retrieved from
\url{http://port70.net/~nsz/c/c++/turing.pdf} on May 18, 2014. 

Archived by WebCite\textsuperscript{\textregistered}\ at
\url{http://www.webcitation.org/6Pf92rHhX}.

\bibitem[McCauley et al. (2008)]{mccauley-et-al-2008}

Ren\'ee McCauley, Sue Fitzgerald, Gary Lewandowski, et al. \emph{Debugging: A
Review of the Literature from an Educational Perspective}. June 2008. Computer
Science Education Vol. 18, No. 2, pp.67--92.

\bibitem[Lerner et al. (2007)]{lerner-et-al-2007}

Benjamin S. Lerner, Matthew Flower, Dan Grossman, and  Craig Chambers.
\emph{Searching for Type-Error Messages}. 2007. In Proceedings of the 2007 ACM
SIGPLAN conference on Programming language design and implementation (PLDI
'07). ACM, New York, NY, USA, 425--434. 

\bibitem[Malan (2010a)]{malan-2010a}

David J. Malan. 2010. \emph{Reinventing CS50}. 2010. In Proceedings of the 41st
ACM technical symposium on Computer science education (SIGCSE '10). ACM, New
York, NY, USA, pp. 152--156. 

\bibitem[Malan (2010b)]{malan-2010b}

David J. Malan. 2010. \emph{Moving CS50 into the cloud}. June 2010. J. Comput.
Small Coll. Vol. 25, No. 6, pp. 111--120.

\bibitem[Malam (2013)]{malan-2013}

David J. Malan. 2013. \emph{CS50 sandbox: secure execution of untrusted code}.
2013.  In Proceeding of the 44th ACM technical symposium on Computer science
education (SIGCSE '13). ACM, New York, NY, USA, pp. 141--146.

\bibitem[Graber (2014)]{graber-2014}

Stéphane Graber. \emph{LXC 1.0 released}. Posted on February 21, 2014.
Retrieved from \url{http://lwn.net/Articles/587545/} on May 19, 2014.

Archived by WebCite\textsuperscript{\textregistered}\ at
\url{http://www.webcitation.org/6PgzTGt4q}.

\bibitem[Linux kernel (v3.14.2)]{linux-kernel-v3.14.2}

The Linux kernel v3.14.2.

Retrieved from
\url{https://www.kernel.org/pub/linux/kernel/v3.x/linux-3.14.2.tar.xz} on April
28, 2014.

Signed by Greg Kroah-Hartman, using the GPG key:

\begin{lstlisting}
pub   4096R/6092693E 2011-09-23
uid     Greg Kroah-Hartman
          (Linux kernel stable release signing key) <greg@kroah.com>
sub   4096R/76D54749 2011-09-23
\end{lstlisting}

Yielding the signature:

\begin{lstlisting}
-----BEGIN PGP SIGNATURE-----
Version: GnuPG v2.0.22 (GNU/Linux)

iQIcBAABAgAGBQJTXE0zAAoJEDjbvchgkmk+0AUQAKrfkqeRXpwePAEHFCBqTqvN
R2fZa6tAY1w5psN8grWh2seu2W1KAtEk53oht/6uZITqs3i2pYmRAJyEzVTBggs9
4BI0rClqebeiO3wkD1biRAIMPQWt6UAB/pvjBeMmMiw4G7FFZHSvBSct91Dsnb87
4A7083ZT/A421C2OtH3vROehyQDyfHp+oL22SKMCoXKCCMCDZp5K07AMVggrzDoZ
KGDEeBpowCSCtoUEEBlrVGz/syyaWZzzcMy+UYeZ12JxpfgnX5oq14w1HIPfAhJn
/P6x7OvmN75oIrxrt4rRs+aUY97iuiEzPpn9F2K4rNruTZUXN7906h/WWCJ/K/b0
D80wC1msaJqMYIEhQICu5kwezVswKVHz3QM9BO1ak3RgObw3j70KKVxJQ95I6jYn
I3uz8RDGXWvp+6aso8vl/HWbQ6dCCA/9plYALJZmRcy2Yg0A0nH3w6+ckC1x/r4l
ZyR6NEcVYg27HQswjmWxbqUhapFMLQGj5oGZ9svbsdwet3ckQTcqAtS5N/YHZZaQ
SnzvY4dZ/MoRwdCGz0hC99RofIgMPgY8ypkc2GGvyGv9uDLsK4koB65ZX1zW/oRw
43eatEoY/Q1QyGWrbwqEWFY91XbZne1KJNwdXYkmTDawMI2F2zApIjsAHpMseJiN
XZPAJqjjAF6nhxRzsrI8
=HZrL
-----END PGP SIGNATURE-----
\end{lstlisting}

\bibitem[resource\_counter.txt]{resource-counter.txt}

Li Zefan, et al.  \emph{The Resource Counter}. Found in
\cite{linux-kernel-v3.14.2}, under \texttt{./Documentation/cgroups/}.

\bibitem[cpuacct.txt]{cpuacct.txt}

Bharata B Rao, et al. \emph{CPU Accounting Controller}.  Found in
\cite{linux-kernel-v3.14.2}, under \texttt{./Documentation/cgroups/}.

\bibitem[memory.txt]{memory.txt}

Li Zefan, et al.  \emph{Memory Resource Controller}. Found in
\cite{linux-kernel-v3.14.2}, under \texttt{./Documentation/cgroups/}.

\bibitem[cgroups.txt]{cgroups.txt}

Paul Menage, Paul Jackson, Chirstoph Lameter, et al. \emph{CGROUPS}.  Found in
\cite{linux-kernel-v3.14.2}, under \texttt{./Documentation/cgroups/}.

\bibitem[devices.txt]{devices.txt}

Li Zefan, Aristeu Rozanski, et al. \emph{Device Whitelist Controller}.  Found
in \cite{linux-kernel-v3.14.2}, under \texttt{./Documentation/cgroups/}.

\bibitem[blkio-controller.txt]{blkio-controller.txt}

Li Zefan, Aristeu Rozanski, et al. \emph{Block IO Controller}.  Found
in \cite{linux-kernel-v3.14.2}, under \texttt{./Documentation/cgroups/}.


\bibitem[sched-design-CFS.txt]{sched-design-cfs.txt}

J. Bruce Fields, et al. \emph{CFS Scheduler}.  Found in
\cite{linux-kernel-v3.14.2}, under \texttt{./Documentation/scheduler/}.

\bibitem[sched-bwc.txt]{sched-bwc.txt}

Bharata B Rao. \emph{CFS Bandwidth Control}.  Found in
\cite{linux-kernel-v3.14.2}, under \texttt{./Documentation/scheduler/}.

\bibitem[sched-rt-group.txt]{sched-rt-group.txt}

J. Bruce Fields. \emph{Real-Time group scheduling}.  Found in
\cite{linux-kernel-v3.14.2}, under \texttt{./Documentation/scheduler/}.

\bibitem[kernel-parameters.txt]{kernel-parameters.txt}

Linus Torvalds, et al. \emph{Kernel Parameters}. Found in
\cite{linux-kernel-v3.14.2}, under \texttt{./Documentation/}.

\bibitem[LSM.txt]{lsm.txt}

Kees Cook. \emph{Linux Security Module Framework}. Found in
\cite{linux-kernel-v3.14.2}, under \texttt{./Documentation/security/}.

\end{thebibliography}


\end{document}

