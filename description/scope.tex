% vim: set spell:

\section*{Scope}

The implemented system should be capable of handling an on-site programming
course at DIKU. This demands a moderate-sized web-based platform, capable of
handling a couple hundred simultaneous submissions.

As we are dealing with Computer Science teachers and students, a command-line
interface with the web-application would suffice. The design however, should
support the latter addition of a graphical user interface, especially for
students.

The programs submitted by students and the analyses specified by teachers and
their assistants may, for all intents and purposes, be regarded as arbitrary
programs that may or may not adhere to some posed requirements. It may however
be assumed that the teachers and assistants have no incentive to deliberately
compromise the system. Furthermore, we would like to stay programming language
agnostic, so we constrain ourselves to the operating system level.

The system should enforce read and write permissions, disallowing students to
pose or change assignments, submit on other students behalf, view or change the
private information of teachers, their assistants, or other students, etc.

The system should stay programming language agnostic and allow the use of
various existing tools for the static and dynamic analyses of student programs,
e.g.  compilers or linters. The use of tools installed natively on the system
poses its own security risks.

These are some the basic building blocks of a learning management
system\cite{lms}.

