% vim: set spell:

\chapter{General Linux Concepts}

This chapter covers some general Linux concepts which are not necessarily to
known to the casual Linux user.

\section{Tasks}

\label{appendix:general-linux-concepts:tasks}

The distinction between a thread and a process in the Linux kernel is somewhat
more subtle than in operating systems textbooks.

In the Linux kernel, a thread of execution, also called a task, has a thread ID
(also called a PID inside the kernel), a thread group ID, and a parent thread
group ID. New tasks can be created using the \cite{man-2-clone} system
call\footnote{The more canonical \cite{man-2-fork} system call is seldom used.
Its behaviour can be mimicked by \cite{man-2-clone}. This is indeed what the
standard glibc \texttt{fork()} does.}. Depending on the parameters passed, the
child task can share various parts of its execution context with its parent
task. For instance, we may choose to stay under the same thread group ID, or
parent thread group ID, share open files, memory, etc.

A process is a nonempty set of tasks that share the same thread group ID. A
process is identified by its thread group ID. This is what is commonly referred
to as the PID in user space, whereas we refer to a thread ID as TID (unlike PID
inside the kernel). The system calls \texttt{gettid(2)}, \texttt{getpid(2)} and
\texttt{getppid(2)} return the thread ID, thread group ID, and parent thread
group ID of the running task, respectively.
