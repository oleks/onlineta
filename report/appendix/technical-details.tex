% vim: set spell:

\chapter{Technical Details}

This appendix covers some of the technical details related to this work.

% TODO: test instance

\section{Control Groups}

\label{appendix:technical-details:control-groups}

This section covers some useful technical details of cgroups. This is not a
general introduction to cgroups. For that, we refer you back to
\referToSection{sandboxing:control-groups}.

Many modern Linux distributions come with cgroups and many standard subsystems
(as discussed in \referToSection{sandboxing:control-groups}) enabled.  Your
system's \texttt{/proc/config.gz} can reveal the setup on your
system\cite{man-5-proc}.  If \texttt{CONFIG\_CGROUPS} is enabled, you have
cgroups support. The variables related to various subsystems are explored
further in the following subsections.

\subsection{Managing cgorups}

\subsubsection{Mounting}

A subsystem may be associated with at most one hierarchy. What hierarchies
already exist, and what subsystems they are associated with, depends on the
system at hand. The system's \texttt{/proc/mounts} can reveal how this is setup
on your system\cite{man-5-proc}.

\subsection{memory}

Due to the considerable overhead of memory and swap accounting, some
distributions do not enable this cgroup, or merely do not enable swap
accounting by default. The latter is especially misleading. If swapping is
enabled, a memory limit with no swap limit has at best a hapless effect.

You can check the setup on your system by checking the options prefixed with
\texttt{CONFIG\_MEMCG\_} in your \texttt{/proc/config.gz}. Swap accounting can
be enabled using the standard kernel parameter
\texttt{swapaccount=1}\cite{kernel-parameters.txt}. Enabling the memory cgroup
can be a little more distribution-specific.  In a Debian kernel, this can be
done using the kernel parameter
\texttt{cgroup\_enable=memory}\cite{hutchings-2011}.

The memory subsystem does not necessarily perform hierarchical accounting. This
can enabled by writing \texttt{1} to the \texttt{memory.use\_hierarchy} control
file in the root cgroup.

\subsection{cpu}

The cpu subsystem also facilitates control over the Real-Time scheduler (RT) of
the Linux kernel\cite{sched-rt-group.txt}. RT schedules real-time tasks, i.e.
tasks for which it is important to meet deadlines. A particular amount of CPU
time must be guaranteed over a particular period of time.

For the RT scheduler, the subsystem parameters facilitate limiting how much CPU
time the real-time tasks in a cgroup may spend in total over a period of time.
Enforcing such limits and meeting real-time deadlines seems like a heedful
task. For simplicity, we'll disallow students from spawning real-time tasks. In
a default setup, spawning real-time tasks requires privileged access, which we
do not want to grant students anyhow.

\subsection{cpuset}

The cpusets subsystem allows us to assign a set of CPU cores and a set of
memory nodes to a cgroup. This can be used to further partition system
resources from a fairly high level.

We choose to let the system be runnable on commodity hardware. Having few
processor cores, and few memory nodes, this subsystem is of little use to us.

\subsection{devices}

The device node type, its major and minor identifiers of can be found using
\texttt{stat}. For instance,

\begin{lstlisting}
$ stat --format "type:%F, major:%t, minor:%T" /dev/urandom 
type:character special file, major:1, minor:9
\end{lstlisting}

The format of a whitelist entry is \texttt{<type> <major>:<minor> <spec>},
where \texttt{<major>} and \texttt{<minor>} can be \texttt{*} indicating all
versions. Writing \texttt{a} to \texttt{devices.allow} or \texttt{devices.deny}
is the same as writing \texttt{a *:* rwm} to the same file.

Reading \texttt{devies.allow} and \texttt{devices.deny} control files is
prohibited. Instead, the current device whitelist for a cgroup can be inspected
by reading the \texttt{devices.list} control file.

\subsection{blkio}

The Block IO subsystem allows us to monitor and mandate access to I/O
operations on block devices using cgroups\cite{blkio-controller.txt}. The
monitoring parameters provide for insight into the I/O performance of a cgroup.
The access mandating parameters provide for proportional and absolute limits on
the number of I/O operations by a cgroup.

We choose to not provide access to block devices to students in general. This
is done due to the ease of implementation of a particular security policy.
Giving unmandated access to a block device, in theory, gives unmandated access
to all data on that device. Protecting data on such a device is a complicated
matter. Although SELinux, with it's per-inode restrictions, could presumably be
used to this end, it is easier to just not give access to block devices in
general.

As will be discussed later, this does not inhibit us in providing students with
a general purpose read/write file system.

\section{Namespaces}

\label{appendix:namespaces}

This section covers some useful technical details of namespaces. This
is not a general introduction to namespaces. For that, we refer you back to
\referToSection{sandboxing:namespaces}.

The namespaces that a task is associated with can be listed using the
\cite{man-5-proc} pseudo file system. There is a pseudo file
\texttt{/proc/[pid]/ns/[nstype]}, for each thread group ID (\texttt{[pid]}) on
the system, and for each namespace type (\texttt{[nstype]}) enabled on the
system.

You can check which namespace types are enabled on your system, by reading the
\texttt{/proc/config.gz} file, or listing the files in a process namespace
subdirectory. For instance, listing the namespaces subdiretory of the init
process:

\begin{lstlisting}
$ sudo ls /proc/1/ns
ipc  mnt  net  pid  uts  user
\end{lstlisting}

Not all namespace types supported by the Linux kernel are enabled by default in
mainstream Linux distributions. As discussed in
\referToSection{sandboxing:namespaces:user}, the user namespace is frequently
disabled by default due to security concerns. To our knowledge, enabling
support for a namespace type requires running a custom kernel in general.
