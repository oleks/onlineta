% vim: set spell:

\chapter{Technical Details}

This appendix covers some of the technical details related to this work.

% TODO: test instance

\section{Control Groups}

Many modern Linux distributions come with cgroups and many of these standard
subsystems enabled. The system's \texttt{/proc/config.gz} can reveal the setup
on your system\cite{man-5-proc}.  If \texttt{CONFIG\_CGROUPS} is enabled, you
have cgroups support. The variables related to various subsystems are explored
further in the following subsections.

\subsection{Managing cgroups}

Cgroups are managed via a pseudo-filesystem: cgroups reside in memory, but can
be manipulated through the virtual file system.  \texttt{cgroup} is therefore
an inherent file system type on systems that have the cgroups functionality
enabled.

\subsubsection{Cgroups, subsystems, and hierarchies}

A cgroup is an association of a set of tasks with a set of preferences for a
set of subsystems.  A hierarchy is a set of cgroups arranged in a rooted tree.
Every task in the system is attached to exactly one cgroup in the hierarchy.
All cgroups in the hierarchy, associate their tasks with the same set of
preferences --- we say that a hierarchy is associated with a set of
subsystems\footnote{\cite{cgroups.txt} is ambiguous wrt. whether a hierarchy
can exist without being associated with at least one subsystem. For the sake of
simplicity, we'll assume that it can; although cgroups without associated
subsystems have few practical applications, as we have already discussed.}.
\referToFigure{hierarchy-example} illustrates a couple example hierarchies.

\begin{figure}[h!]

\centering

\begin{tikzpicture}

\draw (-3.5, 0) node(aa){$\chev{\set{1,2},\set{\alpha,\beta}}$};

\draw (-5, -1) node(ab){$\chev{\set{3},\set{\alpha,\beta}}$};

\draw (-2, -1) node(ac){$\chev{\set{},\set{\alpha\beta}}$};

\draw (1.7, 0) node(ba){$\chev{\set{},\set{\gamma}}$};

\draw (0.5, -1) node(bb){$\chev{\set{1},\set{\gamma}}$};

\draw (3, -1) node(bc){$\chev{\set{2,3},\set{\gamma}}$};

\draw [->] (aa) -- (ab);

\draw [->] (aa) -- (ac);

\draw [->] (ba) -- (bb);

\draw [->] (ba) -- (bc);

\end{tikzpicture}

\caption[]{An illustration of two example cgroup hierarchies on a particular
system. The set of identifiers of the tasks running on the system is
$T=\set{1,2,3}$. The set of subsystems available on the system is
$S=\set{\alpha,\beta,\gamma,\delta}$ (not all subsystems need be associated
with a hierarchy). Every node (cgroup) has type
$\mathcal{P}(T)\times\mathcal{P}(S)$, where $\mathcal{P}$ denotes the
powerset.}

\label{figure:hierarchy-example}

\end{figure}

\subsubsection{Mounting}

What hierarchies already exist, and what subsystems they are associated with,
depends on the system at hand. The system's \texttt{/proc/mounts} can reveal
how this is setup on your system\cite{man-5-proc}.

\subsubsection{Control files and child groups}

After a hierarchy is successfully mounted, we see a range of files, and perhaps
folders, below our target. We refer to these as control files and child groups,
respecitvely.

We monitor/modify the preferences of a cgroup by monitoring/modifying the
control files. We create/remove child groups by creating/removing
subdirectories below our target.

With a few exceptions at the root of the hierarchy, all cgroups contain the
same files, created when the cgrpup is created. Some files are common to all
hierarchies, others are due to the associated subsystems. Two common files are
of particular interest:

\begin{description}[\setleftmargin{0.2in}\breaklabel\setlabelstyle{\tt}]

\item [cgroup.procs] Lists the set of thread group IDs in the current cgroup.
Appending a thread group ID to this file moves all the threads in the thread
group into this cgroup.

\item [tasks] Lists the set of thread IDs in the current cgroup. Appending a
thread ID to this file moves the thread into this cgroup.

\end{description}

\subsubsection{Hierarchical accounting}

Hierarchical accounting is when resource accounting is child group aware. All
resource usage is summed up for all tasks in the cgroup, and recursively
for all child groups. Limits are then imposed on the entire hierarchy. A
subsystem does not necessarily perform hierarchical accounting.

\subsection{The Resource Counter}

\label{section:cgroups:resource-counter}

The resource counter is a framework for managing a resource when using control
groups\cite{resource-counter.txt}.  The internal data structures aside, the
framework makes recommendations wrt. the control files. A couple of the
recommended control files are of interest to us:

\begin{description}[\setleftmargin{0.2in}\breaklabel\setlabelstyle{\tt}]

\item[<resource>.max\_usage\_in\_<unit\_of\_measurement>] Reading this file, we
get the maximal usage of the resource over time, in the given units. Writing to
this file, resets the value to the current usage of the resource. (The data
written is ignored.)

\item[<resource>.limit\_in\_<unit\_of\_measurement>] Reading this file, we get
the maximal allowed usage of the resource, in the given units. Writing to this
file resets the limit to the given value. A special value may indicate no
limit.

\end{description}

These files are of interest to us as they allow us to probe the usage of a
resource in a test instance and set up resource limits for students or staff.

\subsection{memory}

The memory subsystem allows us to monitor and limit the memory usage of the
tasks in a cgroup\cite{memory.txt}. This includes both user and kernel memory
and swap usage.  The subsystem optionally performs hierarchical accounting.

Due to the considerable overhead of memory and swap accounting, some
distributions do not enable this cgroup, or merely do not enable swap
accounting by default. The latter is especially misleading. If swapping is
enabled, a memory limit with no swap limit has at best a hapless effect.

You can check the setup on your system by checking the options prefixed with
\texttt{CONFIG\_MEMCG\_} in your \texttt{/proc/config.gz}. Swap accounting can
be enabled using the standard kernel parameter
\texttt{swapaccount=1}\cite{kernel-parameters.txt}. Enabling the memory cgroup
can be a little more distribution-specific.  In a Debian kernel, this can be
done using the kernel parameter
\texttt{cgroup\_enable=memory}\cite{hutchings-2011}.

The memory subsystem uses a resource counter for a couple different memory
resources. The resource counter control files (see also
\referToSection{cgroups:resource-counter}) are prefixed as follows:

\begin{description}[\setleftmargin{0.2in}\breaklabel\setlabelstyle{\tt}]

\item[memory]

The main memory counter. This includes both user and kernel memory.

\item[memory.memsw]

The main memory, plus swap. Limiting this value to the same value as the main
memory controller, disables swap.

\item[memory.kmem]

Kernel memory. All kernel memory is also accounted for by the main memory
counter. It is not necessary to limit this value if swapping is disabled and
there is a limit on the main memory counter (since kernel memory cannot be
swapped out).

\item[memory.kmem.tcp]

Kernel TCP buffer memory. Although we will disallow networking in general, it
might be a good idea to 0-limit this resource as an extra precaution.

\end{description}

The limits and usage are always measured in bytes. Setting the limit to
\texttt{-1}, removes the limit on the resource.

The memory subsystem does not necessarily perform hierarchical accounting. This
can enabled by writing \texttt{1} to the \texttt{memory.use\_hierarchy} control
file in the root cgroup.

\subsection{cpuacct}

The CPU accounting (cpuacct) subsystem allows us to monitor the CPU time usage
of the tasks in a cgroup\cite{cpuacct.txt}. The cpuacct subsystem always
performs hierarchical accounting.

The cpuacct subsystem provides a couple control files of interest:

\begin{description}[\setleftmargin{0.2in}\breaklabel\setlabelstyle{\tt}]

\item[cpuacct.usage] Shows the total CPU time spent by the cgroup, in
nanoseconds.

\item[cpuacct.usage\_percpu] Shows the total CPU time spent by the cgroup, for
each CPU core, in nanoseconds.

\item[cpuacct.stat] Shows a further division of the CPU time spent. For now,
showing how much of the CPU time was spent running in user mode, and how much
in kernel mode, in the \texttt{USER\_HZ} time unit.

\end{description}

\subsection{cpu}

The cpu subsystem facilitates CPU scheduling parameters for a
cgroup\cite{sched-design-cfs.txt,sched-bwc.txt,sched-rt-group.txt}. The
parameters currently facilitate control over two different schedulers in the
Linux kernel:

\begin{description}[\setleftmargin{0.2in}\breaklabel\setlabelstyle{\it}]

\item[Completely Fair Scheduler (CFS)] A proprtional share CPU scheduler. The
CPU time is divided fairly among tasks depending on their priority
and the share assigned to their cgroup.

\item[Real-Time Scheduler (RT)] A real-time scheduler for real-time tasks, i.e.
tasks for which it is important to meet deadlines. For real-time tasks, a
particular amount of CPU time must be guaranteed over a particular period of
time.

\end{description}

For the RT scheduler, the subsystem parameters facilitate limiting how much CPU
time the real-time tasks in a cgroup may spend in total over a period of time.
Enforcing such limits and meeting real-time deadlines seems like a heedful
task. For simplicity, we'll disallow students from spawning real-time tasks. In
a default setup, spawning real-time tasks requires privileged access, which we
already do not grant to our sandboxed programs.

The CFS parameters facilitate first-and-foremost the enforcing of a lower bound
on the amount of CPU time allocated to a cgroup. This is done by assigning a
relative share (weight) to a cgroup. The shares are enforced, only if tasks
from different cgroups are competing for CPU time. This means that if a cgroup
gets no competition, it gets all the CPU time it wants.

With the advent of ``cloud computing'' however, it has also become relevant to
facilitate enforcing upper bounds on the CPU time over a period of
time\cite{turner-et-al-2010}. This is facilitated similarly to the RT scheduler.

We choose to let the students spend all the CPU time they want, as long as fair
service is ensured for all students and staff. There is therefore only one
control file in this subsystem of interest to us:

\begin{description}[\setleftmargin{0.2in}\breaklabel\setlabelstyle{\tt}]

\item[cpu.shares] Show/set the relative CPU time share of a cgroup. Two cgroups
having share 100, will be given equal service. If one of the groups has share
200, it gets twice as much CPU time under a fully-loaded system. This control
file in the root cgroup, provides for a yard-stick for all other cgroups.

\end{description}

All these options do not allow us to hard limit the amount of CPU time used by
a cgroup in total. To our knowledge there is no ``natural'' way of doing this
in the Linux kernel. We must make due with limiting the wall-clock running time
of an untrusted program.

\subsection{cpuset}

The cpusets subsystem allows us to assign a set of CPU cores and a set of
memory nodes to a cgroup. This can be used to further partition system
resources from a fairly high level.

We choose to let the system be runnable on commodity hardware. With few
processor cores, and few memory nodes, this subsystem is of little use to us.

\subsection{devices}

The devices subsystem allows us to mandate access to device nodes (files) using
cgroups\cite{devices.txt}. The limits are enforced hierarchically using
whitelists --- a cgroup further down in the hierarchy cannot access devices to
which access has not been granted further up in the hierarchy.

A whitelist entry consists of four fields: the device node type, the major and
minor device node identifiers (2 fields), and an access specifier. The access
specifier is a sequence of characters, where \texttt{r} signifies read access,
\texttt{w} write access, and \texttt{m} device node creation.

The device node type is either \texttt{c} for character, \texttt{b} for block,
or \texttt{a} for all devices. \texttt{a} is a wildcard. Using it, discards all
other options, and implies full access to all devices of all types. This is
useful if you want to mandate universal access. The major and minor device
identifiers is what Linux uses to uniquely identify devices.  The device node
type, its major and minor identifiers of can be found using \texttt{stat}. 

\begin{lstlisting}
$ stat --format "type:%F, major:%t, minor:%T" /dev/urandom 
type:character special file, major:1, minor:9
\end{lstlisting}

The following control files facilitate device whitelist management:

\begin{description}[\setleftmargin{0.2in}\breaklabel\setlabelstyle{\tt}]

\item[devices.allow] Writing an entry to this file adds an entry to the device
whitelist. Reading is prohibited.

\item[devices.deny] Writing an entry to thisfile removes an entry from the
device whitelist. Reading is prohibited.

\item[devices.list] Shows the device whitelist. Writing is prohibited.

\end{description}

The format of a whitelist entry is \texttt{<type> <major>:<minor> <spec>},
where \texttt{<major>} and \texttt{<minor>} can be \texttt{*} indicating all
versions. Writing \texttt{a} to \texttt{devices.allow} or \texttt{devices.deny}
is the same as writing \texttt{a *:* rwm} to the same file.

\subsection{blkio}

The Block IO subsystem allows us to monitor and mandate access to I/O
operations on block devices using cgroups\cite{blkio-controller.txt}. The
monitoring parameters provide for insight into the I/O performance of a cgroup.
The access mandating parameters provide for proportional and absolute limits on
the number of I/O operations by a cgroup.

We choose to not provide access to block devices to students in general. This
is done due to the ease of implementation of a particular security policy.
Giving unmandated access to a block device, in theory, gives unmandated access
to all data on that device. Protecting data on such a device is a complicated
matter. Although SELinux, with it's per-inode restrictions, could presumably be
used to this end, it is easier to just not give access to block devices in
general.

As will be discussed later, this does not inhibit us in providing students with
a general purpose read/write file system.
