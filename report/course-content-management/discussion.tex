% vim: set spell:

\section{Discussion}

\subsection{Pull requests}

Our model of a student submission being a Git push to a student branch is not
an accurate model of modern software development. Often, a developer would work
on their own branch (as our students would), and then make a ``pull request''
to merge their changes into the master branch\footnote{Alternatively, a
developer might work on in their own repository, and then make a pull request
for their changes to be merged into the main project
repository\cite{bird-et-al-2009}.}.

Such pull requests make little sense in education where all students are
working on the same problem --- a scenario you'd often go to great lengths to
avoid in industry. Instead, students are always allowed to submit what they
have to their own branch. Code reviews are then done of snapshots of the
student branch.

Alternatively, we could have chosen to have students make a pull request to a
special ``submission branch'', with the other branch being a ``draft branch''.
(This would demand a more complicated access control layer.) The pull request
could then be accepted if the code passed automatic code review.

Unfortunately, it is sometimes instructional to give credit for an attempt at
solving the problem. There may even come a situation where the student has made
it to submit some basic working code in the submission branch but has a more
comprehensive (non-working) solution in their draft branch. In our model, the
commit and assessment history of a branch is sufficient to reveal when the code
had last worked.

\subsection{Responding to students}

Responding to students via a subdirectory in their private branch means that
the students have to pull from their branch before they can make a subsequent
submission (the race condition aside). This is good because it encourages
students to read feedback and not to push in the blind. This is bad because it
might inhibit quick (re)submissions (made within minutes): as practice shows,
this is frequent close to a deadline.

An alternative could be to distribute feedback in a separate private student
branch, which is not writable by students. This is easy to set up in gitolite,
but is more permissive of students pushing in the blind, ignoring all feedback.
It also adds to the complexity of the student's view of the system: some
students may fail to realise that feedback is being given at all.

As another alternative, feedback could be provided interactively, as part of a
Git push operation. For instance, in a \texttt{post-receive} or a
\texttt{post-update} Git hook. These hooks lets us run custom scripts after the
real work of a Git push is done. The benefit is that the connection to the
client is not closed until these scripts end, and standard output and error are
redirected to the client\cite{man-5-githooks}. Although this allows to present
the test results immediately, it is unclear where the test results should be
persisted.

\subsection{SSH Certificate Authority}

\label{section:gitolite-ssh-certificate-authority}

Our choice of gitolite as the access control layer for our Git server,
seemingly prohibits the use of an SSH certificate authority.

An SSH certificate authority is a separate server that certifies client public
keys.  This relinquishes a Git server of the need to keep public keys in an
authorized keys file, and allows to keep a centralized (hiearchical) registry
of client keys. Gitolite relies on the use of an authorized keys file.

Certificate authorities however, still have to forward client certificates to
the Git server. If forwarded on-demand, the certificate authority is a single
point of failure. If forwarded on occasion, certificate authorities are
functionally equivalent to using a Git repository over SSH, as with gitolite.
We therefore do not find this to be an inconvenience.

\subsection{Scalability}

\label{section:git-server-scalability}

We dedicate a Git branch to every student. To our knowledge, there is no
practical limit on the number of branches in a Git repository. If there is a
limit, it has to do with underlying file system limits, as every branch
requires a separate file in a particular subdirectory. This limit should be in
the manner of millions, and so not applicable in a course, or department
context.

Each time a client performs a Git operation, a connection to the Git server is
established. The server performs one of a limited set of (presumably, finite)
operations in a separate user session.

To our knowledge, there are no practical limits on the number of simultaneous
connections to a Linux server. The number of user sessions however, is bound by
the maximum number of processes per user. This limit can be found using:

\begin{lstlisting}
$ ulimit -u
\end{lstlisting}

Although this limit can be lifted, there is a more fundamental limit on the
total number of processes for a Linux system. This limit is typically 32768. It
should be increased with caution, and only if there are sufficient system
resources. The limit can be found using:

\begin{lstlisting}
$ cat /proc/sys/kernel/pid_max
\end{lstlisting}

A dedicated Git server should safely scale to a course, or a department,
provided sufficient memory, CPU, and disk resources and speeds. On a university
scale, it is advisable to use a Git server per department. OpenSSH operations
are fairly CPU intensive, and many simultaneous submissions may lead us to hit
the process number limits.
