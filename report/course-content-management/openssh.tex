% vim: set spell:

\section{OpenSSH}

\label{section:openssh}

OpenSSH is a free (as in free speech) version of the SSH connectivity
tools\cite{openssh-com-2014}.  The tools provide for secure encrypted
communication between untrusted hosts over an insecure network\cite{man-1-ssh}.
They include tools for user authentication, remote command execution, file
management, etc.

An OpenSSH host maintains a private/public key pair used to identify the host.
Upon connection, the host offers its public key to the client, in hope that the
client will accept it and (securely) proceed with authentication with the host.
If authenticated, the client is mapped to a particular user on the host.  After
some session preparation, the client, as that user, can start a session, i.e.
request a shell or the execution of a command.

One of the authentication methods supported by OpenSSH is using public key
cryptography. The idea is that each client creates a private/public key pair,
and informs the host of the public key over some otherwise secure channel, e.g.
using a trusted keyserver.

For any user on the host, a file can be created, e.g.
\texttt{\textasciitilde/.ssh/authorized\_keys}, listing the public keys of
those private/public key pairs that may be used to authenticate as that user.
The format of this file\cite{man-8-sshd}, allows to specify additional options
for each key. The options can be used to e.g. set a session-specific
environment variable, or replace the command executed once the user is
authenticated. The original command is then saved as the environment variable
\texttt{SSH\_ORIGINAL\_COMMAND}.

When using a Git server with OpenSSH, Git operations on the client, will
attempt execute Git operations on the host. Per-key options can be used to make
their execution dependent on the key used for authentication, e.g. performing
authorisation.
