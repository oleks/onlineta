% vim: set spell:

\section{Encrypted and Signed Git Repositories}

Git comes with a built-in a method to digitally sign tags (versions) and
commits. The intent is to allow us to ensure that particular tags or commits
come from particular individuals\cite{gerwitz-2013}.

We could use this to ensure that a particular submission indeed comes from a
particular student, but we cannot expect the students to be proficient enough
in Git to use signed tags or commits. Instead, we consider Git repositories
where commits are always, seamlessly signed.

At the core of Git is a key-value store\cite{chacon-2009}. All
version-controlled data, is retained in so-called Git objects, addressed by the
SHA-1 hash of their contents. We say that commit data to, and checkout data
from this data store.

Git retains diffs of data (compared to its earlier versions), rather than the
data itself. Data in a local Git repository is therefore saved twice, once on
the regular file system, and once as a sequence of diffs, retained as Git
objects. When performing a Git push operation to a remote Git repository, only
Git objects are ever sent along.

The \texttt{filter} Git attribute \cite{man-5-gitattributes} allows us to
filter the data before it is committed to, or when it is checked out of the
data store. This is achieved by two commands: \texttt{smudge} and
\texttt{clean}, respectively. The names suggest that the intended use filters
is to retain only ``clean'' data in the data store.

We can abuse the system a bit, taking inspiration from its own
documentation\cite{man-5-gitattributes}. We sign the data on commit and verify
the signature on checkout. Similarly, we can encrypt the data that goes into
the data store, and decrypt the data when it leaves the data store. This way,
student data stays private, even if their personal Git objects leak from the
Git server.

\subsection{Keys}

GNU Privacy Guard (GPG) is a suite of cryptographic tools. It supports, among
other things, private/public key cryptography, and boasts tools for trustworthy
distribution of public keys. GnuPG is often used to identify physical
individuals on the Internet. Indeed, the built-in Git tag and commit signing
features expect users to retain a GPG private/public key pair, and securely
distribute their public keys to all interested parties.

OpenSSL is an open-source implementation of the SSL and TLS protocols, used for
secure communication on the Internet. Unlike, GnuPG, OpenSSL is not intended
for the identification of physical individuals. OpenSSL however, is often used
to identify users on servers. For instance, OpenSSL is an underlies OpenSSH,
which we used to identify users on our Git server (see also
\referToSection{openssh}).

Both GnuPG and OpenSSL can be used to implement encrypted and signed Git
repositories. GnuPG however is more conventional for the identification of
individuals, and it is compatible with OpenSSH\cite{man-1-gpg-agent}. OpenSSL
is also very badly documented\footnote{\cite{man-1-openssl} does not even
mention that it can be used to encrypt, decrypt and sign data.} and has
recently suffered a severe drop in popularity due to a major
vulnerability\cite{cvedetails-com-2014c}.

Students can make due with securely informing us of their public GPG key, and
use this to both interact with our Git server, encrypt, and sign their
submissions.

We have not found a way to make public key distribution any more simple for
students, than using some kind of a key server, where keys are validated with
the participation from the teaching staff. One easier approach might be to make
use of their online university accounts and have them supply their keys on
their own in a way similar to Github. We have not had time to look into this.

\subsection{Collaboration}

It is easy enough to sign and encrypt the data for one user, but this defeats
many of the useful collaboration features of Git. Besides, we must let the
staff and the assessment engine see the student data as well.

One way in which we can provide for collaborative, encrypted Git repositories
is to encrypt the data for multiple recipients. This is straight-forward using
the OpenPGP suite of tools\cite{man-1-gpg2}. For an implementation of this, see
the attached \texttt{./src/git-init.sh}.

The pitfall of this approach in general is that new individuals cannot easily
join as collaborators as the entire history has not been encrypted for them to
see.  This is not a problem if we do not let students collaborate on
submissions, but again, this defeats many of the useful collaboration features
of Git.

% students can collaborate on a branch different than their submission branch,
% but then they can't use our Git server as a remote data store.

Another limiting factor to student collaboration is our choice of having
students submit to individual branches. One way to let students collaborate, is
to have them be allocated a group branch by the staff.
