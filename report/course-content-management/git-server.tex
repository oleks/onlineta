% vim: set spell:

\section{Git Server}

The Git server serves as a data store and gateway between students and staff.
In the following sections, we refer to students and staff collectively simply
as clients, connecting to our host, the Git server. In
\referToSection{git-server-scalability} we will discuss why we can make due
with just one Git server.

\subsection{Why Git?}

Git is a popular\cite{ohloh-2014}, free, and open source distributed version
control and source code management system\cite{git-2014}. Although perhaps not
the ideal system for all intents and purposes, it is an excellent example that
has cemented itself in both the open source community, academia and
industry\cite{gitprojects-2014}.

% Version control and code review are some of the Core- Tier1 and Tier2
% elements in \cite{cs-curricula-2013}. They are highly suggested topics for
% any undergraduate Computer Science programme.

We hypothesize that using Git for programming assignments can spur the learning
of some of the workflow of modern software development. Ideally, students
collaborate on assignments, while teaching staff offer code reviews, all as if
it were a real software development project.

Git with authentication over SSH is an easy way to provide a scalable, online
general purpose data store and gateway, having fine-grained and reliable
authentication and authorisation procedures.

\subsection{Course as a Repository}

A Git server manages Git repositories. Let a course be represented by a single
Git repository. Let a commit by a student, to this repository, be a submission.
The assignment to which the submission is made, depends on the where in the
repository a commit makes a change.

A Git repository has one or more branches. We choose to let one branch - the
master branch - be used for the distribution of course content and assignments
by the teaching staff. To make submissions, students create branches in their
name and push their changes to these branches onto the server.

Assessment of a student submission is provided in a special subdirectory on
their private branch, and optionally, reported to the student upon submission.
A submission is a particular commit by the student to their branch.

In such an infrastructure it is important that students are not allowed to push
to the master branch, or to other student branches. At the same time, teaching
staff should be allowed to push to both the master (to provide content and
assignments) and student branches (to provide feedback). Last but not least, we
would like to let everyone see course content and assignments, but prohibit
them in seeing student submissions, or pushing to any of the branches.

Such fine-grained authentication and authorisation can be achieved through
OpenSSH and Git hooks.
