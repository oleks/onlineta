% vim: set spell:

\section{Git Server}

The Git server serves as a data store and gateway between students and staff.
In the following we refer to student and staff collectively simply as clients,
connecting to our host, the Git server. In
\referToSection{git-server-scalability} we will discuss why we can make due
with just one Git server.

\subsection{Why Git?}

Git is a popular\cite{ohloh-2014}, free, and open source distributed version
control and source code management system\cite{git-2014}. Although perhaps not
the ideal system for all intents and purposes, it is an excellent example that
has cemented itself in both the open source community, academia and
industry\cite{gitprojects-2014}.

% Version control and code review are some of the Core- Tier1 and Tier2
% elements in \cite{cs-curricula-2013}. They are highly suggested topics for
% any undergraduate Computer Science programme.

We hypothesize that using Git for programming assignments can spur the learning
of some of the workflow of modern software development. Ideally, students
collaborate on assignments, while teaching staff offer code reviews, all as if
it were a real software development project.

Git with authentication over SSH is an easy way to provide a scalable, online
general purpose data store and gateway, having fine-grained and reliable
authentication and authorisation procedures.

\subsection{Course as a Repository}

A Git server manages Git repositories. We choose to let a course be represented
by a Git repository.

A Git repository has one or more branches. We choose to let one branch - the
master branch - be used for the distribution of course content and assignments
by the teaching staff. To make submissions, students create branches in their
name and push their changes to these branches onto the server.

Assessment of a student submission is provided in a special subdirectory on
their private branch, and optionally, reported to the student upon submission.
A submission is a particular commit by the student to their branch.

In such an infrastructure it is important that students are not allowed to push
to the master branch, or to other student branches. At the same time, teaching
staff should be allowed to push to both the master (to provide content and
assignments) and student branches (to provide feedback). Last but not least, we
would like to let everyone see course content and assignments, but prohibit
them in seeing student submissions, or pushing to any of the branches.

Such fine-grained authentication and authorisation can be achieved through
OpenSSH and Git hooks.

\subsection{OpenSSH}

OpenSSH is a free (as in free speech) version of the SSH connectivity
tools\cite{openssh-com-2014}.  The tools provide for secure encrypted
communication between untrusted hosts over an insecure network\cite{man-1-ssh}.
They include tools for user authentication, remote command execution, file
management, etc.

An OpenSSH host maintains a private/public key pair used to identify the host.
Upon connection, the host offers its public key to the client, in hope that the
client will accept it and (securely) proceed with authentication with the host.
If authenticated, the client is mapped to a particular user on the host.  After
some session preparation, the client, as that user, can start a session, i.e.
request a shell or the execution of a command.

One of the authentication methods supported by OpenSSH is using public key
cryptography. The idea is that each client creates a private/public key pair,
and informs the host of the public key over some otherwise secure channel, e.g.
using a trusted keyserver.

For any user on the host, a file can be created, e.g.
\texttt{\textasciitilde/.ssh/authorized\_keys}, listing the public keys of
those private/public key pairs that may be used to authenticate as that user.
The format of this file\cite{man-8-sshd}, allows to specify additional options
for each key. The options can be used to e.g. set a session-specific
environment variable, or replace the command executed once the user is
authenticated. The original command is then saved as the environment variable
\texttt{SSH\_ORIGINAL\_COMMAND}.

When using a Git server with OpenSSH, Git operations on the client, will
attempt execute Git operations on the host. Per-key options can be used to make
their execution dependent on the key used for authentication, e.g. performing
authorisation.

\subsection{Git Hooks}

Git hooks is a Git mechanism for executing custom scripts when important events
happen for Git repository\cite{man-5-githooks}. The scripts can control in how
far certain Git operations should succeed. A Git hook is an adequately named
executable placed in a special subdirectory in a local Git repository. Git
hooks are not part of the version-controlled code base of a repository.

For instance, the \texttt{update} hook is executed whenever the client attempts
to push something to a branch. The client has already been authenticated, but
no changes have yet been made. The hook is passed adequate arguments to
identify the branch or tag being updated and the update taking place. If this
hook exits with a non-zero exit value, the update will duly fail.

Other Git hooks are discussed in subsequent sections.

\subsection{Users}

When using a Git server with OpenSSH, clients must be mapped to users on the
host. There are at least two options for the mapping: each client gets their
own user, or all clients map to the same user.

The first option has higher administration costs, but gives perhaps more
fine-grained access control. The second option is generally more popular
because of less cluttering of the UTS namespace. Additional tools, like
gitolite, are instead used to provide a fine-grained access control layer. We
have also chosen to go with this option.

\subsection{Gitolite}

Gitolite is an access control layer on top of Git\cite{gitolite-com-2014a}.
Gitolite leverages the features of OpenSSH and Git hooks, as discussed above,
to provide fine-grained authentication and
authorisation\cite{gitolite-com-2014b}.

Gitolite is used in multiple communities with high-stakes projects, such as
Fedora, KDE, Gentoo, and kernel.org\cite{gitolite-com-2014c}. Among the reasons
for choosing gitolite, kernel.org lists\cite{kernel-org-2014} ``well maintained
and supported code base'', ``responsive development'', ``broad and diverse
install base'', and ``had undergone an external code
review''\cite{gitolite-google-group-2011}.

There are also other tools out there, such as  Gerrit\footnote{See also
\url{https://code.google.com/p/gerrit/}.} and Stash\footnote{See also
\url{https://www.atlassian.com/software/stash}.}. Both of these provide a lot
more than a simple access control layer.

\subsubsection{Administration}

Administration of the gitolite happens through a special Git repository on the
Git server. There are three important elements to this repository:

\begin{enumerate}

\item The \texttt{./keys} subdirectory which contains the public keys of all
users of the system, thereby defining the users. The names of the key files
designate the user names\cite{gitolite-com-2014d}.

\item The \texttt{./conf/gitolite.conf} configuration file. This file defines
the repositories, and the users' permissions wrt. those repositories.

\item The \texttt{post-update} Git hook on the server side, parsing the above
keys subdirectory and config file, making adequate changes to the server
repositories.

\end{enumerate}

It is important that access to this repository is safely guarded as it gives
complete control over the users and repositories on the Git server.

\subsubsection{Permissions}

User permissions in gitolite can whitelisted for an entire repository, a
branch, tag, or even a subfolder. Users may be granted, read, write,
read-write, and even forced write permissions (more on this in the next
section).  There are some even more fine-grained
permissions\cite{gitolite-com-2014e}, but we will not be concerned with them
here.

\subsection{Attack Surface}

In this section we discuss the possible ways to attempt to attack a Git server.
This is distinct from attempting to attack our sandboxed environments in that
the Git server is not sandboxed in any useful way. It serves as a general
purpose data store and gateway between students and staff.

\subsubsection{Login shell}

It is often important with Git servers to disallow clients' shell requests.
This is typically achieved by setting the user's login shell to something
non-permissive, e.g. a \cite{man-1-git-shell}.

This set up is perhaps a bit superfluous, as gitolite disables interactive
shell login via the authorized keys file. Never-the-less it is a good extra
level of security, as the login shell of any user can only be modified by a
privileged user, which the \texttt{git} user is not.

\subsubsection{Session preparation dialog}

When a client is authenticated with OpenSSH, but before a user session starts,
the client and the host enter into a session preparation dialog.

The client can request a pseudo-tty (e.g. interactive shell), forwarding X11
connections (e.g. remote desktop), forwarding TCP connections (e.g. virtual
private networking), or forwarding the authentication agent connection over the
secure channel (e.g. using the secure connection to establish other secure
connections).

All these options open up the attack surface of our Git server. Fortunately,
all of these session dialog options can be disabled for any key in the
authorized keys file\cite{man-8-sshd}. By default, gitolite disables all of
these options for all keys.

\subsubsection{Forced push and rewriting history}\label{section:forced-push}

Git has a, somewhat controversial\cite{torvalds-2007,hamano-2009,rego-2013},
forced push feature. This bypasses the check that the remote ref being updated
should be an ancestor of the local ref used to overwrite
it\cite{man-1-git-push}. Meaning that the branch being updated should be the
strict base of the update.

Forced push is dangerous because it incautiously overwrites history and can
thereby inhibit assessment or even modify student records.

This is mitigated for by gitolite permissions. Students and staff are simply
not allowed to perform a forced push, supporting the notion introduced in
\referToSection{analysis:other-notions}, that course data is never deleted.
This means that students cannot e.g. amend to a commit that they have already
pushed to the server. The students are encouraged to use
\cite{man-1-git-revert} instead.

\subsubsection{Git, OpenSSH, and Perl}

Despite its popularity, relatively few vulnerabilities have ever been found
outside of the Git development team\cite{cvedetails-com-2014a}.

The security of Git (out of the box) however, depends on the sensibility of the
developers involved. Impersonification and private key leaks are not always
well guarded against\cite{gerwitz-2013}, especially with the advent of modern
Git hosting services\cite{homakov-2012,huang-2013,homakov-2014}. It is the
purpose of our Git server to serve as a guard against impersonification.
Private key leaks are to be guarded against by the students themselves. 

OpenSSH has also had relatively few vulnerabilities discovered outside of the
OpenSSH development team\cite{cvedetails-com-2014b}. However, the underlying
OpenSSL has been a lot less fortunate\cite{cvedetails-com-2014c}.

Perl has only been a bit less fortunate\cite{cvedetails-com-2014d}.

The referenced material does not cover all of the underlying libraries of the
software. However, all of the above are popular pieces of software on public
facing web servers.  Their security therefore, is a matter of grave public
concern.

\subsection{Discussion}

\subsubsection{Pull requests}

Our model of a student submission being a Git push to a student branch is not
an accurate model of modern software development. Often, a developer would work
on their own branch (as our students would), and then make a ``pull request''
to merge their changes into the master branch\footnote{Alternatively, a
developer might work on in their own repository, and then make a pull request
for their changes to be merged into the main project
repository\cite{bird-et-al-2009}.}.

Such pull requests make little sense in education where all students are
working on the same problem --- a scenario you'd often go to great lengths to
avoid in industry. Instead, students are always allowed to submit what they
have to their own branch. Code reviews are then done of snapshots of the
student branch.

Alternatively, we could have chosen to have students make a pull request to a
special ``submission branch'', with the other branch being a ``draft branch''.
(This would demand a more complicated access control layer.) The pull request
could then be accepted if the code passed automatic code review.

Unfortunately, it is sometimes instructional to give credit for an attempt at
solving the problem. There may even come a situation where the student has made
it to submit some basic working code in the submission branch but has a more
comprehensive (non-working) solution in their draft branch. In our model, the
commit and assessment history of a branch is sufficient to reveal when the code
had last worked.

\subsubsection{Responding to students}

Responding to students via a subdirectory in their private branch means that
the students have to pull from their branch before they can make a subsequent
submission (the race condition aside). This is good because it encourages
students to read feedback and not to push in the blind. This is bad because it
might inhibit quick (re)submissions (made within minutes): as practice shows,
this is frequent close to a deadline.

An alternative could be to distribute feedback in a separate private student
branch, which is not writable by students. This is easy to set up in gitolite,
but is more permissive of students pushing in the blind, ignoring all feedback.
It also adds to the complexity of the student's view of the system: some
students may fail to realise that feedback is being given at all.

As another alternative, feedback could be provided interactively, as part of a
Git push operation. For instance, in a \texttt{post-receive} or a
\texttt{post-update} Git hook. These hooks lets us run custom scripts after the
real work of a Git push is done. The benefit is that the connection to the
client is not closed until these scripts end, and standard output and error are
redirected to the client\cite{man-5-githooks}. Although this allows to present
the test results immediately, it is unclear where the test results should be
persisted.

\subsubsection{SSH Certificate Authority}

\label{section:gitolite-ssh-certificate-authority}

Our choice of gitolite as the access control layer for our Git server,
seemingly prohibits the use of an SSH certificate authority.

An SSH certificate authority is a separate server that certifies client public
keys.  This relinquishes a Git server of the need to keep public keys in an
authorized keys file, and allows to keep a centralized (hiearchical) registry
of client keys. Gitolite relies on the use of an authorized keys file.

Certificate authorities however, still have to forward client certificates to
the Git server. If forwarded on-demand, the certificate authority is a single
point of failure. If forwarded on occasion, certificate authorities are
functionally equivalent to using a Git repository over SSH, as with gitolite.
We therefore do not find this to be an inconvenience.

\subsubsection{Scalability}

\label{section:git-server-scalability}

We dedicate a Git branch to every student. To our knowledge, there is no
practical limit on the number of branches in a Git repository. If there is a
limit, it has to do with underlying file system limits, as every branch
requires a separate file in a particular subdirectory. This limit should be in
the manner of millions, and so not applicable in a course, or department
context.

Each time a client performs a Git operation, a connection to the Git server is
established. The server performs one of a limited set of (presumably, finite)
operations in a separate user session.

To our knowledge, there are no practical limits on the number of simultaneous
connections to a Linux server. The number of user sessions however, is bound by
the maximum number of processes per user. This limit can be found using:

\begin{lstlisting}
$ ulimit -u
\end{lstlisting}

Although this limit can be lifted, there is a more fundamental limit on the
total number of processes for a Linux system. This limit is typically 32768. It
should be increased with caution, and only if there are sufficient system
resources. The limit can be found using:

\begin{lstlisting}
$ cat /proc/sys/kernel/pid_max
\end{lstlisting}

A dedicated Git server should safely scale to a course, or a department,
provided sufficient memory, CPU, and disk resources and speeds. On a university
scale, it is advisable to use a Git server per department. OpenSSH operations
are fairly CPU intensive, and many simultaneous submissions may lead us to hit
the process number limits.
