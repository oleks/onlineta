% vim: set spell:

\section{Gitolite}

Gitolite is an access control layer on top of Git\cite{gitolite-com-2014a}.
Gitolite leverages the features of OpenSSH and Git hooks, as discussed above,
to provide fine-grained authentication and
authorisation\cite{gitolite-com-2014b}.

Gitolite is used in multiple communities with high-stakes projects, such as
Fedora, KDE, Gentoo, and kernel.org\cite{gitolite-com-2014c}. Among the reasons
for choosing gitolite, kernel.org lists\cite{kernel-org-2014} ``well maintained
and supported code base'', ``responsive development'', ``broad and diverse
install base'', and ``had undergone an external code
review''\cite{gitolite-google-group-2011}.

There are also other tools out there, such as  Gerrit\footnote{See also
\url{https://code.google.com/p/gerrit/}.} and Stash\footnote{See also
\url{https://www.atlassian.com/software/stash}.}. Both of these provide a lot
more than a simple access control layer.

\subsection{Administration}

Administration of the gitolite happens through a special Git repository on the
Git server. There are three important elements to this repository:

\begin{enumerate}

\item The \texttt{./keys} subdirectory which contains the public keys of all
users of the system, thereby defining the users. The names of the key files
induce the users of the Git server\cite{gitolite-com-2014d}.

\item The \texttt{./conf/gitolite.conf} configuration file. This file defines
the repositories, and the users' permissions wrt. those repositories.

\item The \texttt{post-update} Git hook on the server side, parsing the above
keys subdirectory and config file, making adequate changes to the server
repositories.

\end{enumerate}

It is important that access to this repository is safely guarded as it gives
complete control over the users and repositories on the Git server.

\subsection{Permissions}

User permissions in gitolite can whitelisted for an entire repository, a
branch, tag, or even a subfolder. Users may be granted, read, write,
read-write, and even forced write permissions (more on this in the next
section).  There are some even more fine-grained
permissions\cite{gitolite-com-2014e}, but we will not be concerned with them
here.
