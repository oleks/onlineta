% vim: set spell:

\section{Attack Surface}

In this section, we discuss some of the possible ways that an attacker can
approach our Git server.  This is distinct from attempting to attack our
sandboxed environments in that the Git server is not sandboxed in any relevant
way. It serves as a general purpose data store and gateway between students and
staff.

\subsection{Login shell}

It is often important with Git servers to disallow clients' shell requests.
This is typically achieved by setting the user's login shell to something
non-permissive, e.g. a \cite{man-1-git-shell}, and disabling login for that
user in general.

This set up is perhaps a bit superfluous, as gitolite disables interactive
shell login via the authorized keys file. Never-the-less it is a good extra
level of security, as the login shell of any user can only be modified by a
privileged user, which the \texttt{git} user is not.

\subsection{Session preparation dialog}

When a client is authenticated with OpenSSH, but before a user session starts,
the client and the host enter into a session preparation dialog.

The client can request a pseudo-tty (e.g. interactive shell), forwarding X11
connections (e.g. remote desktop), forwarding TCP connections (e.g. virtual
private networking), or forwarding the authentication agent connection over the
secure channel (e.g. using the secure connection to establish other secure
connections).

All these options open up the attack surface of our Git server. Fortunately,
all of these session dialog options can be disabled for any key in the
authorized keys file\cite{man-8-sshd}. By default, gitolite disables all of
these options for all keys.

\subsection{Forced push and rewriting history}\label{section:forced-push}

Git has a, somewhat controversial\cite{torvalds-2007,hamano-2009,rego-2013},
forced push feature. This bypasses the check that the remote ref being updated
should be an ancestor of the local ref used to overwrite
it\cite{man-1-git-push}. Meaning that the branch being updated should be the
strict base of the update.

Forced push is dangerous because it incautiously overwrites history and can
thereby inhibit assessment or even modify student records.

This is mitigated for by gitolite permissions. Students and staff are simply
not allowed to perform a forced push, supporting the notion introduced in
\referToSection{analysis:other-notions}, that course data is never deleted.
This means that students cannot e.g. amend to a commit that they have already
pushed to the server. The students are encouraged to use
\cite{man-1-git-revert} instead.

\subsection{Git, OpenSSH, and Perl}

Despite its popularity, relatively few vulnerabilities have ever been found
outside of the Git development team\cite{cvedetails-com-2014a}.

The security of Git (out of the box) however, depends on the sensibility of the
developers involved. Impersonification and private key leaks are not always
well guarded against\cite{gerwitz-2013}, especially with the advent of modern
Git hosting services\cite{homakov-2012,huang-2013,homakov-2014}. It is the
purpose of our Git server to serve as a guard against impersonification.
Private key leaks are to be guarded against by the students themselves. 

OpenSSH has also had relatively few vulnerabilities discovered outside of the
OpenSSH development team\cite{cvedetails-com-2014b}. However, the underlying
OpenSSL has been a lot less fortunate\cite{cvedetails-com-2014c}.

Perl has only been a bit less fortunate\cite{cvedetails-com-2014d}.

The referenced material does not cover all of the underlying libraries of the
software. However, all of the above are popular pieces of software on public
facing web servers.  Their security therefore, is a matter of grave public
concern.
