% vim: set spell:

\section{Assessment in Programming Education}

%Computer Science is the study of computable structures and processes. Computer
%Science professionals are (among other things) expected to be eloquent in the
%theory and practice of computer programming\cite{cs-curricula-2013}. To this
%end, practical work in the form of computer programming is a popular basis for
%assessment in Computer Science\cite{carter-et-al-2003}.

Computer programming is concerned with the construction of programs executable
by a computer, solving a particular problem.

To be executable by computers, computer programs are often written in highly
rigorous languages, and so are amenable to assessment with the assistance of
computers. The assessment of computer programs is a wide area of research and
industry, known as software verification or quality assurance.

The use of computers for structuring a submission for a programming assignment
can provide for some rigor. Students can be, and often are, expected to acquire
skills in using various programming tools on their own account, or with some
facilitation from their teachers.

More often than not, computer programs are written for others than the original
authors to use, test, and maintain.  Computer programming is therefore also
often concerned with the construction of usable, testable, readable, and
maintainable programs. To this end, for particular assignments, we may wish for
students to practice following a particular style guide, or practice the use of
a particular programming abstraction.

\subsection{Programming Languages}

Assignments in compute programming will often ask students to write computer
programs in one of a range of different programming languages. Programming
languages come and go, and no language has emerged as the predominant one for
teaching programming. It is most useful, therefore, to facilitate assessment of
programs written in any conceivable programming language.

One way to stay programming language agnostic, is to facilitate assessment at
the operating system level. This requires looking into how student programs,
and various analyses thereof, can be run in safe and fair environments, and how
the various elements of an assessment engine can communicate via the operating
system in a safe and secure way.

\subsection{Environments}

Assignments in computer programming will often ask students to write programs
that operate within environments with particular permissions and restrictions.
We wish to facilitate such permissions and ensure that the restrictions are
adhered to. For instance, we may wish to permit that students can write to a
particular file, but restrict how many I/O operations they may perform in
total.

\subsection{Courses}

Confined to programming education, we may still hold a wide range of different
courses, concerned with a particular programming language, particular
programming techniques, or managing system resources. Considering which of
these courses pose the biggest challenges to computer-assissted assessment,
systems programming in C comes to mind.

C is ubiquitous on modern computer architectures, with the folklore that the
first thing you should write for your new processor is a C compiler. We
conjecture, that anything written for execution on a modern computer, with
sufficient effort, can also be written in C. Especially because we can write
assembly in C. C is also ubiquitous on modern operating systems. Perhaps the
most basic API that a modern operating system offers is a C API. If not, we can
again resort to writing assembly in C.

Systems programming typically also involves a wide range of necessary
permissions and restrictions. Especially if student programs are not written
for some idealised, virtual environment (e.g. Buenos), but are intended for use
in the context of a real system.

Systems programs may run into faults, make obscure system calls, access
devices, perform I/O, not to mention run high on memory use and CPU time. We
conjecture, that if we can facilitate safe and fair environments for a course
in systems programming in C, we can facilitate safe and fair environments for
most courses in computer programming in general.

We will however, make a few simplifying assumptions as we seek to meet certain
safety, fairness, and security requirements. For instanc, we will assume that
students do not need access to networking facilities, or access to more than a
handful of character devices.
