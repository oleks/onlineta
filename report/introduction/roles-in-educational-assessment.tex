% vim: set spell:

\section{Roles in Educational Assessment}

\label{section:analysis:roles-in-educational-assessment}

In the previous sections we introduced the concepts of a student and teacher,
enrolled for a course. In this section we expand on the roles of course
enrollees.

\subsection{Teachers}

Teachers are enrolled for a course for the purposes of teaching. Teaching is
the expediting of learning. Students learn on their own, but teachers
facilitate learning\cite{skinner-1965}. The means of facilitation however, vary
greatly throughout the discipline\cite{ramsden-1992,kember-1997}.  Most would
unite the role of teaching with information delivery and assessment.

A non-empty set of teachers is always held responsible for a course. They are
responsible for ensuring that the students acquire the knowledge or skills
defined for the course. Other teachers may be involved in the course, aiding
the course responsible teachers.

In hope of students' learning, the teachers devise means of delivering
informative content, and assessing in how far students have acquired the said
knowledge or skills. Various techniques in both information delivery and
assessment are used to facilitate and encourage learning. We are not so
concerned with information delivery, as with assessment. Never-the-less,
assignments still have to be delivered to students.

Since teachers facilitate student learning and also their final assessment,
teachers exert great authority over students. Teachers decide whether a student
passes or fails a course, what grade they get, and how tough the course is.

\subsubsection{Teaching assistants}

Teaching assistants assist in teaching responsibilities. They are teaching
subordinates of teachers. They exert some authority over students, but are
often limited in their authority when it comes to important summative
assessments. The result is that teaching assistants perform much of the
formative assessment, and provide guiding remarks either for the purposes of
feedback or to ease important summative assessments for the teachers.

Teaching assistants come about as a scaling mechanism. Once the number of
students enrolled for a course exceeds a certain multiple of teachers, certain
means of information delivery and assessment are simply infeasible for the
teachers. Instead of hiring more teachers, the strategy is often to rely on
some methods of information delivery and assessment that work in large numbers,
and rely on teaching assistants for the rest. Teaching assistants are cheaper,
often less qualified, staff who assist in teaching responsibilities.

We refer to teachers and teaching assistants collectively as teaching staff, or
simply, staff.

\subsubsection{Trust}

Under the authority of the course responsible, the teachers and teaching
assistants can be trusted to make fairly good assessments of student
performance.

In various disciplines, especially in Computer Science, assessment involves
some fairly mechanical processes. Teachers and teaching assistants quickly
develop a desire to get a computer to do what is otherwise fairly laborious
work. To this end they write programs and set up processes that analyse student
submissions.

Teachers and teaching assistants can in general be trusted to write programs
that often perform the analyses correctly, but bugs may lure. It is desirable
that automated assessments can be validated and reverted by the teachers and
teaching assistants. It is also important that their programming mistakes don't
get in the way of student learning, by e.g. consuming all system resources.

\subsection{Students}

Students are enrolled for a course for the purposes of learning --- the
acquisition of knowledge or skills. Learning is a qualitative change of an
individual's view of the world\cite{ramsden-1992}.

\subsubsection{Trust}

Ideally, students engage in learning in hope of being somehow enlightened or
trained for solving particular problems. Realistically, student motivation
varies greatly, and may even change throughout a course. Some are motivated by
the mere idea of a good final assessment, e.g. to impress an perspective
employer, friends, or family.

Unlike teachers, students have few direct responsibilities. In how far they
actually acquire various knowledge or skills, often only bears consequences
much later in life. Students may therefore be interested in the attainment of
deceptive assessments, claiming that they have acquired various knowledge or
skills, when in reality they were never assessed to have done so. This involves
hacking the assessment process with the intent of cheating, or faking
statements of accomplishment.

Students also can not be trusted to always be nice to their peers. Due to
various motivations they may wish to abuse or tamper with the assessments of
other students. Without further ethical consideration, it is perhaps best to
let assessments be a personal matter.

More often however, students just make mistakes, sometimes big mistakes. They
cannot be trusted to write good programs in their first or any subsequent
attempt.  One of the benefits of online automated assessment, can be that
students can fail fast and fail often.  This may facilitate learning.

\subsection{External Examiners}

External examiners facilitate the quality assurance of assessment. An external
examiner's participation in an assessment may vary from mere observation to
avid participation. An external examiner therefore may need varying access to
the elements of a course and assessments.

\subsection{General Public}

The general public includes those who are ultimately interested in the quality
of education and the quality of assessment therein. This includes both
perspective students, future employers, the politically concious, etc. The
intent being to see if the education lives up to social expectations, demands
of the labour market, political promises, etc.

Privacy and anonymity is often a matter public concern. If access is granted to
the general public, it should only identify those who may reasonably be held
responsible for an possible shortcomings. Also, issues of
copyright have to be taken into account.

As the general public would assess education, and not students, students should
not be personally identifiable by the general public. In how far teachers and
teaching assistants may reasonably be held responsible by the general public
for the possible shortcomings, may be a matter of policy of the overarching
authority (e.g.  a university).

As students typically own the content they produce, individual student work or
commentary should not be made available to the general public. In how far
teachers and teaching assistants own the content they produce, may again be a
matter of policy of the overarching authority.
