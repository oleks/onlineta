% vim: set spell:

\section{Related Work}

\cite{sclater-howie-2003}  arrive at a set of ``ultimate'' requirements for an
online assessment engine. Their requirements represent the wishes of a focus
group at a particular institution, who have worked with similar engines before.
It is worth noting, that the focus group does not include students --- an
important user group.

Harvard has recently done a major redesign of their introductory Computer
Science course, CS50\cite{malan-2010a}. They've moved to securely executing
source code coming from students, and even arbitrary users on the
Internet\cite{malan-2010b,malan-2013}. Many of the techniques they use, are
(incidentally) also employed here. We see similar work going on at the
Standford Computer Security Lab\cite{stefan-2013}.

A related area of interest are so-called ``online judges'' for programming
competitions. The stark difference stands in that an online judge seeks to find
the \emph{first} contestants pass an assignment, whereas an online TA aids in
getting \emph{all} students to eventually pass an assignment. The parallels can
be drawn in that the incentives to use (and abuse) an ``online'' judge, are
often similar to the incentives to use (and abuse) an ``online'' TA. A lot of
work has been done in this area to provide for lightweight, safe, secure, and
fair sandboxes for running arbitrary user code. See e.g. \cite{domjudge-2014}.
