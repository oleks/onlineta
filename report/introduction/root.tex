% vim: set spell:

\chapter{Introduction}

\begin{quotation}

\footnotesize\sffamily\itshape

\begin{flushright}

In any teaching of the application of computers it is essential to have the
students do practical programming problems and to grade their results. Such
grading should consider both the formal correctness and the performance of the
programs and tends to become difficult and time consuming as soon as the
teaching is beyond the most elementary level.  The possibility of using the
computer to help in this task therefore soon suggests itself.

\smallbreak

\upshape

--- PETER NAUR, {\itshape BIT 4} (1964)

\end{flushright}

\end{quotation}

The above quote gives about as concise an introduction to the problem at hand
as we can hope to give. Being published in 1964, it also indicates that we are
about to embark upon a fairly long-standing problem --- we do not intend to
solve it here. Instead, we take a few first steps, enabling future work.

In this chapter, we make a few grounding definitions, discuss what programming
education looks like today, and discuss some of the reasons why a solution
remains elusive. If you feel confident about a section heading in this chapter,
feel safe to skip the section, but consider returning to it if in doubt about
the terminology in a subsequent chapter.

% vim: set spell:

\section{Assessment in Education}

\label{section:assessment-in-education}

% Let us assume that a principal goal of education is to bring about learning
% with understanding\cite{formative-vs-summative}.

Assessment, or evaluation, of people is concerned with obtaining information
about their knowledge, attitudes, or skills. Assessment in education is usually
concerned with obtaining information about students and their
learning\footnote{Sometimes, assessment in education is instead concerned with
evaluating teachers and their teaching. We will not be concerned with this form
of assessment here.}\cite{ramsden-1992,assessment-and-burnout}. This is done
with the intent of providing feedback or certification, performing selection or
comparison, improving learning processes, etc.\cite{first-ten-years}.

\subsection{Categorising Assessment}

There are two principal categories of assessment: formative and summative.  The
definition of each category varies somewhat in educational
research\cite{bloom1971, sadler1989, formative-vs-summative}, and their mutual
compatibility is questionable\cite{butler1988}. Our intent is not to advise on
the matter, but to aid in performing assessment, regardless of the flavour.

Let us therefore adopt a primitive distinction, which still supports the
purposes of our further analysis:

\begin{description}[\setleftmargin{60pt}\setlabelstyle{\bf}]

\item[Formative] A student's strengths and weaknesses are documented in free
form. Formative assessments are qualitative and non-standard\-ised: they are
aimed at measuring the quality of a student's learning, rather than whether
they live up to particular criteria. 

\item[Summative] A student is ranked on some well-defined scale, at some
well-defined intervals, based on some well-defined criteria. Summative
assessments are often compoundable and comparable. They may allow to deduce
holistic summative assessments of students or groups, quantitatively measure
student progress, etc.

\end{description}

Formative assessment necessitates the ability to perform personalised
assessments, whereas summative assessment demands the ability to specify
standards and perform standardised assessments.

There are other forms of assessment: diagnostic assessment, self-assessment,
peer-assessment, etc. \cite{bull-mckenna-2004,topping-1998}  These forms of
assessment vary along formative/summative dimensions, but primarily differ in
terms of when, by whom, and of whom the assessment is made.

\subsection{Feedback}

\label{section:assessment-in-computer-science:feedback}

Feedback is information about the difference between the reference level and
the actual level of some parameter which is used to remedy the difference in
some way\cite{ramaprasad1989}.

Feedback is an important bi-product of assessment in
education\cite{black-william-1998}. Ideally, feedback informs the student of
the quality of their work, outlines key errors, provides corrective guidance,
and encourages further student learning. To be so, it is important that
feedback is understandable, timely, and acted upon by
students\cite{gibbs-simpson-2004}.

These requirements are an active area of educational research, and one aiding
approach is to use computer-assisted assessment.

\subsection{Computer-Assisted Assessment}

Computer-assisted assessment is the form of assessment performed with the
assistance of computers\cite{conole-warburton-2005}. The benefit of using
computers is ideally, fast, interactive, consistent, and unbiased
assessment\cite{ala-mutka-2005}. The requirement is that the perceived student
performance can be encoded in some useful digital format.

This requirement however, has proven evasive. Free form performances, such as
essays or oral presentations, are still hard to assess
automatically\cite{valenti-et-al-2003}.  On the other hand, it is questionable
in how far easily assessable performances, such as, multiple-choice
questionnaires, are appropriate for assessment in higher
education\cite{conole-warburton-2005}.

We conjecture, that in how far computers can assist in assessment, depends on
how ``rigorous'' the student performance can be expected to be.  We formalise
this notion in the following sections.

\subsection{Assignment}

An assignment is a request for someone to perform a particular job.  An
assignment in education is a request for a student to make a performance, and
often, to provide a record thereof. One purpose of an assignment is to provide
basis for an assessment. The request therefore, often includes a specification
of what the assessment will be based on, and in what time frame the assignment
should be completed in order to be assessed.

\subsection{Submission}

A submission is a record of student performance, submitted for the purposes of
assessment. A digital submission is a digital encoding of such a record.
Digital submissions are amenable to assessment with the assistance of
computers, and thus of most interest to us.

\subsection{Rigor}

We say that the more features of interest can be extrapolated from a data
structure using efficient algorithms, the more ``rigorous'' the data structure.
A data source is rigorous if the data it delivers is rigorously structured, and
the same suite of algorithms can be used for all the data it delivers.

We conjecture that in how far computers can assist in assessment depends on how
rigorous students, the source of submissions, can be expected to be, where the
features of interest depend on the assignment at hand.

The use of computers for structuring submissions can sometimes provide for high
rigor. For instance, in a multiple-choice test, a computer may present the
student with the questions and options. The student may then respond to the
computer using toggles, and have the computer encode the choices in a tableau.
An assessment then constitutes merely comparing against a reference tableau ---
something computers are notoriously good at.

If instead, the student is asked to write an essay in a natural language,
today's computers can assist with little more than dictionaries, thesauri,
grammar, and mark up. Although this provides for some rigor, vagueness and
ambiguity flourishes in natural languages --- they are at best, somewhat
rigorous.  The extent of this ``somewhat'' is the subject matter of much
research in natural language processing and automated essay
assessment\cite{valenti-et-al-2003}. Natural languages however, are much more
expressive, and allow for much richer assessment\cite{conole-warburton-2005}.

Beyond where computers can help, it is indeed the question of how rigorous one
can expect the students to be. In some disciplines, such as Computer Science,
high rigor can often be expected in certain types of assignments, such as
programming assignments. We explore this in the following section.

% the goal of higher education?

% validity vs. reliability.

% force students to see all feedback?

% Computer-aided assessment


% https://en.wikipedia.org/wiki/Educational_assessment
% https://en.wikipedia.org/wiki/Educational_technology
% https://en.wikipedia.org/wiki/Philosophy_of_education
% https://en.wikipedia.org/wiki/Instructional_theory
% https://en.wikipedia.org/wiki/Educational_theory

% https://en.wikipedia.org/wiki/Computer_aided_assessment
% https://en.wikipedia.org/wiki/E-assessment
% https://en.wikipedia.org/wiki/QTI

% http://www.imsglobal.org/question/index.html

% https://www.brookes.ac.uk/services/ocsld/resources/briefing_papers/p_p_assessment.pdf

% http://www.maasfep.org/spring13/Assessment%26ClassroomLearning_Black%26William.pdf

% http://files.eric.ed.gov/fulltext/ED214952.pdf

%http://www.google.dk/books?hl=en&lr=&id=ZQpUAQAAQBAJ&oi=fnd&pg=PP1&dq=Assessing+student+learning+in+higher+education&ots=iyEc1Evj2D&sig=ZlsuHVag2xcoTDJ2infrzGBbZoE&redir_esc=y#v=onepage&q=Assessing%20student%20learning%20in%20higher%20education&f=false

% https://en.wikipedia.org/wiki/Standards_for_Educational_and_Psychological_Testing

% http://www.ed.ac.uk/schools-departments/institute-academic-development/learning-teaching/staff/advice/researching/publications/experience-of-learning

% assignment text
% tool for testing a submission, providing some automated feedback
% tool for transforming the automated feedback to formal summative and formative feedback.
% summative comments are compoundable.

% per student attempt per exercise we save, their submission, and our summative
% and formative feedback for it.

% the summative feedback, if any, per assignment is defined when the assignment
% is defined, also a flag is given whether the summative feedback should be
% visible to the student or not.

% The feedback kinds is typically something like a scale, or multiple scales.

% ## To define an assignment:
% 1. Provide an assignment text.
% 2. Specify who may submit the assignment -- see group of assignments.
% 3. Specify the sequence of summative feedback kinds to be yielded for every
%    attempt to solve the assignment.
% 4. Specify a means of compounding the summative feedback of all attempts into
%    one holistic summative feedback on an assignment for a submitter.
% 5. Specify which of the summative feedbacks should be visible to the submitter.
% 6. Specify a test tool, providing some raw feedback for an attempt.
% 7. Specify a feedback transformation tool, yielding the said summative
%    and formative feedbacks.
% 8. Specify whether groups are supported.

% The summative feedback is compoundable. The formative feedback is not.

% The summative feedback may be compounded into a progress report - the
% progress report may or may not be made visible to the submitters.

% In all assignments, for all submitters, one type of summative feedback is
% visible to all submitters -- whether or not they have submitted.

% ## A group of assignments is:
% 1. A sequence of assignments.
% 2. A specification of the group of people that may submit the attempts to
%    solve the assignment.

% ## A progress report may be generated for a group of assignments.
% 1. A progress report is individual per pupil. (how about groups?)
% 2. Compound all the holistic summative feedbacks for all assignments into one
%    holistic summative feedback for the group of assignments for the pupil.

% http://psycnet.apa.org/psycinfo/1963-07136-000

% http://www.youtube.com/watch?v=jTH3ob1IRFo

% There is a range of summative data ready for collection when using
% computer-aided assessment. This may be useful in assessment in general.

% Use of computers blurs the definitions of summative and formative feedback
% further. \cite[262--265]{bloom1971}.


% vim: set spell:

\section{Courses}

A course is a unit of education imparted in a series of learning activities. A
student is someone who is enrolled for a course for the purposes of learning.
A teacher is someone who is enrolled for a course for the purposes of teaching.
It is the teachers that impart knowledge and skills onto the students. Other
enrollee roles are discussed in subsequent sections.

The student performance in a course is typically summatively assessed, at
least, on a pass/fail/neither basis. A student passes a course, if the student
has shown to possess some predefined knowledge or skills by the end of the
course, and fails otherwise.  A student may neither pass nor fail in various
extraordinary cases, such as the student dropping out of the course before a
final assessment.

Formative assessments typically happen throughout the course to facilitate
student learning, and sometimes at the end, to facilitate future learning.

In a possible subsequent course evaluation, students assess how well the
teachers performed in teaching. We are not concerned with this part of a course
timeline.


% vim: set spell:

\section{Assessment in Computer Science}


% vim: set spell:

\section{Roles in Educational Assessment}

In the previous sections we introduced the concepts of a student and teacher,
enrolled for a course. We also used notions like ``safe and fair environment''
and ``safe and secure communication''. In this section we expand on the roles
of course enrollees, and formalise the notions above.

\subsection{Teachers}

Teaching is the expediting of learning. Students learn on their own, but
teachers facilitate learning\cite{skinner-1965}. The means of facilitation
however, vary greatly throughout the discipline\cite{ramsden-1992,kember-1997}.
Most would unite the role of teaching with information delivery and assessment.

A non-empty set of teachers is always held responsible for a course. They are
responsible for ensuring that the students acquire the knowledge or skills
defined for the course. Other teachers may be involved in the course, aiding
the course responsible teachers.

In hope of students' learning, the teachers devise means of delivering
informative content, and assessing in how far students have acquired the said
knowledge or skills. Various techniques in both information delivery and
assessment are used to facilitate and encourage learning. We are not so
concerned with information delivery, as with assessment. Never-the-less,
assignments still have to be delivered to students.

Since teachers facilitate student learning and also their final assessment,
teachers exert great authority over students. Teachers decide whether a student
passes or fails a course, what grade they get, and how hard it is for them to
get it.

\subsubsection{Teaching assistants}

Teaching assistants assist in teaching responsibilities. They are teaching
subordinates of teachers. They exert some authority over students, but are
often limited in their authority when it comes to important summative
assessments. The result is that teaching assistants perform much of the
formative assessment, and provide guiding remarks either for the purposes of
feedback or to ease important summative assessments for the teachers.

Teaching assistants come about as a scaling mechanism. Once the number of
students enrolled for a course exceeds certain numbers, certain means of
information delivery and assessment are simply infeasible for the teachers
responsible. Instead of hiring more teachers, the strategy is often to rely on
some methods of information delivery and assessment that work in large numbers,
and rely on teaching assistants for the rest. Teaching assistants are therefore
cheaper, less qualified staff that assist in teaching responsibilities.

\subsection{Learning}

Learning is the gaining of knowledge or skills. Individuals engage in learning
in hope of being somehow enlightened or trained for solving particular kinds of
problems. It is a qualitative change of an individual's view of the
world\cite{ramsden-1992}.

Those engaged in the activity of learning for the purposes of obtaining a
degree are called students.

Learning requires a motivation to learn. Students are motivated by personal development, future employment opportunities, etc. In this context however, students may sometimes fail to see 

\subsection{Censure}

Censure is a process of quality assurance of assessment. A censor's
participation in an assessment in an assessment varies from mere observation to
avid participation. A censor therefore may need access to the individual
elements of a course or of an assessment in such a way that the work of
participants of a course is personally identifiable.

\subsection{General Public}

The general public includes those who are ultimately interested in the outcomes
of education and the quality of assessment therein. This includes both
perspective students, future employers, the politically concious, etc.

The general public may be interested in open access to the elements of
education and assessment for the purposes of assessing the quality of
education. The intent may be to see if the education lives up to social
expectations, demands of the labour market, political promises, etc.

Privacy and anonymity is a matter of grave public concern. If open access is
given, it should only identify those who may reasonably be held responsible for
an eventual lack of quality in education. Also, issues of copyright have to be
taken into account.

As the general public would assess education, and not students, students should
not be personally identifiable by the general public. In how far teachers and
teaching assistants may reasonably be held responsible by the general public
for the quality of education may be a matter of university policy (as their
employer). It is important that the assessed are sufficiently informed of their
assessment and have conceded to it.

As students typically own the content they produce, individual student work or
commentary should not be made available to the general public. In how far
teachers and teaching assistants own the content they produce may again be a
matter of university policy (as their employer). It is important that the
owners of content have command over its reproduction.

