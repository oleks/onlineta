% vim: set spell:

\chapter{Assessment Engine}

\label{section:assessment-engine}

\begin{quotation}

\footnotesize\sffamily\itshape

\begin{flushright}

/* War is peace. Verbosity is silence. MS\_VERBOSE is deprecated. */

\smallbreak

\upshape

--- DAVID HOWELLS, Linux Source Code (2012)

\end{flushright}

\end{quotation}

When a student makes a submission, it needs to be assessed. The
assessment is specified by the teaching staff, and may consist of a range of
static and dynamic analyses, as well as a feedback processing step.

In general, we refer to these steps as the processes of an assessment. The
processes of an assessment, each run their own sandbox and communicate with
other processes using standard in/out, and perhaps an in-memory file system. 

The combination of these sandboxed processes forms an assessment engine, taking
the student submission in at one end, and producing ``student-friendly''
feedback at the other.

The general approach to designing these processes is intended to be to monitor
a reference solution: which files, libraries does it need, how much memory and
CPU time does it need, etc. This data is then used to create a conservative
sandbox for running the processes of an assessment.

\textbf{Remark}: Due to time constraints, I have not been able to complete this
chapter. This chapter should be read as notes, rather than a fine-grained text.

\section{Monitoring}

Control groups, as discussed in \referToSection{sandboxing:control-groups}, can
be used to measure the memory, and CPU time of processes. \cite{man-1-time}
user space utility can be used to measure wall-clock time.

A bit more cumbersome is figuring out what files are needed in the root file
system for a process to complete successfully. Some files are necessitated by
the assignment itself. Sometimes, it will also suffice to execute statically
linked programs. All this is easy enough.

The complication comes for executables that make use of other programs for
their execution (e.g. shell scripts) dynamically linked executables. We can
figure out the files that are opened throughout the execution of a program
using \texttt{strace(1)}.

For a dynamically-linked ELF executable, we can inspect it using
\texttt{mreadelf(1)}. Here, we are looking for interpreter it is requesting,
which will dynamically link the relevant library. Next, we look for the shared
libraries that it requests. Including the interpreter and the related shared
libraries in the root file system is sufficient to run a dynamically linked
executable within that root file system.

\section{Sharing a file system between a host and a container}.

One way to share data between the host and a Linux container is using a shared
in-memory file system. We sketch the algorithm for this using the techniques we
discussed in the previous chapter.

\begin{enumerate}

\item Mount a temporary in-memory file system under a target \texttt{./output},
e.g. of maximum size 1 megabyte.

\item Enter a new mount namespace.

\item Make sure that original root marks all mount points as private.

\item Mount the new root as a readonly in-memory slave under a target
\texttt{./readonly}, e.g. using a \texttt{squashfs} file system type for extra
``read-onlyness''.

\item Bind mount \texttt{./output} under \texttt{./readonly/output}.

\item Pivot root to the new root.

\item Unmount all but the new root and the output directory.

\item Drop privileges.

\item Run the contained program. (Exit the mount namespace.)

\item Clean up, and save the data in \texttt{./output}, if necessary.

\end{enumerate}

This way, the host provides a safe scratchspace for the container. Both the
host and the container have read and write access to this scratchspace.  Yet,
the container needs no access to any underlying block device. If the data a
container produces should be retained, we do this in the last step.

% vim: set spell:

\section{Other Aspects}

\subsection{Pivot root}

One particularly useful application of mount namespaces (see also
\referToSection{namespaces:mnt}) is pivoting the file system root to some other
point in the file system using \cite{man-2-pivot-root}. Pivoting the root in a
container does not affect the host, or other containers. At the same time,
pivoting the root moves all the dependencies on the old root, to the new root,
within the container.

This allows us to subsequently unmount the old root, provided that the old root
is not busy. The old root is busy if there are mounts to targets under the old
root, or the running process depends on the old root.  By closing all open
files, switching to a process originating from under the new root, and
unmounting all mounts under the old root, we can get to unmount the old root.

\cite{man-2-pivot-root} hides the original root file system in a matter similar
to \cite{man-2-chroot}, but makes reestablishing the old root slightly more
cumbersome, since the old root has to be properly remounted first. By mounting
the new root in a \texttt{tmpfs}, or perhaps even \texttt{squashfs}, we can
subsequently deny access to all block devices within the container (see also
\referToSection{control-groups:devices}). This makes remounting the old root
all the more cumbersome.

