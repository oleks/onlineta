% vim: set spell:

\chapter{Assessment Engine}

\begin{quotation}

\footnotesize\sffamily\itshape

\begin{flushright}

/* War is peace. Verbosity is silence. MS\_VERBOSE is deprecated. */

\smallbreak

\upshape

--- DAVID HOWELLS, Linux Source Code (2012)

\end{flushright}

\end{quotation}

When a student makes a submission, it needs to be assessed. The
assessment is specified by the teaching staff, and may consist of a range of
static and dynamic analyses, as well as a feedback processing step.

In general, we refer to these steps as the processes of an assessment. The
processes of an assessment, each run their own sandbox and communicate with
other processes using standard in/out, and perhaps an in-memory file system. 

The combination of these sandboxed processes forms an assessment engine, taking
the student submission in at one end, and producing ``student-friendly''
feedback at the other.

The general approach to designing these processes is intended to be to monitor
a reference solution: which files, libraries does it need, how much memory and
CPU time does it need, etc. This data is then used to create a conservative
sandbox for running the processes of an assessment.

\section{Monitoring}

Control groups, as discussed in \referToSection{sandboxing:control-groups}, can
be used to measure the memory, and CPU time of processes. \cite{man-1-time}
user space utility can be used to measure wall-clock time.

A bit more cumbersome is figuring out what files are needed in the root file
system for a process to complete successfully.
