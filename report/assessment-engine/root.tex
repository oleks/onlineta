% vim: set spell:

\chapter{Assessment Engine}

\label{section:assessment-engine}

\begin{quotation}

\footnotesize\sffamily\itshape

\begin{flushright}

This is the Unix philosophy: Write programs that do one thing and do it well.
Write programs to work together. Write programs to handle text streams, because
that is a universal interface.

\smallbreak

\upshape

--- Doug Mcllroy (1994)

\end{flushright}

\end{quotation}

\begin{quotation}

\footnotesize\sffamily\itshape

\begin{flushright}

Some of the more obscure uses of \texttt{pivot\_root()} may quickly lead to insanity.

\smallbreak

\upshape

--- \texttt{pivot\_root(2)} (2012)

\end{flushright}

\end{quotation}

When a student makes a submission, it needs to be assessed. The
assessment is specified by the teaching staff, and may consist of a range of
static and dynamic analyses, as well as a feedback processing step.

In general, we refer to these steps as the \emph{processes} of an assessment.
The processes of an assessment, each run their own sandbox and communicate with
other processes using standard in/out, and perhaps an in-memory file system.
The combination of these sandboxed processes forms an \emph{assessment engine},
taking the student submission in at one end, and producing ``student-friendly''
feedback at the other.

The general approach to designing these processes is intended to be to monitor
a reference solution: which files, libraries does it need, how much memory and
CPU time does it need, etc. This data is then used to create a conservative
sandbox for running the processes of an assessment.

% vim: set spell:

\section{Sandboxes and Chaining}

\label{section:assessment-engine:sandboxes-and-chaining}

We follow the Unix philosophy\cite{salus-1994} in our sandboxing approach, i.e.
we write programs that do one thing and (eventually) do it well. More
sophisticated sandboxes are achieved by chaining together simple sandboxing
programs. A simple sandboxing program puts itself into a simple sandbox and
executes (or forks and executes) the next program on its argument list.

This keeps the individual sandboxing quirks to the individual programs, and
allows for an easy composition of sandboxes into sophisticated sandboxed
environments. Unlike LXC, and other tools we've seen, it also gives us full
control of the order in which a sandbox is constructed.

Some of our simple sandboxing programs are summarised below:

\begin{description}[\setleftmargin{0.2in}\breaklabel\setlabelstyle{\tt}]

\item [cgroups] Attach the process to a set of cgroups. The cgroup
\texttt{tasks} files (where to append the pid) are passed as options to the
program.

\item [mnt] Enter a new mount namespace.

\item [user] Enter a new user namespace. The current user is mapped to uid 0
within the container.

\item [pumin] Enter a new (p)id, (u)ts, (m)ount, (i)pc, and (n)et namespace.

\item [sh-rootfs-bind] Bind-mount a root file system. By default, the subfolder
\texttt{rootfs} is assumed to contain the new root file system, and
\texttt{rootfs-target} is assumed to be the target for the mount.

\item [sh-rootfs-squashfs] Similar to \texttt{sh-rootfs-bind}, but mounts a
root file system as a SquashFS. By default, assumes that the SquashFS file is
located at \texttt{squashfs}.

\item [sh-iofs] Mounts a tmpfs for use for file-system-based input to and
output from a container. By default, assumes that the input should be copied
from \texttt{input}, and the output should be copied to \texttt{output} (after
deleting all other files). Also, assumes that the tmpfs target is
\texttt{tmpfs-target}, and under the new root,
\texttt{rootfs-target/home/student}. Should only be called after the new rootfs
has been mounted.

\item [pivot-root] Pivot the root file system. By default, it is assumed that
the new root is under \texttt{root}, and the old root can be placed under
\texttt{rootfs/.oldroot}.

\item [umount-oldroot] Unmount the old root file system. By default, assumed to
be under \texttt{/.oldroot}.

\item [setuid] Set the user id. By default, to 1000.

\item [rlimits] Set resource limits. By default, set core file size to 0, the
maximum number of processes to 20, and the maximum CPU time to 1 second.

\end{description}

We can chain these together to form sophisticated sandboxes. For instance, let
\texttt{rootfs} be a subfolder in the current working directory, containing an
executable \texttt{rootfs/home/student/forkbomb} (more on the necessity of
\texttt{sudo} later):

\begin{lstlisting}
$ sudo ./pumin ./sh-rootfs-bind ./pivot-root /home/student/forkbomb
\end{lstlisting}

The keen reader might notice that e.g. \texttt{setuid} and \texttt{pivot-root}
already have Linux user space equivalents. We reimplement these to fit our
chaining approach. We leave it to future work to not reimplement existing
simple sandboxing programs, or patch these programs to suit our needs.

\textbf{Remark}: Under the current implementation, the default is typically the
only option for the above programs.


% vim: set spell:

\section{Monitoring and Limits}

In general, we would like to monitor the execution of our programs. For
instance, we may monitor a reference solution to an assignment, so as to arrive
at sound limits on submissions for that assignment. Monitoring a submission may
also give some useful information wrt. feedback.

We introduce the program \texttt{monitor} performing monitoring and imposing
limits. The program uses a sophisticated sandbox (see above) to impose many
general limits, and allows optionally for a limit wall-clock time, CPU time,
and memory.  The program also has the option of using either a bind-mounted or
a SquashFS root file system.

\begin{lstlisting}
$ sudo ./monitor -h
Usage: ./monitor
  [-w <timeout>]
  [-m <memlimit-in-bytes>]
  [-t <cpu-time-limit-in-seconds>]
  [-s] (use squashfs for the rootfs)
  <executable-on-rootfs>
\end{lstlisting}

For instance, let's try monitoring a forkbomb that prints a \texttt{.} for
every process it gets to spawn:

\begin{lstlisting}
$ sudo ./monitor /home/student/forkbomb
...............Wall-clock time (seconds): 0.04
Max memory use (bytes): 524288
Total CPU time (nanoseconds): 3493968
Total CPU time in user mode (USER_HZ): 0
Total CPU time in kernel mode (USER_HZ): 0
\end{lstlisting}

These results come from putting the program into a memory, and cpuacct
namespace, as well as making the \cite{man-1-time} Linux user space program
part of our program chain.

You can experiment with this on the referenced virtual machine, or browse the
code under the attached \texttt{./src/jail/monitor}. The results of applying
various optional limits are also discussed in \referToSection{discussion}. In
the following sections we discuss some of the more interesting aspects of our
sandboxes and monitoring.


% vim: set spell:

\section{Pivot Root}

One particularly useful application of mount namespaces (see also
\referToSection{namespaces:mnt}) is pivoting the file system root to some other
point in the file system using \cite{man-2-pivot-root}. Pivoting the root in a
container does not affect the host, or other containers. At the same time,
pivoting the root moves all the dependencies on the old root, to the new root,
within the container.

This allows us to subsequently unmount the old root, provided that the old root
is not busy. So \cite{man-2-pivot-root} can be used to hide the original root
file system in a matter similar to \cite{man-2-chroot}, but makes
reestablishing the old root slightly more cumbersome, since the old root has to
be properly remounted first. By mounting the new root in a \texttt{tmpfs}, or
perhaps even \texttt{squashfs}, we can subsequently deny access to all block
devices within the container (see also
\referToSection{control-groups:devices}). This makes remounting the old root
all the more cumbersome.

\subsection{Unmount vs. Detach Old Root}

The old root is busy if there are mounts to targets under the old root, or the
running process depends on the old root.  By closing all open files, switching
to a process originating from under the new root, and unmounting all mounts
under the old root, we can get to unmount the old root.

However, unmount requires root privileges on the host. The approach above
therefore requires that we pivot the root while still having root privileges on
the host. We have not solved the problem of entering a user namespace after
pivoting the root to some bare bones root file system. This is bad because it
essentially renders user namespaces inapplicable in our case.

The other approach is to detach everything under the old root, meaning that it
will be unmounted when the old root is no longer busy. We can then take a leap
of faith that the old root will indeed be detached when we switch to a process
under the new root (rendering the old root not busy). We have not had time to
implement this.


% vim: set spell:

\section{Root File System}

\label{section:assessment-engine:root-file-system}

We prefer a bare bones root file system providing little more than exactly what
is needed to execute our programs. It can be a bit challenging to assemble such
root file systems.

We assemble a root file system by copying everything that we need under e.g. a
\texttt{rootfs} subdirectory. For instance, in the attached source code, and
for the referenced virtual machine (see also \referToSection{discussion}), we
have assembled:

\begin{description}[\setleftmargin{0.2in}\breaklabel\setlabelstyle{\tt}]

\item [rootfs/lib64/ld-linux-x86-64.so.2] A dynamic linker for 64-bit
executables on e.g. Ubuntu.

\item [rootfs/lib/x86\_64-linux-gnu/libc.so.6] \texttt{glibc} for dynamic
linking on e.g. Ubuntu.

\item [rootfs/.oldroot] Directory for the old root.

\item [rootfs/home/student] Directory for I/O file system.

\end{description}

With this root file system, we can execute any standard dynamically-linked
64-bit ELF executable built on our system.

We can figure out which dynamic linker a dynamically-linked ELF executable
needs using \texttt{readelf(1)}\footnote{Look for ``Required interpreter''.}.
Similarly, we can figure out some of the shared libraries that the executable
needs\footnote{Look for ``Shared library''.}. However, once we've included the
necessary dynamic linker, it will report if a shared library is missing. This
way, we can assemble the necessary shared libraries using a fixed-point
iteration.

Our programs may also use shell scripts, or require other obscure elements of
the file system. To this end, we can use \texttt{strace(1)} (with the
\texttt{-f} option) on our program to figure out what sort of files it is
looking for and failing to find.

\subsection{SquashFS}

SquashFS is a compressed read-only file system for Linux\footnote{See also
\url{http://squashfs.sourceforge.net/}.}. In principle, it can serve as a file
system type for read-only root file systems for our containers.  Being
read-only by design, this may provide some extra confidence that the root file
system will remain read-only.

SquashFS does come with a couple quirks. A SquashFS is a file which can be
mounted as a loopback device. The Linux kernel puts a bound on the number of
loopback devices. This bound can be increased.  Also, being mounted as a
loopback device means that we can't blindly deny access to all devices.
Luckily, loopback devices (and virtual character devices) always have the major
device number 7\cite{devices.txt-b}.


% vim: set spell:

\section{I/O File System}

One way to share data between the host and a Linux container is using a shared
in-memory file system.

Once we've mounted a root file system (but before we pivot the root), we can
mount a \texttt{tmpfs} under the current (old) root, and bind mount it under
what will be the new root. Entering a new mount namespace, and pivoting the
root moves all dependencies on the old root to the new root, and so the bind
mound becomes a \texttt{tmpfs} mount inside the container, after the root is
pivoted.

Of course, nothing happens to the mount outside the mount namespace, and the
seemingly two \texttt{tmpfs} mounts indeed point to the same temporary file
system. This way, we can share a file system between a host and container
without giving access to a block device.

(See also the attached \texttt{./src/jail/sh-iofs} and
\texttt{./src/jail/monitor}.)

