% vim: set spell:

\section{Monitoring and Limits}

In general, we would like to monitor the execution of our programs. For
instance, we may monitor a reference solution to an assignment, so as to arrive
at sound limits on submissions for that assignment. Monitoring a submission may
also give some useful information wrt. feedback.

We introduce the program \texttt{monitor} performing monitoring and imposing
limits. The program uses a sophisticated sandbox (see above) to impose many
general limits, and allows optionally for a limit wall-clock time, CPU time,
and memory.  The program also has the option of using either a bind-mounted or
a SquashFS root file system.

\begin{lstlisting}
$ sudo ./monitor -h
Usage: ./monitor
  [-w <timeout>]
  [-m <memlimit-in-bytes>]
  [-t <cpu-time-limit-in-seconds>]
  [-s] (use squashfs for the rootfs)
  <executable-on-rootfs>
\end{lstlisting}

For instance, let's try monitoring a forkbomb that prints a \texttt{.} for
every process it gets to spawn:

\begin{lstlisting}
$ sudo ./monitor /home/student/forkbomb
...............Wall-clock time (seconds): 0.04
Max memory use (bytes): 524288
Total CPU time (nanoseconds): 3493968
Total CPU time in user mode (USER_HZ): 0
Total CPU time in kernel mode (USER_HZ): 0
\end{lstlisting}

These results come from putting the program into a memory, and cpuacct
namespace, as well as making the \cite{man-1-time} Linux user space program
part of our program chain.

You can experiment with this on the referenced virtual machine, or browse the
code under the attached \texttt{./src/jail/monitor}. The results of applying
various optional limits are also discussed in \referToSection{discussion}. In
the following sections we discuss some of the more interesting aspects of our
sandboxes and monitoring.
