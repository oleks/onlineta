% vim: set spell:

\section{Root File System}

We prefer a bare bones root file system providing little more than exactly what
is needed to execute our programs. It can be a bit challenging to assemble such
root file systems.

We assemble a root file system by copying everything that we need under e.g. a
\texttt{rootfs} subdirectory. For instance, in the attached source code, and
for the referenced virtual machine, we have assembled:

\begin{description}[\setleftmargin{0.2in}\breaklabel\setlabelstyle{\tt}]

\item [rootfs/lib64/ld-linux-x86-64.so.2] A dynamic linker for 64-bit executables on Ubuntu.

\item [rootfs/lib/x86\_64-linux-gnu/libc.so.6] \texttt{glibc} for dynamic linking.

\item [rootfs/.oldroot] Directory for the old root.

\item [rootfs/home/student] Directory for I/O file system.

\end{description}

With this root file system, we can execute any standard dynamically-linked
64-bit ELF executable built on our system.

We can figure out which dynamic linker a dynamically-linked ELF executable
needs using \texttt{readelf(1)}\footnote{Look for ``Required interpreter''.}.
Similarly, we can figure out some of the shared libraries that the executable
needs\footnote{Look for ``Shared library''.}. However, once we've included the
necessary dynamic linker, it will report if a shared library is missing. This
way, we can assemble the necessary shared libraries using a fixed-point
iteration.

Our programs may also use shell scripts, or require other obscure elements of
the file system. To this end, we can use \texttt{strace(1)} (with the
\texttt{-f} option) on our program to figure out what sort of files it is
looking for and failing to find.

\subsection{SquashFS}

SquashFS is a compressed read-only file system for Linux\footnote{See also
\url{http://squashfs.sourceforge.net/}.}. In principle, it can serve as a file
system type for read-only root file systems for our containers.  Being
read-only by design, this may provide some extra confidence that the root file
system will remain read-only.

SquashFS does come with a couple quirks. A SquashFS is a file which can be
mounted as a loopback device. The Linux kernel puts a bound on the number of
loopback devices. This bound can be increased.  Also, being mounted as a
loopback device means that we can't blindly deny access to all devices.
Luckily, loopback devices (and virtual character devices) always have the major
device number 7\cite{devices.txt-b}.
