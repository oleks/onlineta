% vim: set spell:

\section{Sandboxes and Chaining}

We follow the Unix philosophy\cite{salus-1994} in our sandboxing approach, i.e.
we write programs that do one thing and (eventually) do it well. More
sophisticated sandboxes are achieved by chaining together simple sandboxing
programs. A simple sandboxing program puts itself into a simple sandbox and
executes (or forks and executes) the next program on its argument list.

This keeps the individual sandboxing quirks to the individual programs, and
allows for an easy composition of sandboxes into sophisticated sandboxed
environments. Unlike LXC, and other tools we've seen, it also gives us full
control of the order in which a sandbox is constructed.

Some of our simple sandboxing programs are summarised below:

\begin{description}[\setleftmargin{0.2in}\breaklabel\setlabelstyle{\tt}]

\item [cgroups] Attach the process to a set of cgroups. The cgroup
\texttt{tasks} files (where to append the pid) are passed as options to the
program.

\item [mnt] Enter a new mount namespace.

\item [user] Enter a new user namespace. The current user is mapped to uid 0
within the container.

\item [pumin] Enter a new (p)id, (u)ts, (m)ount, (i)pc, and (n)et namespace.

\item [sh-rootfs-bind] Bind-mount a root file system. By default, the subfolder
\texttt{rootfs} is assumed to contain the new root file system, and
\texttt{rootfs-target} is assumed to be the target for the mount.

\item [sh-rootfs-squashfs] Similar to \texttt{sh-rootfs-bind}, but mounts a
root file system as a SquashFS. By default, assumes that the SquashFS file is
located at \texttt{squashfs}.

\item [sh-iofs] Mounts a tmpfs for use for file-system-based input to and
output from a container. By default, assumes that the input should be copied
from \texttt{input}, and the output should be copied to \texttt{output} (after
deleting all other files). Also, assumes that the tmpfs target is
\texttt{tmpfs-target}, and under the new root,
\texttt{rootfs-target/home/student}. Should only be called after the new rootfs
has been mounted.

\item [pivot-root] Pivot the root file system. By default, it is assumed that
the new root is under \texttt{root}, and the old root can be placed under
\texttt{rootfs/.oldroot}.

\item [umount-oldroot] Unmount the old root file system. By default, assumed to
be under \texttt{/.oldroot}.

\item [setuid] Set the user id. By default, to 1000.

\item [rlimits] Set resource limits. By default, set core file size to 0, the
maximum number of processes to 20, and the maximum CPU time to 1 second.

\end{description}

We can chain these together to form sophisticated sandboxes. For instance, let
\texttt{rootfs} be a subfolder in the current working directory, containing an
executable \texttt{rootfs/home/student/program} (more on the necessity of
\texttt{sudo} later):

\begin{lstlisting}
$ sudo ./minpuf ./mount-rootfs ./pivot-root /home/student/program
\end{lstlisting}

The keen reader might notice that e.g. \texttt{setuid} and \texttt{pivot-root}
already have Linux user space equivalents. We reimplement these to fit our
chaining approach. We leave it to future work to not reimplement existing
simple sandboxing programs, or patch these programs to suit our needs.

\textbf{Remark}: Under the current implementation, the default is typically the
only option for the above programs.
