% vim: set spell:

\section{Pivot Root}

One particularly useful application of mount namespaces (see also
\referToSection{namespaces:mnt}) is pivoting the file system root to some other
point in the file system using \cite{man-2-pivot-root}. Pivoting the root in a
container does not affect the host, or other containers. At the same time,
pivoting the root moves all the dependencies on the old root, to the new root,
within the container.

This allows us to subsequently unmount the old root, provided that the old root
is not busy. So \cite{man-2-pivot-root} can be used to hide the original root
file system in a matter similar to \cite{man-2-chroot}, but makes
reestablishing the old root slightly more cumbersome, since the old root has to
be properly remounted first. By mounting the new root in a \texttt{tmpfs}, or
perhaps even \texttt{squashfs}, we can subsequently deny access to all block
devices within the container (see also
\referToSection{control-groups:devices}). This makes remounting the old root
all the more cumbersome.

\subsection{Unmount vs. Detach Old Root}

The old root is busy if there are mounts to targets under the old root, or the
running process depends on the old root.  By closing all open files, switching
to a process originating from under the new root, and unmounting all mounts
under the old root, we can get to unmount the old root.

However, unmount requires root privileges on the host. The approach above
therefore requires that we pivot the root while still having root privileges on
the host. We have not solved the problem of entering a user namespace after
pivoting the root to some bare bones root file system. This is bad because it
essentially renders user namespaces inapplicable in our case.

The other approach is to detach everything under the old root, meaning that it
will be unmounted when the old root is no longer busy. We can then take a leap
of faith that the old root will indeed be detached when we switch to a process
under the new root (rendering the old root not busy). We have not had time to
implement this.
