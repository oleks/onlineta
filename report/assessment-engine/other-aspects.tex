% vim: set spell:

\section{Other Aspects}

\subsection{Pivot root}

One particularly useful application of mount namespaces (see also
\referToSection{namespaces:mnt}) is pivoting the file system root to some other
point in the file system using \cite{man-2-pivot-root}. Pivoting the root in a
container does not affect the host, or other containers. At the same time,
pivoting the root moves all the dependencies on the old root, to the new root,
within the container.

This allows us to subsequently unmount the old root, provided that the old root
is not busy. The old root is busy if there are mounts to targets under the old
root, or the running process depends on the old root.  By closing all open
files, switching to a process originating from under the new root, and
unmounting all mounts under the old root, we can get to unmount the old root.

\cite{man-2-pivot-root} hides the original root file system in a matter similar
to \cite{man-2-chroot}, but makes reestablishing the old root slightly more
cumbersome, since the old root has to be properly remounted first. By mounting
the new root in a \texttt{tmpfs}, or perhaps even \texttt{squashfs}, we can
subsequently deny access to all block devices within the container (see also
\referToSection{control-groups:devices}). This makes remounting the old root
all the more cumbersome.
