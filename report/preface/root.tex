% vim: set spell:

\chapter{Preface}

The intent of this work is to explore the use of computers in assisting in the
assessment of practical work in computer programming courses.

\section{Audience}

This work is primarily aimed at computer programming educators. At least two
other classes of readers should also feel at home in distinct parts of the
report: The first few chapters deal with assessment in general, and how
computers can assist in assessment. The remainder of the report deals with
sandboxing capabilities in the Linux kernel, and using cryptography and Git to
provide for a safe and secure course content management system.

The reader should be familiar with the basics of computer programming and a
Unix-like operating system. Ideally, the reader should be familiar with the
basics of C, Posix shell scripting, and Git.

\section{References}

We have preferred material published in relevant scientific journals, yet
accessible to the general public. This can be complicated when dealing with
such a community driven effort as Linux. Of course, Linux source code is
meticulously reviewed, but in a matter which can at times bes regarded as
inferior, and at times, superior to scientific review.

To ensure long-lasting access to publicly available, web-based resources, we
have additionally archived most of them using
WebCite\textsuperscript{\textregistered}.

\section{About the Author}

This report is written in third person out of aesthetic considerations, with
the exception of this section.

When referring to ``practice'' wrt. teaching, I refer to my own practice as a
teaching assistant in various courses at DIKU, or the perceivable practice of
teachers and teaching assistants around me. I have worked as a TA for DIKU
under the following courses (for some, multiple times): Introduction to
Programming, Object-oriented Programming and Design, Introduction to
Algorithms, Compilers, Operating-systems and Multiprogramming, Statistical
Methods for Machine Learning, and Advanced Programming.

I have also been extensively involved with a recent redefinition and redesign
of the Object-oriented Programming and Design course at DIKU. I have been a
student at DIKU since 2009.
