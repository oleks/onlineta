% vim: set spell:

\chapter{Assessment Pipeline}

When a student makes a submission, it needs to be assessed. The process of
assessment is specified by the teaching staff, and may consist of a series of
tasks --- an assessment pipeline.

Concerned with assessment in Computer Science, there are two principal ways to
assess a computer program --- statically, i.e. analyze the program text, and
dynamically, i.e. execute the program. The static and dynamic analyses may
furthermore be performed within various wrappers, e.g. monitors or
interpreters.

To execute a program, some input data may need to be generated. Useful programs
will typically also produce some output data. Perhaps the student program is
even interactive. This requires an interactive co-executing process.

The output from static analyses, the execution of the program, the monitoring
programs, is what we'll collectively call ``raw feedback''. This raw feedback
may be too obscure or too detailed for a student to benefit from it. It may
therefore be a good idea to post-process the raw feedback into a more
fine-grained feedback for the student.
