% vim: set spell:

\section{Higher Education Organisation}

In the following sections we cover some other basic organisational aspects of
higher education. These are of relatively little interest to us, but do provide
some context for the assessments we wish to assist in.

\subsection{Course}

A course is a unit of education imparted in a series of learning activities.
The student performance in a course is always assessed, at least, on a
pass/fail/neither basis. A student passes a course, if the student has shown to
possess some predefined knowledge or skills by the end of the course, and fails
otherwise.  A student may neither pass nor fail in various extraordinary cases,
such as the student dropping out of the course before a final assessment.

\subsection{Programme}

A programme in higher education is a series of courses that a student has to
pass. At the end of a programme the student is awarded a degree: a statement of
accomplishment in the courses passed.

\subsection{Institution}

A higher education institution is a legal entity that offers higher education
programmes. Typically this is a university or a college. Students are enrolled
for programmes at particular higher education institutions. Teaching staff are
employed by the higher education institution and facilitate the education.
Other roles in educational assessment have less definite relationships to
institutions.
