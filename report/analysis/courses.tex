% vim: set spell:

\section{Courses}

A course is a unit of education imparted in a series of learning activities. A
student is someone who is enrolled for a course for the purposes of learning.
A teacher is someone who is enrolled for a course for the purposes of teaching.
Other roles are discussed in subsequent sections.

A course has some predefined knowledge or skills that the students should
acquire by the end of the course. It is the teachers that impart this knowledge
or skills onto the students, in hope that they retain it.

The student performance is typically summatively assessed at the end of the
course. At least, on a pass/fail/neither basis. A student passes a course, if
the student has shown to have acquired the said knowledge or skills, and fails
otherwise.  A student may neither pass nor fail in various extraordinary cases,
such as the student dropping out of the course before a final assessment.

Formative assessments of students are typically conducted throughout the course
to facilitate and courage student learning, and sometimes also at the end, to
facilitate and encourage future learning.

In a possible subsequent course evaluation, students may assess how well the
teachers performed in their teaching. We will not be concerned with this part
of a course.
