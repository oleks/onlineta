% vim: set spell:

\section{Courses}

A course is a unit of education imparted in a series of learning activities. A
student is someone who is enrolled for a course for the purposes of learning.
A teacher is someone who is enrolled for a course for the purposes of teaching.
It is the teachers that impart knowledge and skills onto the students. Other
enrollee roles are discussed in subsequent sections.

The student performance in a course is typically summatively assessed, at
least, on a pass/fail/neither basis. A student passes a course, if the student
has shown to possess some predefined knowledge or skills by the end of the
course, and fails otherwise.  A student may neither pass nor fail in various
extraordinary cases, such as the student dropping out of the course before a
final assessment.

Formative assessments typically happen throughout the course to facilitate
student learning, and sometimes at the end, to facilitate future learning.

In a possible subsequent course evaluation, students assess how well the
teachers performed in teaching. We are not concerned with this part of a course
timeline.
