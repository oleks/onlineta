% vim: set spell:

\section{Static Analyses}

We may wish for a digital submission to adhere to a particular format, but we
cannot rely on students to meet such requirements in general. One basic use of
computer-assisted assessment is checking whether a submission meets certain
formatting requirements, before any further assessment.

For a programming assignment, we may wish for a submission to consist of a
computer program and a report --- we'll use this as a running example.

It is fairly straight-forward to enforce the requirement that a digital
submission be a non-empty set of digital files, distinguished by their file
names. Similarly, we can require files with particular names to be submitted,
stating a particular file type.  For instance, two text files, named
\texttt{main.c} and \texttt{report.txt}. It is also fairly straight-forward to
enforce a limit on the sizes of the files in a submission.  This has some
security benefits. % TODO: name them later.

We may further enforce the requirement that particular files look like they are
written in particular languages. For instance, that the computer program is
written in C, and the report is written in English. We say ``look like''
because neither language in this example has a formal definition, and so cannot
be guaranteed to be recognisable by a computer, but both are in frequent use.

For a lot of programming languages --- we're in luck --- the job of a parser,
as part of e.g. a compiler or assembler, is indeed to recognise, whether a
sequence of bits can be interpreted as a statement in that
language\footnote{The question of whether a sequence of bytes can be
interpreted as e.g. a C program, is often reduced to whether or not it is
recognisable by a particular C compiler.}.

With natural languages, being less formal in general, we're in a bit less luck.
We can often analyse various features of a report probabilistically, with
fairly high confidence. Notably, such probabilistic techniques give only
limited corrective guidance. We conjecture that reports will require subsequent
assessment by the teaching staff much more often than computer programs.

Checking that a submission meets such requirements is usually a matter of
static analysis. C++ Templates aside\cite{veldhuizen-2003}, checking that a
submission meets programming language requirements should not result in the
execution of programs written by students in a Turing-complete language.

Putting a C program through a C compiler, we get an executable out. This
enables a subsequent dynamic analysis. For other languages, we may perhaps only
have a (lazy) interpreter, where the syntax of a part of a program is (only)
checked immediately before it is executed. This would require checking a valid
syntax requirement in lockstep with a dynamic analysis.

We will often also want to check that the program is readable, testable, and
maintainable by others, that it uses particular programming abstractions, and
uses them right. This superimplies that we want submissions to include the
source code of the programs. The above requirements can then be checked by
checking that the source code adhere to comprehensive, assignment-specific,
style guides.

This amounts to a range of static analyses that we may want to perform, some of
them in lockstep with dynamic analyses.
