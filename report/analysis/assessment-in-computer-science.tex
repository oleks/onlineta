% vim: set spell:

\section{Assessment in Computer Science}

Computer Science is the study of computable structures and processes. It is the
subject of study of some, and the interest of many. Computer Science
professionals are (among other things) expected to be eloquent in the theory
and practice of computer programming\cite{cs-curricula-2013}.

To this end, practical work is a popular basis for assessment in Computer
Science\cite{carter-et-al-2003}. Practical work is concerned with the
composition of programs to be executed by a computer, solving a particular
problem.

To be executable by computers, computer programs are often written in highly
rigorous languages, and so are amenable to assessment with the assistance of
computers. The assessment of computer programs is a wide area of research and
industry, known as software verification or quality assurance.

With the advent of modern computer technology, computer programs can also be
assumed to be structured with the assistance of computers. Indeed, assemblers,
compilers, linkers, interpreters, etc. have become ubiquitous. Students can be,
and often are, expected to acquire skills in using these tools on their own
account, or with some facilitation from their teachers.

\subsection{Programming Languages}

Assignments in Computer Science will often ask students to write computer
programs in one of a range of different programming languages. Programming
languages come and go, and no language has emerged as the predominant one for
teaching Computer Science. It is most useful, therefore, to facilitate
assessment of programs written in any conceivable programming language.

Modern programs always run within the context of an operating system. One way
to stay programming language agnostic, is to facilitate assessment at the
operating system level. This requires looking into how student programs, and
various analyses thereof, can be run in safe and fair environments, and how the
various elements of an assessment engine can communicate via the operating
system in a safe and secure way.

\subsection{Permissions and Restrictions}

Assignments in Computer Science will often also ask students to write programs
that operate within environments with particular permissions and restrictions.
We should facilitate such permissions and ensure that the restrictions are
adhered to. Both enable certain types of assessment. For instance, we may wish
to permit that students can write to a particular file, but restrict how many
I/O operations they may perform in total.

\subsection{A Reference Course}

A good reference course, is a course in operating systems programming in C.

C is ubiquitous on modern computer architectures, with the folklore that the
first thing you should write for your new processor is a C compiler. We
conjecture, that anything written for execution on a modern computer, with
sufficient effort, can also be written in C. Especially because we can write
assembly in C. C is also ubiquitous on modern operating systems. Perhaps the
most basic API that a modern operating system offers is a C API. If not, we can
again resort to writing assembly in C.

Operating systems courses typically also involve a wide range of necessary
permissions and restrictions. Especially if student programs are not written
for some idealised, virtual environment (e.g. Buenos), but are intended for
running in the context of a real operating system.

Such programs may run into faults, make obscure system calls, access devices,
perform I/O, not to mention run high on memory use and CPU time. We conjecture,
that if we can facilitate safe and fair environments for a course in operations
systems programming in C, we can facilitate safe and fair environments for most
courses in Computer Science.
