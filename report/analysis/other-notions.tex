% vim: set spell

\section{Other Notions}

In the past we've used notions such as ``safety'', ``fairness'', and
``security''.  In this section, we formalise these notions wrt. our system.

In general we seek that the running of other processes on the system does not
interfere with the assessment of a submission. Other processes may be other
assessments, or other parts of the system. This subdivides further into a set
of safety, fairness and security requirements.

For simplicity, we will assume that the system retains all course data, until
all of it is securely wiped by a system administrator. It follows that some of
the data may loose integrity or become vulnerable over time.

% Ensure that when the system is busy, all running assessment get a fair share
% of resources, and yet meet the requirements of their environments.

% Ensure integrity of previous submissions.

\subsection{Safety}

Safety requirements are concerned with ensuring the integrity of assessments
over time: A process $p$ should not interfere with an assessment $a$, where
$a\neq p$, in the sense that the feedback generated from assessment $a$, is the
same, regardless of whether process $p$ is ever executed.

\subsection{Fairness}

Fairness requirements are concerned with ensuring that all assessments get a
fair share of system resources: A process $p$ should not interfere with an
assessment $a$, where $a\neq p$, in the sense that the assessment $a$ gets
allocated the same system resources, and should run in roughly the same wall
clock time, regardless of whether process $p$ is ever executed.

If resources and wall-clock time cannot be guaranteed, an assessment should be
queued on a first-come-first-serve basis, provided that a student can have at
most one submission assessed at a time.

\subsection{Security}

Security requirements are concerned with ensuring that no assessment is made
vulnerable over time. An assessment is made vulnerable if it reveals data to
unauthorised users.

In \label{section:roles-in-educational-assessment}, we discussed various
enrollee roles for a system.  Below, we whitelist permissions for all these
roles.

\begin{description}

\item[Teaching Staff] \ \\ Can see all versions of and create new (or new
versions of) learning content and assignments.  Can see all student submissions
as well as processed and unprocessed feedback. Can create new versions of the
processed feedback, adding to a previous version or discarding some of it.

\item[Students] \ \\ Can see all versions of the learning content and
assignments.  Can create new personal or group (done in a group with other
students) submissions in response to assignments. Can see all past personal or
group submissions.  Can see all versions of the processed feedback on the above
submissions.

\item[General Public] \ \\ Can see all versions of the learning content and
assignments.

\end{description}
