% vim: set spell:

\section{Roles in Educational Assessment}

In the previous sections we introduced the concepts of a student and teacher,
enrolled for a course. We also used notions like ``safe and fair environment''
and ``safe and secure communication''. In this section we expand on the roles
of course enrollees, and formalise the notions above.

\subsection{Teachers}

Teaching is the expediting of learning. Students learn on their own, but
teachers facilitate learning\cite{skinner-1965}. The means of facilitation
however, vary greatly throughout the discipline\cite{ramsden-1992,kember-1997}.
Most would unite the role of teaching with information delivery and assessment.

A non-empty set of teachers is always held responsible for a course. They are
responsible for ensuring that the students acquire the knowledge or skills
defined for the course. Other teachers may be involved in the course, aiding
the course responsible teachers.

In hope of students' learning, the teachers devise means of delivering
informative content, and assessing in how far students have acquired the said
knowledge or skills. Various techniques in both information delivery and
assessment are used to facilitate and encourage learning. We are not so
concerned with information delivery, as with assessment. Never-the-less,
assignments still have to be delivered to students.

Since teachers facilitate student learning and also their final assessment,
teachers exert great authority over students. Teachers decide whether a student
passes or fails a course, what grade they get, and how hard it is for them to
get it.

\subsubsection{Teaching assistants}

Teaching assistants assist in teaching responsibilities. They are teaching
subordinates of teachers. They exert some authority over students, but are
often limited in their authority when it comes to important summative
assessments. The result is that teaching assistants perform much of the
formative assessment, and provide guiding remarks either for the purposes of
feedback or to ease important summative assessments for the teachers.

Teaching assistants come about as a scaling mechanism. Once the number of
students enrolled for a course exceeds certain numbers, certain means of
information delivery and assessment are simply infeasible for the teachers
responsible. Instead of hiring more teachers, the strategy is often to rely on
some methods of information delivery and assessment that work in large numbers,
and rely on teaching assistants for the rest. Teaching assistants are therefore
cheaper, less qualified staff that assist in teaching responsibilities.

\subsection{Learning}

Learning is the gaining of knowledge or skills. Individuals engage in learning
in hope of being somehow enlightened or trained for solving particular kinds of
problems. It is a qualitative change of an individual's view of the
world\cite{ramsden-1992}.

Those engaged in the activity of learning for the purposes of obtaining a
degree are called students.

Learning requires a motivation to learn. Students are motivated by personal development, future employment opportunities, etc. In this context however, students may sometimes fail to see 

\subsection{Censure}

Censure is a process of quality assurance of assessment. A censor's
participation in an assessment in an assessment varies from mere observation to
avid participation. A censor therefore may need access to the individual
elements of a course or of an assessment in such a way that the work of
participants of a course is personally identifiable.

\subsection{General Public}

The general public includes those who are ultimately interested in the outcomes
of education and the quality of assessment therein. This includes both
perspective students, future employers, the politically concious, etc.

The general public may be interested in open access to the elements of
education and assessment for the purposes of assessing the quality of
education. The intent may be to see if the education lives up to social
expectations, demands of the labour market, political promises, etc.

Privacy and anonymity is a matter of grave public concern. If open access is
given, it should only identify those who may reasonably be held responsible for
an eventual lack of quality in education. Also, issues of copyright have to be
taken into account.

As the general public would assess education, and not students, students should
not be personally identifiable by the general public. In how far teachers and
teaching assistants may reasonably be held responsible by the general public
for the quality of education may be a matter of university policy (as their
employer). It is important that the assessed are sufficiently informed of their
assessment and have conceded to it.

As students typically own the content they produce, individual student work or
commentary should not be made available to the general public. In how far
teachers and teaching assistants own the content they produce may again be a
matter of university policy (as their employer). It is important that the
owners of content have command over its reproduction.
