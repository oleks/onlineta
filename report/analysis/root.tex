% vim: set spell:

\chapter{Analysis}

Students make submissions in response to assignments. We would like to assess
their work with the assistance of computers.

We can rely on students to attempt to deliver their submissions in a digital
format, over an insecure network, e.g. the Internet. The network is insecure in
the following sense: it cannot guarantee that a particular submission
\begin{inparaenum}[(a)] \item comes from a particular student, nor \item that
it has not been tampered with by an adversary.\end{inparaenum}

In assessing programming assignments, we distinguish between two types of
assessment. Static assessment is the analysis of a submission without executing
student programs. Dynamic assessment is the analysis of the runtime behaviour
of student programs. A static assessment often enables a subsequent dynamic
assessment, and sometimes runs in lockstep with it.

\section{Static Analyses}

We may wish for a digital submission to adhere to a particular format, but we
cannot rely on students to meet such requirements in general. One basic use of
computer-assisted assessment is checking whether a submission meets certain
formatting requirements, before any further assessment.

For a programming assignment, we may wish for a submission to consist of a
computer program and a report --- we'll use this as a running example.

It is fairly straight-forward to enforce the requirement that a digital
submission be a non-empty set of digital files, distinguished by their file
names. Similarly, we can require files with particular names to be submitted,
stating a particular file type.  For instance, two text files, named
\texttt{main.c} and \texttt{report.txt}. It is also fairly straight-forward to
enforce a limit on the sizes of the files in a submission.  This has some
security benefits. % TODO: name them later.

We may further enforce the requirement that particular files look like they are
written in particular languages. For instance, that the computer program is
written in C, and the report is written in English. We say ``look like''
because neither language in this example has a formal definition, and so cannot
be guaranteed to be recognisable by a computer, but both are in frequent use.

For a lot of programming languages --- we're in luck --- the job of a parser,
as part of e.g. a compiler or assembler, is indeed to recognise, whether a
sequence of bits can be interpreted as a statement in that
language\footnote{The question of whether a sequence of bytes can be
interpreted as e.g. a C program, is often reduced to whether or not it is
recognisable by a particular C compiler.}.

With natural languages, being less formal in general, we're in a bit less luck.
We can often analyse various features of a report probabilistically, with
fairly high confidence. Notably, such probabilistic techniques give only
limited corrective guidance. We conjecture that reports will require subsequent
assessment by the teaching staff much more often than computer programs.

Checking that a submission meets such requirements is usually a matter of
static analysis. C++ Templates aside\cite{veldhuizen-2003}, checking that a
submission meets programming language requirements should not result in the
execution of programs written by students in a Turing-complete language.

Putting a C program through a C compiler, we get an executable out. This
enables a subsequent dynamic analysis. For other languages, we may perhaps only
have a (lazy) interpreter, where the syntax of a part of a program is (only)
checked immediately before it is executed. This would require checking a valid
syntax requirement in lockstep with a dynamic analysis.

We will often also want to check that the program is readable, testable, and
maintainable by others, that it uses particular programming abstractions, and
uses them right. This superimplies that we want submissions to include the
source code of the programs. The above requirements can then be checked by
checking that the source code adhere to comprehensive, assignment-specific,
style guides.

This amounts to a range of static analyses that we may want to perform, some of
them in lockstep with dynamic analyses.

\section{Dynamic Analyses}

We are not only concerned with students submitting good-looking source code,
but also that their programs solve the problem at hand. Sometimes, this can be
answered by a static analysis. More often however, we must resort to executing
the student programs and analyzing their runtime behaviour.

Such programs often require some input data, and produce some output data ---
they may even be interactive\footnote{For simplicity, we currently ignore
assignments that require building a graphical user interface. This still leaves
rich interactiveness capabilities.}. We need to generate data for our student
programs and validate that the data they output is correct. For a particular
assignment, we make the following observations:

\begin{enumerate}[(a)]

\item in how far an output is correct, may depend on the input;

\item there may be more than one correct output for the same input.

\end{enumerate}

Other than produce wrong results, student programs may misbehave in a myriad of
different ways, perhaps even intentionally. If let to their own devices,
student programs may never terminate, abuse memory, leak memory, fiddle with
devices, make obscure system calls, and generally fail in unpredictable ways.

Of course, this also applies to static analyses. A static analysis can be
thought of as an idealised computer, whose instructions are fed by the input to
the analysis. The designer of a static analysis may not anticipate all the
possible failure scenarios, or even let the language it accepts be
Turing-complete.

Such intentionally or unintentionally misbehaving analysis of submissions may
interfere with other analyses on the system, causing faulty assessments, or
even discarding assessments.

We must protect against such abuse for both the static and dynamic analyses. We
conjecture that the latter will be (unintentionally) abused more often than the
former. We conjecture that good feedback on either abuse, but especially the
dynamic assessment, can be an effective learning tool. For instance, a tight
bound on time and space may train students to write more efficient programs.

As we protect against such abuse, we must monitor the execution of student
programs. The data gathered from this monitoring may also be useful for further
feedback to the student.

\section{Feedback}

In case of failure, the feedback from the static and dynamic analyses may be
generally incomprehensible to a human being\cite{lerner-et-al-2007}, let alone
a student learning to program\cite{mccauley-et-al-2008}.

In general, feedback is an important pedagogical tool. We would like to support
processing the feedback, before it is delivered to the student. Such a
processor may additionally be parametrised by the original submission and some
student data to support e.g. individualized learning.

This processing may include compounding feedback from the static and dynamic
analyses to form a final feedback, providing hints to solve problems (instead
of presenting error messages), or just pass or fail the submission without
further note. We do not consider feedback processing beyond the notion that it
is a parametrised process defined on a per-assignment basis. Providing good
feedback in response to programming assignments is beyond the scope of this
work.

Feedback processing may include having a member of the teaching staff look over
the feedback to make sure that it is correct, and often, add to it. For
instance, the system may \begin{inparaenum}[(a)] \item validate that the
student submission contains a computer program and a report, \item validate
that it is written in the intended language, \item validate that it uses proper
style, and expected programming abstractions, \item validate that it produces
valid results and runs within the given resource bounds.\end{inparaenum}~This
leaves it to the teaching staff to validate that the report properly reports
about the written program.

Human involvement does not need to happen before a student is informed of the
original (automated) feedback. For instance, a common scenario might be that
the student makes a sequence of submissions, in attempt to pass the automated
testing. Once a submission passes, they can move on to a new assignment, while
they wait for more extensive feedback on their submission. The teaching staff
then, only look at the submission that passed the automated testing. Retaining
old submissions may be useful to assessment in general. 

% vim: set spell:

\section{Course Content}

The feedback aside, the teaching staff would like to deliver the assignments
themselves and various learning material to students. We conjecture that
gathering all of these elements into one system provides a better interface
between the students and the teaching staff.

Having a single system for all this content opens up authenticity and
authorisation concerns: we would like to make sure that all users of the system
corresponds to a particular enrollee on the course (authentication), and that
their user permissions correspond to their enrollee roles (authorisation).


% vim: set spell

\section{Safety, Fairness, and Security}

In the previous sections we used notions such as ``safe and fair environment'',
as well as ``safe and secure communication''. In this section, we formalise
these notions.

By ``environment'', we mean the environment within which a student submission
is automatically assessed. It is the staff that define these invironments by
specifying certain permissions and restrictions. The staff also provide various
analysis programs, which examine a student submission within an environment.

Such environments should be safe in the following senses:

\begin{itemize}

\item The assessment of one submission, cannot interfere with another. Be that
submission from the same student or another, in the past or future.

\item The submission cannot interfere with the analyses supplied by the staff
to produce a better result than intended.

\end{itemize}

The last point is not feasible in general. Although we provide some protection,
we rely on the staff to produce good programs for analysis.

For such environments, we have some
requirements wrt. permissions and restrictions, set by the staff. In addition
to environment being safe and secure, we intend to also meet these
requirements.

\subsection{Safety}

A user using the system should not interfere

We would like to provide for safe environments and communications. The assessment of a submission should be 

In the first case, an environment (for analysing student programs) should be atomic, and not be interfered 

\subsection{Fairness}

We would like to ensure fair service to users of all roles on the system.

We conjecture that students, teachers, teaching assistants, and external
examiners, would all find it fair to be allocated an equal share of the system
resources under maximum loads, provided that the permissions and requirements
for assessment environments can still be met. When the system is not busy, the
active users get all the resources they need, so long as they don't .

\subsection{Security}




% % vim: set spell:

\section{Purpose and Environment}

The purpose of the OnlineTA is to aid traditional on-site education at the
Department of Computer Science, University of Copenhagen, Denmark (DIKU).

Denmark is a signatory of the Bologna process, an international European
agreement for the development of a common European Higher Education Area
\cite{ehea1999}. The intent of the agreement is to further international
cooperation, allow students and researchers to move freely across borders, and
to make European universities more attractive worldwide.

To adhere to the agreement, an education at the University of Copenhagen is
split into two main cycles: a 3-year Bachelor education, and a subsequent
2-year Master's education, and as the Danish law \cite{bek814} mandates:

\begin{itemize}

\item \textbf{\textsection\ 6, part 1 \& 2:} A Bachelor and Master's
education is accredited using the European Credit Transfer System (ECTS).

\item \textbf{\textsection\ 6, part 3:} 60 ECTS points corresponds to 1 year's
worth of full time studies. 

\item \textbf{\textsection\ 11, part 1:} A Bachelor education is attributed
180 ECTS points.

\item \textbf{\textsection\ 17, part 1:} A Master's education is attributed 120
ECTS points.

\end{itemize}

The Department of Computer Science is a subordinate of the Faculty of Science
at the University of Copenhagen. As such, it adheres to the faculty-wide block
structure\cite{cnd2004}. A study year is split into 4 blocks, each
corresponding to 9 weeks worth of full time work and 15 ECTS credits. Most
courses require half time work for 9 weeks, yielding 7.5 ETCS credits. Some
courses span multiple blocks, or require full time work, yielding 15 ECTS
credits, or more. Courses aside, students can also take up projects, write a
Bachelor project, or a Thesis, but we will not concern ourselves with such
matters here.

According to Danish law \cite[\textsection 6, parts 1 \& 2]{bek814}, a
Bachelor and Master's education is built up of multiple modules. A module is
a unit of study, providing the student with some qualifications and skills,
yielding a certain number of ECTS points, and completed by passing one or more
exams. A DIKU course roughly corresponds to a module: every course is completed
by passing an exam.

According to Danish law \cite[\textsection 14, parts 1 \& 4]{bek666}, a
student enrolled for a course is automatically admitted to the exam in that
course, unless the student throughout the course does not meet the requirements
beforehand set by the university for being admitted to the exam. According to
Danish law \cite[\textsection 3, parts 2 \& 3]{bek666} this may be
requirements of participation, requirements to submit written assignments
during the course, or similar. The requirements are further specified by the
university curriculum.

The Faculty of Science at the University of Copenhagen has both a shared BSc
and MSc curricula for study programmes\cite{faelles-del-ENG}, and
programme-specific curricula.

According to the shared faculty curricula, the individual courses offered by
the faculty, including the requirements to be met in order to be admitted to
the exam, are described in the University of Copenhagen course catalogue, at
\url{http://kurser.ku.dk}. This excludes the aforementioned projects, Bachelor
project and Thesis, which are described in the faculty curricula itself.

Most courses at DIKU run for the aforementioned 9 weeks, and expect half time
study. The first 7 of the weeks are used for teaching and various mandatory
written assignments, one week is left for exam study, and the last week is left
for the exams themselves. The written assignments serve as the basis for the
judgement of whether the individual student should be admitted to the exam or
not.

The purpose of the OnlineTA is to aid in the judgement of whether a student is
to be admitted to the exam as well as aid in the evaluation of the student's
exam performance itself.

\section{Notes}

Bestået/ikke bestået. Mødepligt.

This is the
main decision to be made in the course of the first 7 weeks, individually for
every student. The purpose of the OnlineTA is to aid in this judgement.

The purpose of the eduction is, in part, to develop competence in computable
processes and structures. This typically involves both theory and practice,
i.e. both reasoning about computable processes and structures, and implementing
them on a computer. The two approaches reinforce each other, providing insight
both into what is reasonable, and what is feasible.

% If we use a reference to CS2013, we have to justify why its relevant to the
% education at DIKU.

Computer Science is not a unimodal science. There are many different topics,
and many different areas of interest, all of which cannot fit into the head of
any one student.

A course spans a period of time, has a curriculum, and demands the student to
complete some sort of individual work. It is the student's individual work that
forms the basis for the evaluation of the student's performance. It is the
accumulative performance of the students in all their courses that form their
degrees.

All courses may either be passed or not passed by a student.

Most courses are graded on a 7-grade scale, but some are merely pass or no pass

 which will be used as the basis for the assessment of a student's work.

A course is a something that spans a 

The education involves the student participating in various courses. A course
spans a period of time, has a particular curriculum, and particular problems
that the student is asked to solve. The performance of the student is then
judged on the basis of their work.

 The system
should aid the teaching staff in their judgements of individual students.

Of particular interest are programming courses, i.e. courses that involve the
writing of programs to be executed by a computer.


% What is a domain analysis? Why not simply a ``requirements analysis''?

% environment

% The environment of the system - a university department, computer science.
% Talk about the structure of a course -- someone not familiar with a modern
% university department, or our department in particular should be able to read
% this report. -- Note the reader expectations.

% What sort of tasks are involved in running a course, and who performs those
% tasks -- what are the different user roles?
