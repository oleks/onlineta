% vim: set spell:

\section{Dynamic Analyses}

We are not only concerned with students submitting good-looking source code,
but also that their programs solve the problem at hand. Sometimes, this can be
answered by a static analysis. More often however, we must resort to executing
the student programs and analyzing their runtime behaviour.

Student programs often require some input data, and produce some output data
--- they may even be interactive\footnote{For simplicity, we currently ignore
assignments that require building a graphical user interface. This still leaves
rich interactiveness capabilities.}. For a particular assignment, we make the
following observations:

\begin{enumerate}[(a)]

\item in how far an output is correct, may depend on the input;

\item there may be more than one correct output for the same input.

\end{enumerate}

This requires for an input generator and an output validator to run in lockstep
with a student program. We'll regard them as simple interactive (stdin/stdout)
processes, and leave their design to teaching staff in general.

Other than produce wrong results, student programs may misbehave in a myriad of
different ways, perhaps even intentionally. If let to their own devices,
student programs may never terminate, abuse memory, leak memory, fiddle with
devices, make obscure system calls, and generally fail in unpredictable ways.

Of course, this also applies to static analyses. A static analysis can be
thought of as an idealised computer, whose instructions are fed by the input to
the analysis. The designer of a static analysis may not anticipate all the
possible failure scenarios, or even let the language it accepts be
Turing-complete.

Such intentionally or unintentionally misbehaving analyses of submissions may
interfere with other analyses, causing faulty assessments. We seek to protect
against such abuse for both the static and dynamic analyses.

We conjecture that the latter will be (unintentionally) abused more often than
the former. Furthermore, good feedback on either abuse, but especially the
dynamic assessment, can be an effective learning tool. For instance, a tight
bound on time and space may train students to write more efficient programs.

In general, dynamic analyses generate feedback wrt. whether the student program
produces the expected results, and its resource use (or abuse). A misbehaving
student program may be halted prematurely. We intend to process this feedback
before handing it back to the students.
