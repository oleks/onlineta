% vim: set spell

\section{Safety, Fairness, and Security}

In the previous sections we used notions such as ``safe and fair environment'',
as well as ``safe and secure communication''. In this section, we formalise
these notions.

In general we seek that the running of other processes on the system does not
interfere with the assessment of a submission. Other processes may be other
assessments, or other parts of the system. This subdivides further into a set
of safety, fairness and security requirements.

% Ensure that when the system is busy, all running assessment get a fair share
% of resources, and yet meet the requirements of their environments.

% Ensure integrity of previous submissions.

\subsection{Safety}

Safety requirements are concerned with ensuring the integrity of assessments
over time.

A process $p$ should not interfere with an assessment $a$, where $a\neq p$, in
the sense that the feedback generated from assessment $a$, is the same,
regardless of whether process $p$ is ever run or not.

\subsection{Fairness}

Fairness requirements are concerned with ensuring that all assessments get a
fair share of resources.

A process $p$ should not interfere with an assessment $a$, where $a\neq p$, in
the sense that the assessment $a$ gets allocated the same system resources, and
should run in roughly the same wall clock time, regardless of whether process
$p$ is ever run or not. If wall-clock time cannot be guaranteed, an assessment
should be queued on a first-come-first-serve basis, provided that a student can
have at most one submission assessed at a time.

\subsection{Security}

Security requirements are concerned with ensuring that no assessment is made
vulnerable over time. An assessment is made vulnerable if it reveals data to
unauthorised users.

Who should be authorized to see what has already been discussed in
\referToSection{analysis:roles-in-educational-assessment}.
