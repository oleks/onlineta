% vim: set spell

\section{Safety, Fairness, and Security}

In the previous sections we used notions such as ``safe and fair environment'',
as well as ``safe and secure communication''. In this section, we formalise
these notions.

By ``environment'', we mean the environment within which a student submission
is automatically assessed. It is the staff that define these invironments by
specifying certain permissions and restrictions. The staff also provide various
analysis programs, which examine a student submission within an environment.

Such environments should be safe in the following senses:

\begin{itemize}

\item The assessment of one submission, cannot interfere with another. Be that
submission from the same student or another, in the past or future.

\item The submission cannot interfere with the analyses supplied by the staff
to produce a better result than intended.

\end{itemize}

The last point is not feasible in general. Although we provide some protection,
we rely on the staff to produce good programs for analysis.

For such environments, we have some
requirements wrt. permissions and restrictions, set by the staff. In addition
to environment being safe and secure, we intend to also meet these
requirements.

\subsection{Safety}

A user using the system should not interfere

We would like to provide for safe environments and communications. The assessment of a submission should be 

In the first case, an environment (for analysing student programs) should be atomic, and not be interfered 

\subsection{Fairness}

We would like to ensure fair service to users of all roles on the system.

We conjecture that students, teachers, teaching assistants, and external
examiners, would all find it fair to be allocated an equal share of the system
resources under maximum loads, provided that the permissions and requirements
for assessment environments can still be met. When the system is not busy, the
active users get all the resources they need, so long as they don't .

\subsection{Security}


