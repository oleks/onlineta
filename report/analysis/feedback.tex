% vim: set spell:

\section{Feedback}

In case of failure, the feedback from the static and dynamic analyses may be
generally incomprehensible to a human being\cite{lerner-et-al-2007}, let alone
a student learning to program\cite{mccauley-et-al-2008}.

In general, feedback is an important pedagogical tool. We would like to support
processing the feedback, before it is delivered to the student. Such a
processor may additionally be parametrised by the original submission and some
student data to support e.g. individualized learning.

This processing may include compounding feedback from the static and dynamic
analyses to form a final feedback, providing hints to solve problems (instead
of presenting error messages), or just pass or fail the submission without
further notice. We do not consider feedback processing beyond the notion that
it is an interactive (stdin/stdout) process, defined per-assignment by the
teaching staff. Providing good feedback in response to programming assignments
is beyond the scope of this work.

Feedback processing may include having a member of the teaching staff look over
the feedback to make sure that it is correct, and often, to add to it. We would
like to give the teaching staff the ability to elaborate on, or discard
feedback after it has already been given to the student. That is, automatic
feedback is given immediately, but can be overridden by the teaching staff,
yielding new feedback for the student. This way, a student can get immediate
(automatic) feedback on their program, and later, (manual) feedback on their
report.

This is for both a particular submission, or all submissions for an assignment
at once. The latter is especially useful when an error is discovered in either
the input generator, the output validator, or the feedback processor ---
all processes designed by the teaching staff.
