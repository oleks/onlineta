% vim: set spell:

\section{Feedback}

In case of failure, the feedback from the static and dynamic analyses may be
generally incomprehensible to a human being\cite{lerner-et-al-2007}, let alone
a student learning to program\cite{mccauley-et-al-2008}.

In general, feedback is an important pedagogical tool. We would like to support
processing the feedback, before it is delivered to the student. Such a
processor may additionally be parametrised by the original submission and some
student data to support e.g. individualized learning.

This processing may include compounding feedback from the static and dynamic
analyses to form a final feedback, providing hints to solve problems (instead
of presenting error messages), or just pass or fail the submission without
further note. We do not consider feedback processing beyond the notion that it
is a parametrised process defined on a per-assignment basis. Providing good
feedback in response to programming assignments is beyond the scope of this
work.

Feedback processing may include having a member of the teaching staff look over
the feedback to make sure that it is correct, and often, add to it. For
instance, the system may \begin{inparaenum}[(a)] \item validate that the
student submission contains a computer program and a report, \item validate
that it is written in the intended language, \item validate that it uses proper
style, and expected programming abstractions, \item validate that it produces
valid results and runs within the given resource bounds.\end{inparaenum}~This
leaves it to the teaching staff to validate that the report properly reports
about the written program.

Human involvement does not need to happen before a student is informed of the
original (automated) feedback. For instance, a common scenario might be that
the student makes a sequence of submissions, in attempt to pass the automated
testing. Once a submission passes, they can move on to a new assignment, while
they wait for more extensive feedback on their submission. The teaching staff
then, only look at the submission that passed the automated testing. Retaining
old submissions may be useful to assessment in general. 
