% vim: set spell:

\section{Namespaces}

\label{section:sandboxing:namespaces}

The purpose of a Linux namespace is to abstract over a system resource, and
make it appear to tasks within the namespace, as though they have their own
independent instance of the resource. Various namespace types abstract over
various system resources. For each namespace type, a task is associated with
exactly one namespace of that type.

Namespaces are hierarchical in the following a sense: A system boots with one
global namespace for each namespace type.  Tasks inherit their parent's
namespaces by default. A task can be disassociated from a namespace and
associated with another at runtime.

A task can be associated with a namespace using the \cite{man-2-unshare},
\cite{man-2-setns}, or \cite{man-2-clone} system calls.  The first
disassociates the process from a namespace, associating it with a new
namespace. The second, associates the process with an already existing
namespace. The last is a general system call for task creation, allowing to
create a task, already in a new namespace.

We say that a parent namespace is a ``host'', and a child namespace is a
``container''. A host may host many containers, and a container may have many
hosts, i.e. all of its ancestors. Typically, we'll only talk about a direct
child-parent relationship.

We discuss various namespace types in the following subsections. Some types,
such as mnt, pid, and net, require for a user to be priveleged to create a
child namespace. We also discuss some technical details in
\referToAppendix{namespaces}.

\subsection{mnt}

The mount (MNT) namespace abstracts over the mount points of a system. This
allows for processes in different namespaces to have different views of the
file system. Within a container, we can unmount points that are perhaps needed
by the host, but not by the container, and would perhaps make the host
vulnerable, if the container had access to them.

\subsubsection{Pivot root}

One particularly useful application of mount namespaces is pivoting the file
system root to some other point in the file system using
\cite{man-2-pivot-root}. Pivoting the root in a container does not affect the
host, or other containers. At the same time, pivoting the root moves all the
dependencies on the old root, to a new root within the container.

This allows us to subsequently unmount the old root, provided that the new root
does not depend on this mount point. This can be achieved by having the new
root mounted as a \texttt{tmpfs}, or perhaps a read-only \texttt{squashfs}.
This hides the original root file system in a matter similar to
\cite{man-2-chroot}, but makes reestablishing the old root slightly more
cumbersome, since the old root has to be properly remounted first.

\subsection{uts}

The UNIX Time-sharing System (UTS) namespace abstracts over the host- and domain
name of a system. This allows each container to retain a personal host- and
domain name, perhaps different from the underlying host.

\subsection{ipc}

The Interprocess Communication (IPC) namespace abstracts over the IPC
resources, in particular System V IPC objects and POSIX message queues.
Processes within one namespace cannot communicate with processes in another
namespace using these primitives.

\subsection{pid}

The Process Identifier (PID) namespace abstracts over the task identifiers of a
system. Tasks in different namespaces can have the same pid within their
respective namespaces, but they all have distinct pid's on their hosts.
Hierarchies are implemented for pid namespaces such that a host can see all the
processes created within a container, while a container cannot see any of the
processes on a host. Effectively, the first process in a child namespace gets
pid 1, the same as an init process.

\subsection{net}

The Network (NET) namespace abstracts over the system resources associated with
networking. Each network namespace has its own network devices, IP addresses,
IP routing tables, port numbers, etc.

For simplicity, we'll prohibit students in doing in networking. This is easy to
limit with a network namespace --- all networking configuration of the host is
dropped for a new child namespace.

\subsection{user}

\label{section:sandboxing:namespaces:user}

The User (USER) namespace abstracts over the user and group ID number space.
A new namespace has its own set of identifiers, starting at 0.

Hierarchies are implemented such that a user id in a container is mapped to a
user id on the host (and likewise for groups). For instance, we can have a
particular designated ``container user'' on the host, and map this user to UID
0 inside the container. This way, even if a malicious user managed to perform a
privilege escalation within the container, this would merely correspond to an
unprivileged user on the host.

This opens up containers to a wide range of capabilties, which would have
otherwise required a privileged user on the host. For instance, containers can
now be created using an unprivileged user in general, by creating a user
namespace first.

Arguably, this leaves too much of the kernel wide open for a container, and
many find that user namespaces deserve to be tested further before being
enabled by default\cite{kerrisk-2013, arch-linux-userns,fedora-userns}.
