% vim: set spell:

\section{Namespaces}

The purpose of a Linux namespace is to abstract over a global resource, and
make it appear to processes within the namespace, as though they have their own
isolated instance of the global resource. Various types of namespaces allow to
abstract over various global resources.

A task can be associated with a namespace using an \cite{man-2-unshare},
\cite{man-2-setns}, or \cite{man-2-clone} system call. The first disassociates
the process from a namespace, associating it with a new namespace. The second
reassociates the process with an existing namespace. The last is the general
system call for task creation, allowing to create a task in a new namespace.

The namespaces that a task is associated with are identified in the
\cite{man-5-proc} pseudo file system. The general pattern of the file names is
\texttt{/proc/[pid]/ns/[nstype]}, where \texttt{[pid]} is the thread group
identifier of the task (see also
\referToAppendix{general-linux-concepts:tasks}), and \texttt{[nstype]} is one
of a range of supported namespace types.

We discuss some of these types in the following sections. Support for more
types may come in the future, as containers demand more resource isolation.
Furthermore, not all of these types are necessarily enabled on your system. The
user namespace is frequently omitted by many distributions as it opens up a
large part of the kernel, previously not available to the non-privileged user.
Some believe it requires a lot more testing before being enabled by default.

To our knowledge, enabling a namespace type requires compiling your own kernel.
You can check which namespaces are enabled on your system, by reading the
\texttt{/proc/config.gz} file, or listing the files in a process namespace
subdirectory. For instance,

\begin{lstlisting}
$ ls /proc/$$/ns
ipc  mnt  net  pid  uts
\end{lstlisting}

\subsection{mnt}

The mount(mnt) namespace abstracts over the mount points of a system. This
allows for processes in different namespaces to have different views of the
file system. Within a container, we can unmount points that are perhaps needed
by the host, but not by the container, and would perhaps make the host
vulnerable, if the container had access to them.

\subsubsection{Pivot root}

One particularly useful application of mount namespaces is pivoting the file
system root to some other point in the file system using
\cite{man-2-pivot-root}. Pivoting the root in a container does not affect the
host, or other containers. At the same time, pivoting the root moves all the
dependencies on the old root, to a new root within the container.

This allows us to subsequently unmount the old root, provided that the new root
does not depend on this mount point. This can be achieved by having the new
root mounted as a \texttt{tmpfs}, or perhaps a read-only \texttt{squashfs}.
This hides the original root file system in a matter similar to
\cite{man-2-chroot}, but makes reestablishing the old root slightly more
cumbersome, since the old root first has to be properly remounted first.

\subsection{uts}

The UNIX Time-sharing System(UTS) namespace abstracts over the host- and domain
name of a system. This allows each container to retain a personal host- and
domain name, perhaps different from the underlying host.

\subsection{ipc}

\subsection{pid}

\subsection{net}

\subsection{usr}
