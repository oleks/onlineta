% vim: set spell:

\section{Linux Security Modules}

Linux Security Modules (LSM) is an access control framework. It provides a
mechanism for various security checks to be raised whenever a kernel operation
may result in an access control violation\cite{wright-et-al-2002,lsm.txt}.

There are a number of competing kernel extensions which hook into the LSM
framework to provide for comprehensive security policies. We discuss some of
these below.

\subsection{SELinux}

The Security Enhanced Linux\textsuperscript{\textregistered} (SELinux) project
predates LSM. It was initially envisioned inside the US National Security
Agency (NSA). After 10 years of development and testing, it was released to the
general public under a GPL license in 2000\cite{ivashko-2012}.

When suggested to include it in the mainstream Linux kernel, Linus Torvalds
charged the security community to make it a module instead, as there were other
competing projects out there, e.g. AppArmor. The result was the implementation
of LSM in the kernel\cite{wright-et-al-2002}.

\subsection{AppArmor}

AppArmor is perhaps a more community-driven alternative to SELinux, which is
presumably easier to use and more portable than SELinux\cite{spenneberg-2006}.
This sometimes comes at the cost of some security. For instance, with AppArmor,
file system policies are enforced based on file paths, rather than inodes,
protecting paths, not data.

\subsection{Capabilities}

Linux capabilities is an effort to subdivide the privileges of a privileged
user into units. This allows to grant a user particular privileges, without
making them privileged in general.

Linux capabilities is the default LSM for the Linux kernel\cite{lsm.txt}.
Rather intentionally, Linux capabilities provide only very coarse-grained
access control options, compared to e.g. SELinux or AppArmor. For simplicity,
we choose to deal with Linux capabilities rather than any other LSM.

Linux capabilities are useful to us in connection with user namespaces. We may
designate a container user or group on the host, having extended capabilities.
This user may be used to create a user namespace, creating the illusion full
capabilities, but having only limited capabilities on the host.
