% vim: set spell:

\section{Control Groups}

Control groups (cgroups) provide a mechanism of hierarchically
grouping/partitioning tasks (see also
\referToAppendix{general-linux-concepts:tasks}) and their future
children\cite{cgroups-cgroups-since-linux-3.14}. On their own, cgroups are
perhaps only useful for simple job tracking. The idea, is to have other
subsystems hook into the cgroups functionality and provide accounting or
limiting of various system resources.

The canonical cgroup subsystems include subsystems to monitor and limit memory,
CPU time, I/O, and device activity. Most modern Linux distributions come with
control groups and many of these canonical subsystems enabled. The system's
\texttt{/proc/config.gz} can reveal the setup on your system\cite{man-5-proc}.
If \texttt{CONFIG\_CGROUPS} is enabled, you have cgroups support. The variables
related to various subsystems are explored further in the following
subsections.

In the first subsection, we discuss how cgroups can be accessed and manipulated
from user space in general. The remaining subsections discuss the relevant
canonical subsystems in greater depth.

\subsection{Managing cgroups}

Cgroups are managed via a pseudo-filesystem: cgroups reside in memory, but can
be manipulated through the virtual file system.  \texttt{cgroup} is therefore
an inherent file system type on systems that have the cgroups functionality
enabled.

When mounting a cgroup file system, we mount a particular control group
hierarchy. A hierarchy is a set of cgroups arranged in a tree.  Multiple
subsystems can be combined and mounted onto a single target.  The subsystems to
mount can be listed as mount options:

\begin{lstlisting}
$ mount -t cgroup -o cpu,cpuacct cgroup ./cgroup/cpu,cpuacct
\end{lstlisting}

This combines and mounts the \texttt{cpu} and \texttt{cpuacct} cgroup
subsystems in the (pre-existing) subfolder \texttt{./cgroup/cpu,cpuacct}. Once
a series of subsystems are mounted onto a target, the series can be mounted
onto another target. However, the root hierarchies involved cannot be combined
in new ways. For instance, the following fails with a rather cryptic message:

\begin{lstlisting}
$ mount -t cgroup -o cpu,memory cgroup ./cgroup/fun
mount: cgroup is already mounted or ./cgroup/fun busy
\end{lstlisting}

How various cgroup root hierarchies are combined and mounted depends on the
system at hand. The system's \texttt{/proc/mounts} can reveal how this is setup
on your system.

After a successful cgroup file system mount, we see a range of files below our
target. Some files are common to all cgroups, others are specific to the chosen
root hierarchies. We cover some of the common files below:

\begin{description}[\setleftmargin{0.2in}\breaklabel\setlabelstyle{\tt}]

\item [cgroup.procs] Lists the set of thread group IDs in the current cgroup.
Appending a thread group ID to this file moves all the threads in the thread
group into this cgroup.

\item [tasks] Lists the set of thread IDs in the current cgroup. Appending a
thread ID to this file moves the thread into this cgroup.

\end{description}

In addition to files, we see subfolders..

We discuss the various cgroup type-specific files in the next subsections. Of
interest to us here, is only the \texttt{tasks} file. Every cgroup has such
file, listing a set of PIDs in this cgroup.  To add a task to a cgroup, we
simply append its PID to the end of this file.

Cgroups are hierarchical. To create a new subhierarchy 

\subsection{memory}

\subsection{cpuacct}

\subsection{cpu}

\subsection{cpuset}

\subsection{devices}

\subsection{blkio}
