% vim: set spell:

% TODO: test instance

\section{Control Groups}

\label{section:sandboxing:control-groups}

Control groups (cgroups) provide a mechanism of hierarchically
grouping/partitioning tasks (see also
\referToAppendix{general-linux-concepts:tasks}), and their
children\cite{cgroups.txt}. Some more technical details, as well as less
related aspects of cgroups are discussed in
\referToAppendix{technical-details:control-groups}.

On their own, cgroups are perhaps only useful for simple job tracking. The
idea, is to have other subsystems hook into the cgroups functionality and
provide for management of system resources. The standard cgroup subsystems
include subsystems to monitor and limit memory, CPU time, I/O, and device
activity. Subsystems therefore are often also called ``resource controllers''.

\subsection{Managing cgroups}

Cgroups are managed via a pseudo-filesystem: cgroups reside in memory, but can
be manipulated through the virtual file system.  ``cgroup'' is therefore an
inherent file system type on systems that have cgroups enabled.

\subsubsection{Cgroups, subsystems, and hierarchies}

A cgroup is an association of a set of tasks with a set of preferences for a
set of subsystems.  A hierarchy is a set of cgroups arranged in a rooted tree.
Every task on the system is attached to exactly one cgroup in the hierarchy.
All cgroups in the hierarchy, associate their tasks with the same set of
preferences --- we say that a hierarchy is associated with a set of
subsystems\footnote{\cite{cgroups.txt} is ambiguous wrt. whether a hierarchy
can exist without being associated with at least one subsystem. For the sake of
simplicity, we'll assume that it can; although cgroups without associated
subsystems have few practical applications, as we have already discussed.}.
\referToFigure{hierarchy-example} illustrates a couple example hierarchies.

\begin{figure}[h!]

\centering

\begin{tikzpicture}

\draw (-3.5, 0) node(aa){$\chev{\set{1,2},\set{\alpha,\beta}}$};

\draw (-5, -1) node(ab){$\chev{\set{3},\set{\alpha,\beta}}$};

\draw (-2, -1) node(ac){$\chev{\set{},\set{\alpha\beta}}$};

\draw (1.7, 0) node(ba){$\chev{\set{},\set{\gamma}}$};

\draw (0.5, -1) node(bb){$\chev{\set{1},\set{\gamma}}$};

\draw (3, -1) node(bc){$\chev{\set{2,3},\set{\gamma}}$};

\draw [->] (aa) -- (ab);

\draw [->] (aa) -- (ac);

\draw [->] (ba) -- (bb);

\draw [->] (ba) -- (bc);

\end{tikzpicture}

\caption[]{An illustration of two example cgroup hierarchies on a particular
system. The set of identifiers of the tasks running on the system is
$T=\set{1,2,3}$. The set of subsystems available on the system is
$S=\set{\alpha,\beta,\gamma,\delta}$ (not all subsystems need be associated
with a hierarchy). Every node (cgroup) has type
$\mathcal{P}(T)\times\mathcal{P}(S)$, where $\mathcal{P}$ denotes the
powerset.}

\label{figure:hierarchy-example}

\end{figure}

\subsubsection{Mounting}

When mounting a cgroup file system, we create a new hierarchy. The set of
subsystems to associate with the hierarchy is listed as mount
options\footnote{Omitting the subsystem list, attempts to associate all
subsystems available on the system.}:

\begin{lstlisting}
$ mount -t cgroup -o cpu,cpuacct cgroup ./cgroup/cpu,cpuacct
\end{lstlisting}

This associates the ``cpu'' and ``cpuacct'' subsystems with a new hierarchy,
and mounts the hiearchy under the target \texttt{./cgroup/cpu,cpuacct}, unless
one of the subsystems is busy.

A subsystem is busy e.g. if it is associated with a hierarchy having tasks
attached.  Since a hierarchy is automatically attached to all tasks on the
system, this effectively means that a subsystem may be associated with at most
one hierarchy. If there already exists a hierarchy, assosciated with the exact
same set of subsystems, the hierarchy will be mirrored under the new target.

\subsubsection{Control files and child groups}

After a hierarchy is successfully mounted, we see a range of files, and perhaps
folders, below our target.

We refer to these as control files and child groups, respecitvely. We
monitor/modify the preferences of a cgroup by monitoring/modifying the control
files. We create/remove child groups by creating/removing subdirectories below
our target.

With a few exceptions at the root of a hierarchy, all cgroups contain the same
files, created when the cgroup is created. Some files are common to all
hierarchies, others are due to the associated subsystems. One common file is of
particular interest:

\begin{description}[\setleftmargin{0.2in}\breaklabel\setlabelstyle{\tt}]

\item [cgroup.procs] Lists the set of thread group IDs in the current cgroup.
Appending a thread group ID to this file attaches the thread group to this
cgroup.

\end{description}

\subsubsection{Hierarchical accounting}

Hierarchical accounting when resource accounting is aware of the child groups
of a cgroup.  For instance, all resource usage is summed up for all tasks in
the cgroup, and recursively for all child groups.  Limits are then imposed on
the subhierarchy rooted at the cgroup. A subsystem does not necessarily perform
hierarchical accounting.

\subsection{The Resource Counter}

\label{section:cgroups:resource-counter}

The resource counter is a framework for managing a resource when using control
groups\cite{resource-counter.txt}.  The internal data structures aside, the
framework makes recommendations wrt. the control files. A couple of the
recommended control files are of interest to us:

\begin{description}[\setleftmargin{0.2in}\breaklabel\setlabelstyle{\tt}]

\item[<resource>.max\_usage\_in\_<unit\_of\_measurement>] Reading this file, we
get the maximal usage of the resource over time, in the given units. Writing to
this file, resets the value to the current usage of the resource. (The data
written is ignored.)

\item[<resource>.limit\_in\_<unit\_of\_measurement>] Reading this file, we get
the maximal allowed usage of the resource, in the given units. Writing to this
file resets the limit to the given value. A special value may indicate no
limit.

\end{description}

These files are of interest to us as they allow us to probe the usage of a
resource in a test instance and set up resource limits for students or staff.

\subsection{memory}

The memory subsystem allows us to monitor and limit the memory usage of the
tasks in a cgroup\cite{memory.txt}. This includes both user and kernel memory
and swap usage.  The subsystem optionally performs hierarchical accounting.

The memory subsystem uses a resource counter for a couple different memory
resources. The resource counter control files (see also
\referToSection{cgroups:resource-counter}) are prefixed as follows:

\begin{description}[\setleftmargin{0.2in}\breaklabel\setlabelstyle{\tt}]

\item[memory]

The main memory counter. This includes both user and kernel memory.

\item[memory.memsw]

The main memory, plus swap. Limiting this value to the same value as the main
memory controller, disables swap.

\item[memory.kmem]

Kernel memory. All kernel memory is also accounted for by the main memory
counter. It is not necessary to limit this value if swapping is disabled and
there is a limit on the main memory counter (since kernel memory cannot be
swapped out).

\item[memory.kmem.tcp]

Kernel TCP buffer memory. Although we will disallow networking in general, it
might be a good idea to 0-limit this resource as an extra precaution.

\end{description}

The limits and usage are always measured in bytes. Setting the limit to
\texttt{-1}, removes the limit on the resource.

\subsection{cpuacct}

The CPU accounting (cpuacct) subsystem allows us to monitor the CPU time usage
of the tasks in a cgroup\cite{cpuacct.txt}. The cpuacct subsystem always
performs hierarchical accounting.

The cpuacct subsystem provides a couple control files of interest:

\begin{description}[\setleftmargin{0.2in}\breaklabel\setlabelstyle{\tt}]

\item[cpuacct.usage] Shows the total CPU time spent by the cgroup, in
nanoseconds.

\item[cpuacct.usage\_percpu] Shows the total CPU time spent by the cgroup, for
each CPU core, in nanoseconds.

\item[cpuacct.stat] Shows a further division of the CPU time spent. For now,
showing how much of the CPU time was spent running in user mode, and how much
in kernel mode, in the \texttt{USER\_HZ} time unit.

\end{description}

\subsection{cpu}

The cpu subsystem facilitates CPU scheduling parameters for a cgroup.

The parameters facilitate control over e.g. the Completely Fair Scheduler (CFS)
in the Linux kernel\cite{sched-design-cfs.txt}. CFS is a proprtional share CPU
scheduler. The CPU time is divided fairly among tasks depending on their
priority and the share assigned to their cgroup.

The CFS parameters facilitate first-and-foremost, the enforcing of a lower
bound on the amount of CPU time allocated to a cgroup. This is done by
assigning a relative share (weight) to each cgroup. The shares are enforced,
only if tasks from different cgroups are competing for CPU time. This means
that if a cgroup gets no competition, it gets all the CPU time it wants.

With the advent of ``cloud computing'', it has also become relevant to enforce
upper bounds on the CPU time over a period of time\cite{turner-et-al-2010}. We
choose to let the students spend all the CPU time they want, as long as fair
service is ensured for all students and staff. There is therefore only one
control file in this subsystem of interest to us:

\begin{description}[\setleftmargin{0.2in}\breaklabel\setlabelstyle{\tt}]

\item[cpu.shares] Show/set the relative CPU time share of a cgroup. Two cgroups
having share 100, will be given equal service. If one of the groups has share
200, it gets twice as much CPU time under a fully-loaded system. The control
file in the root cgroup, provides for a yard-stick for all other cgroups.

\end{description}

The cpu subsystem paramenters do not allow us to hard limit the amount of CPU
time used by a cgroup in total. To our knowledge there is no ``natural'' way of
doing this in the Linux kernel. We must make due with limiting the wall-clock
time of our programs.

\subsection{devices}

\label{section:control-groups:devices}

The devices subsystem allows us to mandate access to device nodes (files) using
cgroups\cite{devices.txt-a}. The limits are enforced hierarchically using
whitelists --- a cgroup further down in the hierarchy cannot access devices to
which access has not been granted further up in the hierarchy.

A whitelist entry consists of four fields: the device node type, the major and
minor device node identifiers (2 fields), and an access specifier. The access
specifier is a sequence of characters, where \texttt{r} signifies read access,
\texttt{w} write access, and \texttt{m} device node creation.

The device node type is either \texttt{c} for character, \texttt{b} for block,
or \texttt{a} for all devices. Using \texttt{a}, discards all other options,
and implies full access to all devices of all types. This is useful if you want
to mandate universal access. The major and minor identifiers is what Linux uses
to uniquely identify devices.

The following control files are of interest to us:

\begin{description}[\setleftmargin{0.2in}\breaklabel\setlabelstyle{\tt}]

\item[devices.allow] Writing an entry to this file adds an entry to the device
whitelist.

\item[devices.deny] Writing an entry to thisfile removes an entry from the
device whitelist.

\end{description}
