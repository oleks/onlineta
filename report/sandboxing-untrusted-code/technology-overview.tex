% vim: set spell:

\section{Technology Overview}

A program is executed within a program execution environment. A sandboxed
execution environment ensures the non-interference of the sandboxed program
with other programs on the system. There are two general approaches to
sandboxing: sandboxing the operating system, or sandboxing programs within the
operating system.

In \referToSection{analysis}, we discussed that for each submission we would
like to run a range of static and dynamic analyses, and run their feedback
through a feedback processor. Furthermore, a dynamic analysis constitutes
executing the student program, an input generator, and an output validator, all
in lockstep with one another. This constitutes a range of simultaneous and
interactive processes involved in an assessment.

If the assessment happens all in a single sandboxed operating system, it seems
infeasible to monitor, or impose limits on, e.g. the student program,
independent of the rest. Alternatively, we could host each separate process in
its own, sandboxed operating system. Although seemingly more useful, it adds to
the complexity of communication between the processes of an assessment.

We choose to discard sandboxing the operating system, and consider sandboxing
within the operating system. As we shall see, this allows for fine-grained
resource monitoring and limits, while retaining simple means of interprocess
communication.

\subsection{Operating system-level virtualization}

With operating system-level virtualization (OSLV), instead of running an
operating system within a virtual environment, we create virtual environments
within the operating system --- all sandboxes run within the same kernel.

Time-sharing systems have for a long time provided for multiple simultaneous
user space instances on within a single kernel. Combined with file-system user
permissions and user groups, these provided for the very first sandboxing
capabilities.

Recent developments in modern operating systems have facilitated more
fine-grained sandboxing by virtualizing underlying system resources. Such a
virtualized user space instance is typically called a ``jail'' or a
``container''. Similarly, user programs are ``jailed'', or ``contained''.

The pitfall of OSLV in general, is that we become more vulnerable to kernel
vulnerabilities. If a sandboxed program can utilize a kernel vulnerability, the
whole system is under threat.

\subsubsection{FreeBSD Jails}

FreeBSD's \cite{man-8-jail} is perhaps the first and most notorious (due to
naming) implementation of OSLV. The problem is that it only runs on strictly
FreeBSD-based operating systems, e.g. not Mac OS X. This is impractical because
FreeBSD is less popular than Linux in general, and so a lot of software,
perhaps necessary for assessment, may be a hurdle to maintain.

\subsubsection{Linux Kernel Containment}

The Linux kernel, has in the recent decade acquired many of the features of
FreeBSD jails, and more. Containerisation in the Linux kernel is also done in a
fairly extensible fashion, providing for new extensions, abstracting over new
system resources in the future.

A lot of this work is inspired by OpenVZ, an OSLV technology, based on a
modified Linux kernel. Many of their changes to the kernel have been making
their way into the mainstream kernel\cite{kerrisk-2012}. Due to this gradual
merge, and having more confidence in the mainstream Linux kernel, we consider
only the containerisation options already available there.

\subsubsection{LXC et al.}

There exist a number of userspace interfaces to the Linux kernel containment
features. We have found all these tools lacking in some respect, and so have
gone with setting up our own lightweight interface. Ideally, we'd probably just
contribute some changes to e.g. the LXC project. The process has however,
provided for a better understanding of this part of the Linux kernel.

Linux Containers (LXC) were originally developed by Daniel Lezcano (IBM) to
ease life for kernel developers working on the kernel containerisation
features\footnote{See also \url{https://linuxcontainers.org/}.}. To our
knowledge, LXC is the most feature-rich, and stable userspace interface to the
Linux kernel containment features\cite{graber-2014}.

We would like for the processes of a submission to run in separate sandboxes,
and yet be able to communicate with one another. We would like for many
submissions to be assessable at once, and for many sandboxes to share various
resources, such as a read-only root file system.

LXC boasts pretty much all the features that we need, but we have been unable
to set up mount points as we'd prefer. Also, LXC demands creating new, managed
containers before code can be run in a sandbox. With our interface, sandboxes
exist only for the lifetime of the sandboxed program, and sandboxes are
fundamentally capable of sharing e.g. a read-only squashfs root file system.

There is also a range of high-level frameworks for managing Linux containers,
using LXC, and other drivers. For instance, libvirt-lxc\footnote{See also
\url{http://libvirt.org/}.} and Docker\footnote{See also
\url{https://www.docker.io/}.}. These tools seem to focus on providing a
sandboxed userland rather than a sandboxed execution of a program. In a way,
this stems from the general LXC approach to containers, providing for managed
containers, rather than managed classes of containers.

These frameworks, together with LXC, can prove useful when designing standard
templates for courses. Having e.g. appropriate software already installed. We
will not be concerned with this here.
