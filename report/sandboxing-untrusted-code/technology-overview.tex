% vim: set spell:

\section{Technology Overview}

There are two general approaches to providing sandboxed program execution
environments. First is providing a sandboxed operating system, second is
providing an isolated program execution environment within a regular operating
system. Of course, the ``regular'' operating system may be sandboxed for other
reasons, unrelated to the particular case of sandboxed program execution.

\subsection{Dedicated Servers}

We can provide an off-the-shelf operating system sandbox using a dedicated
server for each executing program. This however, relinquishes remote control of
the execution environment, and may demand physical access to the machine in
case of failure. This is impractical.

\subsection{Virtual Machines}

The next option is to provide an operating system by means of hardware or
software virtualization. This retains remote control of the execution
environment. However, it imposes huge costs on every execution. An entire
operating system has to boot up before testing can commence.

Alternatively a pool of virtual machines could be kept online, pulling tasks
from a task queue. Such a set up is not fail-fast, again because an entire
operating system may have to be rebooted in case of failure. Moreover, virtual
machines do not provide fine-grained control of system resources. Combined with
empirical evidence that student programs fail often, this is impractical.

Besides, such high-level virtualization makes the execution environment hard to
monitor. The overhead of the operating system may dilute the true costs
inherent in executing student or staff programs. This is desirable for the
purposes of evaluation and sandbox tuning.

\subsection{Operating system-level virtualization}

Operating system-level virtualization avoids having a separate kernel for every
program execution environment. Instead, multiple user space instances are
virtualized on top of a single operating system kernel. Such an instance is
typically called a jail or a container.

The pitfall is that we become more vulnerable to vulnerabilities in the host
kernel. If a program in a contained environment utilizes a kernel
vulnerability, they may affect the entire system.

\subsubsection{FreeBSD Jails}

\subsubsection{Linux Kernel Containment}

\begin{itemize}

\item LXC

\item libvirt-lxc

\item Docker

\end{itemize}

