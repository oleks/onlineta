% vim: set spell:

\section{Technology Overview}

A program is executed within a program execution environment. A sandboxed
execution environment ensures the non-interference of the program with other
programs being executed on the host.

There are two general approaches to providing sandboxed program execution
environments: sandboxing the operating system, or sandboxing within an
operating system. The first two subsections discuss the former, the last
subsection discusses the latter.

\subsection{Dedicated Servers}

We can provide an off-the-shelf operating system sandbox using a dedicated
server for every program. This however, relinquishes remote control of the
execution environment, and may demand physical access to the machine in case of
failure. This is impractical. Also, this is expensive since most computer
systems today are intended as time-sharing systems.

\subsection{Virtual Machines}

The next option is to provide an operating system by means of hardware or
software virtualization. This retains remote control of the execution
environment. However, it imposes huge costs on every execution. An entire
operating system has to boot up before testing can commence.

Alternatively a pool of virtual machines could be kept online, pulling tasks
from a task queue. Such a set up does not always fail fast, again because an
entire operating system may have to be rebooted in case of failure. Combined
with empirical evidence that student programs fail often, this is impractical.

Such high-level virtualization also makes the execution environment hard to
monitor. The overhead of the operating system may dilute the true costs
inherent in executing various programs. For similar reasons, fine-grained
resource limits are often hard to enforce. All this is desirable for the
purposes of evaluating our programs and tuning our sandboxes.

\subsection{Operating system-level virtualization}

Operating system-level virtualization alleviates the need for a separate kernel
for sandboxing program execution environments.

Time-sharing systems have for a long time provided for multiple simultaneous
user space instances on top of a single kernel. Combined with file-system user
permissions and user groups, these provided for the very first sandboxing
capabilities.

Recent developments in modern operating systems have facilitated more
fine-grained sandboxing by virtualizing underlying system resources. Such a
virtualized user space instance is typically called a ``jail'' or a
``container''.

The pitfall of operating system-level virtualization in general, is that we
become more vulnerable to vulnerabilities in the kernel. If a contained program
can utilize a kernel vulnerability, the whole system is under threat.

\subsubsection{FreeBSD Jails}

\subsubsection{Linux Kernel Containment}

\begin{itemize}

\item LXC

\item libvirt-lxc

\item Docker

\end{itemize}

