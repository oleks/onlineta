% vim: set spell:

\section{Resource Limits}

The system call \cite{man-2-getrlimit}, and it's sibling \texttt{setrlimit(2)}
and \texttt{prlimit(2)}, can be used to manage per-user soft and hard limits on
various resources. In general, when a process reaches a soft limit, it is
warned, and upon reaching a hard limit it is killed, or simply prohibited in
acquiring more of the resource.

An unprivileged user can freely change their soft limit to any value between 0
and the hard limit, or irreversibly lower their hard limit. A privileged user
can freely change either value.

Having a designated ``container'' user or group, we can enforce these limits at
their login using Pluggable Authentication Modules (PAM). Of particular
interest, is the \cite{pam-limits} module. We now discuss some interesting
resources that we can limit, and how they can be of use.

\subsection{nofile}

The limit on the number of open files. This can be used to control how much I/O
students perform and mitigate for various denial-of-service attacks.

\subsection{nproc}

The limit on the number of tasks that a user can create. This combined, with a
limit on the wall-clock time can be used to mitigate for fork bombs. A fork
bomb is a process that recursively creates new tasks in attempt to cause a
denial-of-service.
