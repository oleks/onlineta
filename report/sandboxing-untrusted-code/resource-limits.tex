% vim: set spell:

\section{Resource Limits}

\label{section:sandboxing:resource-limits}

The system call \cite{man-2-getrlimit}, and its siblings, \texttt{setrlimit(2)}
and \texttt{prlimit(2)}, can be used to manage per-user soft and hard limits on
various resources. In general, a process is warned upon reaching a soft limit,
and killed, or prohibited in acquiring more of that resource, upon reaching a
hard limit.

An unprivileged user can freely change their soft limit to any value between 0
and the hard limit, or irreversibly lower their hard limit. A privileged user
can freely change either value. Limits are enforced per user session, and are
enforced throughout the lifetime of a user session.

We discuss some interesting
possible limits below. Some further technical details are discussed in
\referToAppendix{resource-limits}.

\begin{description}[\setleftmargin{0.2in}\breaklabel\setlabelstyle{\tt}]

\item[RLIMIT\_NPROC] A limit on the number of tasks a user can create per session.

\item[RLIMIT\_CPU] A limit on the total CPU time per process.

\item[RLIMIT\_CORE] A limit on the size of the core file.

\end{description}

\texttt{RLIMIT\_NPROC}, combined with \texttt{RLIMIT\_CPU} and a limit on the
wall-clock time, can be used to mitigate for fork bombs, busy, and long-running
processes. A fork bomb is a process that recursively creates new tasks in
attempt to cause a denial-of-service. A limit on the wall-clock time ensures to
kill off long-running processes which consume little to no CPU time, e.g.
interactive processes waiting unduly for user input.

Putting a limit on the core file size can be a good idea to prevent students
from inspecting us inspecting the runtime of their programs. This may reveal
certain aspects of the system, which they in turn may use to exploit it.

There are other interesting limits, but combined with control groups, we have
found these to be of little use.
