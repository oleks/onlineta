% vim: set spell:

\section{Seccomp}

Student programs may often be expected to use but a handful of system calls.
Whitelisting the syscalls that students may use provides both for making sure
that they use the intended tools, and that student programs cannot abuse
obscure, or even vulnerable parts of the Linux kernel.

The secure computing (seccomp) part of the Linux kernel is provides for such
syscall filtering\cite{seccomp-filter.txt}.

Seccomp was originally conceived by Andrea Arcangeli\cite{arcangeli-2005} while
working on the cpushare project, a service that would let you to contribute
your idle CPU, for those in need of CPU time\cite{lwn.net-2005}. This requires
executing arbitrary code on your computer, and so limiting the allowed syscalls
to read/write/exit/sigreturn was a simple sandboxing idea. The Chromium team
has since worked on extending this functionality\cite{tinnes-2012} with an
approach inspired by Berkley Packet Filters\cite{drewry-2012}.

Seccomp filters can be applied using the \cite{man-2-prctl} system call. We can
use this to specify a filter program, that given a particular syscall will
reply whether the syscall should be executed or not. Some sample filter
programs can be found in \cite{linux-kernel-v3.14.2}, under
\texttt{./samples/seccomp/}.
