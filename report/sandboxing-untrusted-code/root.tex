% vim: set spell:

\chapter{Sandboxing untrusted code}

\begin{quotation}

\footnotesize\sffamily\itshape

\begin{flushright}

Going all the way back to early time-sharing systems, we systems people
regarded the users, and any code they wrote, as the mortal enemies of us and
each other. We were like the police force in a violent slum.

\smallbreak

\upshape

--- ROGER NEEDHAM, IEEE Symposium on Security and Privacy (1999)

\end{flushright}

\end{quotation}

Students submit digital files in response to assignments. Some of these files
may specify executable computer programs. Computer-assisted assessment consists
of a range of static and dynamic analyses of a submission. Static analyses
constitute executing computer programs specified by the teaching staff, which
read and analyze student files. Dynamic analyses constitute executing, and
monitoring the runtime behaviour of student programs.

Student programs may misbehave in a myriad of different ways. The programs of
the teaching staff, although more trustworthy, may also misbehave. If nothing
else, they may undermine the misbehaviour of students. The intent of this
chapter is to discuss the means in which we can mitigate for such misbehaviour
for all parties, and ensure fair service.

In the first section we provide a high-level overview of the technologies that
can be used for such ``sandboxing''. We come to the conclusion that
operating-system level virtualization is a best candidate option. The remainder
of the chapter deals with basic principles of virtualizing and limiting system
resources in \cite{linux-kernel-v3.14.2}, henceforth the Linux kernel.

% vim: set spell:

\section{Technology Overview}

A program is executed within a program execution environment. A sandboxed
execution environment ensures the non-interference of the program with other
programs being executed on the host.

There are two general approaches to providing sandboxed program execution
environments: sandboxing the operating system, or sandboxing within an
operating system. The first two subsections discuss the former, the last
subsection discusses the latter.

\subsection{Dedicated Servers}

We can provide an off-the-shelf operating system sandbox using a dedicated
server for every program. This however, relinquishes remote control of the
execution environment, and may demand physical access to the machine in case of
failure. This is impractical. Also, this is expensive since most computer
systems today are intended as time-sharing systems.

\subsection{Virtual Machines}

The next option is to provide an operating system by means of hardware or
software virtualization. This retains remote control of the execution
environment. However, it imposes huge costs on every execution. An entire
operating system has to boot up before testing can commence.

Alternatively a pool of virtual machines could be kept online, pulling tasks
from a task queue. Such a set up does not always fail fast, again because an
entire operating system may have to be rebooted in case of failure. Combined
with empirical evidence that student programs fail often, this is impractical.

Such high-level virtualization also makes the execution environment hard to
monitor. The overhead of the operating system may dilute the true costs
inherent in executing various programs. For similar reasons, fine-grained
resource limits are often hard to enforce. All this is desirable for the
purposes of evaluating our programs and tuning our sandboxes.

\subsection{Operating system-level virtualization}

Operating system-level virtualization alleviates the need for a separate kernel
for sandboxing program execution environments.

Time-sharing systems have for a long time provided for multiple simultaneous
user space instances on top of a single kernel. Combined with file-system user
permissions and user groups, these provided for the very first sandboxing
capabilities.

Recent developments in modern operating systems have facilitated more
fine-grained sandboxing by virtualizing underlying system resources. Such a
virtualized user space instance is typically called a ``jail'' or a
``container''.

The pitfall of operating system-level virtualization in general, is that we
become more vulnerable to vulnerabilities in the kernel. If a contained program
can utilize a kernel vulnerability, the whole system is under threat.

\subsubsection{FreeBSD Jails}

\subsubsection{Linux Kernel Containment}

\begin{itemize}

\item LXC

\item libvirt-lxc

\item Docker

\end{itemize}



% vim: set spell:

% TODO: test instance

\section{Control Groups}

Control groups (cgroups) provide a mechanism of hierarchically
grouping/partitioning tasks (see also
\referToAppendix{general-linux-concepts:tasks}) and their future
children\cite{cgroups.txt}. On their own, cgroups are perhaps only useful for
simple job tracking. The idea, is to have other subsystems hook into the
cgroups functionality and provide for management of system resources.

The standard cgroup subsystems include subsystems to monitor and limit memory,
CPU time, I/O, and device activity. Most modern Linux distributions come with
cgroups and many of these standard subsystems enabled. The system's
\texttt{/proc/config.gz} can reveal the setup on your system\cite{man-5-proc}.
If \texttt{CONFIG\_CGROUPS} is enabled, you have cgroups support.

The variables related to various subsystems are explored further in the
following subsections. First however, we discuss how cgroups can be accessed
and manipulated from user space in general.

\subsection{Managing cgroups}

Cgroups are managed via a pseudo-filesystem: cgroups reside in memory, but can
be manipulated through the virtual file system.  \texttt{cgroup} is therefore
an inherent file system type on systems that have the cgroups functionality
enabled.

\subsubsection{Cgroups, subsystems, and hierarchies}

A cgroup is an association of a set of tasks with a set of preferences for one
or more subsystems.  A hierarchy is a set of cgroups arranged in a tree; every
task in the system is contained in exactly one cgroup of a hierarchy; all
cgroups in a hierarchy, associate their tasks with the same set of preferences.

We say that a hierarchy is ``associated'' with a set of subsystems, and that a
task is ``attached'' to a cgroup.

(picture!!)

\subsubsection{Mounting}

When mounting a cgroup file system, we create a new hierarchy. The set of
subsystems to associate with the hierarchy is listed as mount options:

\begin{lstlisting}
$ mount -t cgroup -o cpu,cpuacct cgroup ./cgroup/cpu,cpuacct
\end{lstlisting}

This associates the \texttt{cpu} and \texttt{cpuacct} subsystems with a new
hierarchy, and mounts the hiearchy under the target
\texttt{./cgroup/cpu,cpuacct}, unless one of the subsystems is busy.

A subsystem is busy e.g. if it is associated with a hierarchy having tasks
attached.  Since a hierarchy is automatically attached to all tasks in the
system, this effectively means that a subsystem may be associated with at most
one hierarchy. If a hierarchy assosciated with the exact same set of subsystems
already exists however, it will be reused for the new mount.

What hierarchies already exist, and what subsystems they are associated with,
depends on the system at hand. The system's \texttt{/proc/mounts} can reveal
how this is setup on your system\cite{man-5-proc}.

\subsubsection{Control files and child groups}

After a hierarchy is successfully mounted, we see a range of files, and perhaps
folders, below our target. We refer to these as control files and child groups,
respecitvely.

We monitor/modify the preferences of a cgroup by monitoring/modifying the
control files. We create/remove child groups by creating/removing
subdirectories below our target.

With a few exceptions at the root of the hierarchy, all cgroups contain the
same files, created when the cgrpup is created. Some files are common to all
hierarchies, others are due to the associated subsystems. Two common files are
of particular interest:

\begin{description}[\setleftmargin{0.2in}\breaklabel\setlabelstyle{\tt}]

\item [cgroup.procs] Lists the set of thread group IDs in the current cgroup.
Appending a thread group ID to this file moves all the threads in the thread
group into this cgroup.

\item [tasks] Lists the set of thread IDs in the current cgroup. Appending a
thread ID to this file moves the thread into this cgroup.

\end{description}

\subsubsection{Hierarchical accounting}

Hierarchical accounting is when resource accounting is child group aware. All
resource usage is summed up for all tasks in the cgroup, and recursively
for all child groups. Limits are then emposed on the entire hierarchy. A
subsystem does not necessarily perform hierarchical accounting.

\subsection{The Resource Counter}

\label{section:cgroups:resource-counter}

The resource counter is a framework for managing a resource when using control
groups\cite{resource-counter.txt}.  The internal data structures aside, the
framework makes recommendations wrt. the control files. A couple of the
recommended control files are of interest to us:

\begin{description}[\setleftmargin{0.2in}\breaklabel\setlabelstyle{\tt}]

\item[<resource>.max\_usage\_in\_<unit\_of\_measurement>] Reading this file, we
get the maximal usage of the resource over time, in the given units. Writing to
this file, resets the value to the current usage of the resource. (The data
written is ignored.)

\item[<resource>.limit\_in\_<unit\_of\_measurement>] Reading this file, we get
the maximal allowed usage of the resource, in the given units. Writing to this
file resets the limit to the given value. A special value may indicate no
limit.

\end{description}

These files are of interest to us as they allow us to probe the usage of a
resource in a test instance and set up resource limits for students or staff.

\subsection{memory}

The memory subsystem allows us to monitor and limit the memory usage of the
tasks in a cgroup\cite{memory.txt}. This includes both user and kernel memory
and swap usage.  The subsystem optionally performs hierarchical accounting.

Due to the considerable overhead of memory and swap accounting, some
distributions do not enable this cgroup, or merely do not enable swap
accounting by default. The latter is especially misleading. If swapping is
enabled, a memory limit with no swap limit has at best a hapless effect.

You can check the setup on your system by checking the options prefixed with
\texttt{CONFIG\_MEMCG\_} in your \texttt{/proc/config.gz}. Swap accounting can
be enabled using the standard kernel parameter
\texttt{swapaccount=1}\cite{kernel-parameters.txt}. Enabling the memory cgroup
can be a little more distribution-specific.  In a Debian kernel, this can be
done using the kernel parameter
\texttt{cgroup\_enable=memory}\cite{hutchings-2011}.

The memory subsystem uses a resource counter for a couple different memory
resources. The resource counter control files (see also
\referToSection{cgroups:resource-counter}) are prefixed as follows:

\begin{description}[\setleftmargin{0.2in}\breaklabel\setlabelstyle{\tt}]

\item[memory]

The main memory counter. This includes both user and kernel memory.

\item[memory.memsw]

The main memory, plus swap. Limiting this value to the same value as the main
memory controller, disables swap.

\item[memory.kmem]

Kernel memory. All kernel memory is also accounted for by the main memory
counter. It is not necessary to limit this value if swapping is disabled and
there is a limit on the main memory counter (since kernel memory cannot be
swapped out).

\item[memory.kmem.tcp]

Kernel TCP buffer memory. Although we will disallow networking in general, it
might be a good idea to 0-limit this resource as an extra precaution.

\end{description}

The limits and usage are always measured in bytes. Setting the limit to
\texttt{-1}, removes the limit on the resource.

The memory subsystem does not necessarily perform hierarchical accounting. This
can enabled by writing \texttt{1} to the \texttt{memory.use\_hierarchy} control
file in the root cgroup.

\subsection{cpuacct}

The CPU accounting (cpuacct) subsystem allows us to monitor the CPU time usage
of the tasks in a cgroup\cite{cpuacct.txt}. The cpuacct subsystem always
performs hierarchical accounting.

The cpuacct subsystem provides a couple control files of interest:

\begin{description}[\setleftmargin{0.2in}\breaklabel\setlabelstyle{\tt}]

\item[cpuacct.usage] Shows the total CPU time spent by the cgroup, in
nanoseconds.

\item[cpuacct.usage\_percpu] Shows the total CPU time spent by the cgroup, for
each CPU core, in nanoseconds.

\item[cpuacct.stat] Shows a further division of the CPU time spent. For now,
showing how much of the CPU time was spent running in user mode, and how much
in kernel mode, in the \texttt{USER\_HZ} time unit.

\end{description}

\subsection{cpu}

The cpu subsystem facilitates CPU scheduling parameters for a
cgroup\cite{sched-design-cfs.txt,sched-bwc.txt,sched-rt-group.txt}. The
parameters currently facilitate control over two different schedulers in the
Linux kernel:

\begin{description}[\setleftmargin{0.2in}\breaklabel\setlabelstyle{\it}]

\item[Completely Fair Scheduler (CFS)] A proprtional share CPU scheduler. The
CPU time is divided fairly among tasks depending on their priority
and the share assigned to their cgroup.

\item[Real-Time Scheduler (RT)] A real-time scheduler for real-time tasks, i.e.
tasks for which it is important to meet deadlines. For real-time tasks, a
particular amount of CPU time must be guaranteed over a particular period of
time.

\end{description}

For the RT scheduler, the subsystem parameters facilitate limiting how much CPU
time the real-time tasks in a cgroup may spend in total over a period of time.
Enforcing such limits and meeting real-time deadlines seems like a heedful
task. For simplicity, we'll disallow students from spawning real-time tasks. In
a default setup, spawning real-time tasks requires privileged access, which we
already do not grant to our sandboxed programs.

The CFS parameters facilitate first-and-foremost the enforcing of a lower bound
on the amount of CPU time allocated to a cgroup. This is done by assigning a
relative share (weight) to a cgroup. The shares are enforced, only if tasks
from different cgroups are competing for CPU time. This means that if a cgroup
gets no competition, it gets all the CPU time it wants.

With the advent of ``cloud computing'' however, it has also become relevant to
facilitate enforcing upper bounds on the CPU time over a period of
time\cite{turner-et-al-2010}. This is facilitated similarly to the RT scheduler.

We choose to let the students spend all the CPU time they want, as long as fair
service is ensured for all students and staff. There is therefore only one
control file in this subsystem of interest to us:

\begin{description}[\setleftmargin{0.2in}\breaklabel\setlabelstyle{\tt}]

\item[cpu.shares] Show/set the relative CPU time share of a cgroup. Two cgroups
having share 100, will be given equal service. If one of the groups has share
200, it gets twice as much CPU time under a fully-loaded system. This control
file in the root cgroup, provides for a yard-stick for all other cgroups.

\end{description}

All these options do not allow us to hard limit the amount of CPU time used by
a cgroup in total. To our knowledge there is no ``natural'' way of doing this
in the Linux kernel. We must make due with limiting the wall-clock running time
of an untrusted program.

\subsection{cpuset}

The cpusets subsystem allows us to assign a set of CPUs and a set of memory
nodes to a cgroup. This can be used to further partition system resources from
a fairly high level.

We choose to let the system be runnable on commodity hardware. With few
processor cores, and few memory nodes, this subsystem is of little use to us.

\subsection{devices}

\subsection{blkio}


% vim: set spell:

\section{Namespaces}

\label{section:sandboxing:namespaces}

The purpose of a Linux namespace is to abstract over a system resource, and
make it appear to tasks within the namespace, as though they have their own
independent instance of the resource. Various namespace types abstract over
various system resources. For each namespace type, a task is associated with
exactly one namespace of that type.

Namespaces are hierarchical in the following a sense: A system boots with one
global namespace for each namespace type.  Tasks inherit their parent's
namespaces by default. A task can be disassociated from a namespace and
associated with another at runtime.

A task can be associated with a namespace using the \cite{man-2-unshare},
\cite{man-2-setns}, or \cite{man-2-clone} system calls.  The first
disassociates the process from a namespace, associating it with a new
namespace. The second, associates the process with an already existing
namespace. The last is a general system call for task creation, allowing to
create a task, already in a new namespace.

We say that a parent namespace is a ``host'', and a child namespace is a
``container''. A host may host many containers, and a container may have many
hosts, i.e. all of its ancestors. Typically, we'll only talk about a direct
child-parent relationship.

We discuss various namespace types in the following subsections. Some types,
such as mnt, pid, and net, require for a user to be priveleged to create a
child namespace. We also discuss some technical details in
\referToAppendix{namespaces}.

\subsection{mnt}

\label{section:namespaces:mnt}

The mount (MNT) namespace abstracts over the mount points of a system. This
allows for processes in different namespaces to have different views of the
file system. Within a container, we can unmount points that are perhaps needed
by the host, but not by the container, and would perhaps make the host
vulnerable, if the container had access to them.

\subsubsection{Pivot root}

One particularly useful application of mount namespaces is pivoting the file
system root to some other point in the file system using
\cite{man-2-pivot-root}. Pivoting the root in a container does not affect the
host, or other containers. At the same time, pivoting the root moves all the
dependencies on the old root, to the new root within the container.

This allows us to subsequently unmount the old root, provided that the old rot
is not busy. The old root is busy if there are mounts to targets under the old
root, the running process originates, or has files open under the old root.  By
closing all open files, switching to a process originating from under the new
root, and unmounting all mounts under the old root, we can get to unmount the
old root.

\cite{man-2-pivot-root} hides the original root file system in a matter similar
to \cite{man-2-chroot}, but makes reestablishing the old root slightly more
cumbersome, since the old root has to be properly remounted first. By mounting
the new root in a \texttt{tmpfs}, or perhaps even \texttt{squashfs}, we can
also deny access to all block devices using the devices control group subsystem
(see also \referToSection{control-groups:devices}). This makes remounting the
old root all the more cumbersome.

\subsection{uts}

The UNIX Time-sharing System (UTS) namespace abstracts over the host- and domain
name of a system. This allows each container to retain a personal host- and
domain name, perhaps different from the underlying host.

\subsection{ipc}

The Interprocess Communication (IPC) namespace abstracts over the IPC
resources, in particular System V IPC objects and POSIX message queues.
Processes within one namespace cannot communicate with processes in another
namespace using these primitives.

\subsection{pid}

The Process Identifier (PID) namespace abstracts over the task identifiers of a
system. Tasks in different namespaces can have the same pid within their
respective namespaces, but they all have distinct pid's on their hosts.
Hierarchies are implemented for pid namespaces such that a host can see all the
processes created within a container, while a container cannot see any of the
processes on a host. Effectively, the first process in a child namespace gets
pid 1, the same as an init process.

\subsection{net}

The Network (NET) namespace abstracts over the system resources associated with
networking. Each network namespace has its own network devices, IP addresses,
IP routing tables, port numbers, etc.

For simplicity, we'll prohibit students in doing in networking. This is easy to
limit with a network namespace --- all networking configuration of the host is
dropped for a new child namespace.

\subsection{user}

\label{section:sandboxing:namespaces:user}

The User (USER) namespace abstracts over the user and group ID number space.
A new namespace has its own set of identifiers, starting at 0.

Hierarchies are implemented such that a user id in a container is mapped to a
user id on the host (and likewise for groups). For instance, we can have a
particular designated ``container user'' on the host, and map this user to UID
0 inside the container. This way, even if a malicious user managed to perform a
privilege escalation within the container, this would merely correspond to an
unprivileged user on the host.

This opens up containers to a wide range of capabilties, which would have
otherwise required a privileged user on the host. For instance, containers can
now be created using an unprivileged user in general, by creating a user
namespace first.

Arguably, this leaves too much of the kernel wide open for a container, and
many find that user namespaces deserve to be tested further before being
enabled by default\cite{kerrisk-2013, arch-linux-userns,fedora-userns}.


% vim: set spell:

\section{Resource Limits}

The system call \cite{man-2-getrlimit}, and it's sibling \texttt{setrlimit(2)}
and \texttt{prlimit(2)}, can be used to manage per-user soft and hard limits on
various resources. In general, when a process reaches a soft limit, it is
warned, and upon reaching a hard limit it is killed, or simply prohibited in
acquiring more of the resource.

An unprivileged user can freely change their soft limit to any value between 0
and the hard limit, or irreversibly lower their hard limit. A privileged user
can freely change either value.

Having a designated ``container'' user or group, we can enforce these limits at
their login using Pluggable Authentication Modules (PAM). Of particular
interest, is the \cite{pam-limits} module. We now discuss some interesting
resources that we can limit, and how they can be of use.

\subsection{nofile}

The limit on the number of open files. This can be used to control how much I/O
students perform and mitigate for various denial-of-service attacks.

\subsection{nproc}

The limit on the number of tasks that a user can create. This combined, with a
limit on the wall-clock time can be used to mitigate for fork bombs. A fork
bomb is a process that recursively creates new tasks in attempt to cause a
denial-of-service.


% vim: set spell:

\section{Linux Security Modules}

Linux Security Modules (LSM) is a framework that provides a mechanism for
various security checks to be hooked (responded to) by new kernel
extensions\cite{wright-et-al-2002,security-lsm-since-linux-3.3}. The main use
of LSM is the implementation of mandatory access control, providing a
comprehensive security policy.

\subsection{SELinux}

\subsection{AppArmor}

\subsection{Capabilities}



% vim: set spell:

\section{Seccomp}

Student programs may often be expected to use but a handful of system calls.
Whitelisting the syscalls that students may use provides both for making sure
that they use the intended tools, and that student programs cannot abuse
obscure, or even vulnerable parts of the Linux kernel.

The secure computing (seccomp) part of the Linux kernel is provides for such
syscall filtering\cite{seccomp-filter.txt}.

Seccomp was originally conceived by Andrea Arcangeli\cite{arcangeli-2005} while
working on the cpushare project, a service that would let you to contribute
your idle CPU, for those in need of CPU time\cite{lwn.net-2005}. This requires
executing arbitrary code on your computer, and so limiting the allowed syscalls
to read/write/exit/sigreturn was a simple sandboxing idea. The Chromium team
has since worked on extending this functionality\cite{tinnes-2012} with an
approach inspired by Berkley Packet Filters\cite{drewry-2012}.

Seccomp filters can be applied using the \cite{man-2-prctl} system call. We can
use this to specify a filter program, that given a particular syscall will
reply whether the syscall should be executed or not. Some sample filter
programs can be found in \cite{linux-kernel-v3.14.2}, under
\texttt{./samples/seccomp/}.

