% vim: set spell:

\section{Motivation}

Our current approach keeps students and staff close to the command line (as
opposed to e.g. using a web-based interface). The reason is part idealogical
and part practical. We believe that students become better programmers when
kept close to the command line. At the same time, it is easier to implement a
comprehensive information security policy in a command line bound system. A
myriad of free and open source command line tools exist for assuring integral
and authentic data interchange.

In particular, as an extra security precaution, we choose to encrypt student
data, and have students digitally sign their data. This way, even if
unauthorised users do gain access to the data, they still have to break the
encryption to see the data, or fake a digital signature to tamper with it.  For
now, it is hard to imagine, let alone implement, such precautions in a
web-based interface.

In the following subsections we briefly justify our choice of public key
cryptography, and discuss various technologies that we can use to implement our
cryptographic scheme.

\subsection{Public Key Cryptography}

\subsection{OpenSSL}

OpenSSL is an open source implmentation of the SSL and TLS protocols, providing
at the same time a comprehensive general purpose cryptography
library\cite{openssl-org-2014}. Despite popular belief, the \texttt{openssl(1)}
command-line utility can be used to encrypt and sign regular files.

\subsection{OpenPGP}
