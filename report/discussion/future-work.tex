% vim: set spell:

\section{Future Work}

We find that our approach to Linux containment is an interesting contrast to
the rest. Combining simple, lightweight sandboxes into more sophisticated ones
seems more true to the ``Unix philosophy''. It also gives us greater control of
how our sandboxes are set up.

A lot of work remains to be done with our individual sandboxes, making them
even more robust, independent, and lightweight. Also, a lot of sandboxing
capabilities stand to be added, for instance, seccomp.

Clearly, we need to apply the techniques discussed above and combine them into
a system usable for a course.

Once we do, we are likely to realise needs for a range of user space tools for
the configuration of our assessment engines. For instance, tools that can
automatically figure out which shared libraries should be included in a root
file system for a particular dynamically linked program to be sandboxed.

If we ever move on to actually use such a system in a real course, we will
probably observe a lack of general tools for the testing of student programs.
This may become a large area of interest in and of itself --- to discover a
useful set of testing tools for all the conceivable courses.

\subsection{Course Content and Key Management}

Using Git for course content management may not be very user-friendly for
students in disciplines other than computer programming. Even beginning
computer programmers might find it perplexing.

Git gave us a simple way to retain multiple submissions to assignments,
multiple versions of assignments, and multiple versions of assignment feedback.
Git also gave us useful hooks which we used to provide for an encrypted data
store and a way for students to digitally sign their submissions. We also used
Git hooks to automatically trigger the assessment of a submission when it
arrives.

An interesting direction of future work would be to provide a more
user-friendly interface to all this, or find a way to achieve similar
objectives with e.g. a web-based platform. The major incumbent to this is our
idea of students encrypting and signing their submissions. One source of
inspiration, might be the new HTML5 \texttt{<keygen>} element\footnote{See also
\url{http://www.w3.org/html/wg/drafts/html/master/forms.html}.}.
