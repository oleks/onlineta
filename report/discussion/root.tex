% vim: set spell:

\chapter{Discussion}

\label{section:discussion}

\begin{quotation}

\footnotesize\sffamily\itshape

\begin{flushright}

Some of the more obscure uses of \texttt{pivot\_root()} may quickly lead to insanity.

\smallbreak

\upshape

--- \texttt{pivot\_root(2)} (2012)

\end{flushright}

\end{quotation}

% We've sketched some aspects of a system that may serve as a framework for
% performing automated assessment of practical work in computer programming
% courses. We've discussed many technologies that we can use both to provide
% sandboxed execution environments, as well as a safe and secure course
% management system. The techniques discussed provide for lightweight system,
% capable of guaranteeing a high level of safety, security and fairness in
% performing assessments and managing course data.

% Unfortunately, due to time constraints, the techniques described have only
% been tried out in a scratchbook fashion. They have not been all compiled into
% a system that is ready for demonstration at the time of writing.

For the purposes of discussion, we've created an Open Virtualization Appliance
(OVA) using VirtualBox, running Ubuntu 14.04, the server edition. We chose
Ubuntu, because it comes with user namespaces enabled by default (see also
\referToSection{sandboxing:namespaces:user}). The OVA is intended to
demonstrate some of the techniques discussed above.

% We've created a couple mirrors for the distribution of the OVA:

% \begin{itemize}

% \item TODO

% \end{itemize}

% The SHA-512 sum of the OVA file is: TODO.

We'll be using 4 system users: \texttt{onlineta} --- the system administrator,
\texttt{git} --- the Git user, and \texttt{kristian}, \texttt{louise}, and
\texttt{mikkel} --- students. For each user, their login password, OpenSSH, and
GnuPG password is their username.

We'll abuse the system administrator account for the tasks of the teaching
staff, and the Git user for running assessments. Ideally, each of these would
have their own dedicated users. In a real setting, students would not need to
have users on the system either. We do this for the purposes of a simple
demonstration.

You should be able to log into the OVA using SSH, once booted on your host.
For instance,

\begin{lstlisting}
$ ssh -p 2222 onlineta@127.0.0.1
\end{lstlisting}

If not, you may need to set up port forwarding in your OVA client. In the case
above, we forward host port 2222 to guest port 22.

% vim: set spell:

\section{Future Work}

Using Git for course content management may not be very user-friendly for
students in disciplines other than computer programming. Even beginning
computer programmers might find it perplexing.

Git gave us a simple way to retain multiple submissions to assignments,
multiple versions of assignments, and multiple versions of assignment feedback.
Git also gave us useful hooks which we used to provide for an encrypted data
store and a way for students to digitally sign their submissions. We also used
Git hooks to automatically trigger the assessment of a submission when it
arrives.

An interesting direction of future work would be to provide a more
user-friendly interface to all this, or find a way to achieve similar
objectives with e.g. a web-based platform. The major incumbent to this is our
idea of students encrypting and signing their submissions. One source of
inspiration, might be the new HTML5 \texttt{<keygen>} element\footnote{See also
\url{http://www.w3.org/html/wg/drafts/html/master/forms.html}.}.


