% vim: set spell:

\chapter{Discussion}

\label{section:discussion}

\begin{quotation}

\footnotesize\sffamily\itshape

\begin{flushright}

/* War is peace. Verbosity is silence. MS\_VERBOSE is deprecated. */

\smallbreak

\upshape

--- DAVID HOWELLS, Linux Source Code (2012)

\end{flushright}

\end{quotation}

We've sketched some aspects of a system that may serve as a framework for
performing automated assessment of practical work in computer programming
courses. We've discussed many technologies that we can use both to provide
sandboxed execution environments, as well as a safe and secure course
management system. The techniques discussed provide for lightweight system,
capable of guaranteeing a high level of safety, security and fairness in
performing assessments and managing course data.

We've also implemented some basic sandboxing capabilities that can be chained
together to form more complicated sandboxes. Notably, we can communicate with
our sandboxed programs using standard in, out, the file system, and even trace
them with e.g. \texttt{strace(1)}.

Unfortunately, due to time constraints, the techniques described have only been
tried out in a scratchbook fashion. They have not been all compiled into a
system that is ready for any course at the time of writing. In particular, an
infrastructure for static and dynamic analyses, as well as feedback processing
is missing. Never-the-less we believe that this will be easy to assemble given
the work done here.

\section{Virtual Machine}

For the purposes of some discussion, we've created an Open Virtualization
Appliance (OVA) using VirtualBox, running Ubuntu 14.04, server edition. We
chose Ubuntu, because it comes with user namespaces enabled by default (see
also \referToSection{sandboxing:namespaces:user}). The OVA is intended to
demonstrate some of the techniques discussed above.

% We've created a couple mirrors for the distribution of the OVA:

% \begin{itemize}

% \item TODO

% \end{itemize}

% The SHA-512 sum of the OVA file is: TODO.

We'll be using 4 system users: \texttt{onlineta} --- the system administrator,
\texttt{git} --- the Git user, and \texttt{kristian}, \texttt{louise}, and
\texttt{mikkel} --- students. For each user, their login password, OpenSSH, and
GnuPG password is their username.

We'll abuse the system administrator account for the tasks of the teaching
staff, and the Git user for running assessments. Ideally, each of these would
have their own dedicated users. In a real setting, students would not need to
have users on the system either. We do this for the purposes of a simple
demonstration.

You should be able to log into the OVA using SSH, once booted on your host.
For instance,

\begin{lstlisting}
$ ssh -p 2222 onlineta@127.0.0.1
\end{lstlisting}

If not, you may need to set up port forwarding in your OVA client. In the case
above, we forward host port 2222 to guest port 22.

\textbf{Remark}: For simplicity, we've disabled swap.

\section{Monitoring and Limits}

Under \texttt{/home/onlineta/jail} you will find the sandboxing programs
discussed in \referToSection{assessment-engine:sandboxes-and-chaining}, as well
as the \texttt{monitor} program discussed in
\referToSection{assessment-engine:monitoring-and-limits}.

Although we've implemented a farily sophisticated sandbox, we have not made use
of the following concepts:

\begin{itemize}

\item User namespaces.

\item Capabilities.

\item Seccomp.

\item Disabling block device access.

\item Detaching rather than unmounting the root file system (see also
\referToSection{assessment-engine:root-file-system}).

\end{itemize}

Notably, the \texttt{monitor} program also has to be run with \texttt{sudo} as
it needs to perform mounting operations --- something that, to our knowledge,
requires root privileges on the host.

\subsection{The Observer Effect}

Our \texttt{monitor} program provides for a couple options for monitoring and
limits. We've noticed that applying a timeout, or using SquashFS has
considerable effects on our monitoring results. We present these results below
together with an execution of \texttt{monitor} using bind mounts and no timeout
(we pressed a key after some time).

\begin{lstlisting}
$ sudo ./monitor /home/student/stall
Press any key to stop stalling..
Wall-clock time (seconds): 8.29
Max memory use (bytes): 221184
Total CPU time (nanoseconds): 3019991
Total CPU time in user mode (USER_HZ): 0
Total CPU time in kernel mode (USER_HZ): 0
\end{lstlisting}

\begin{lstlisting}
$ sudo ./monitor -s /home/student/stall
Press any key to stop stalling..
Wall-clock time (seconds): 4.50
Max memory use (bytes): 3100672
Total CPU time (nanoseconds): 13034532
Total CPU time in user mode (USER_HZ): 0
Total CPU time in kernel mode (USER_HZ): 1
\end{lstlisting}

\begin{lstlisting}
sudo ./monitor -w 5 /home/student/stall
Press any key to stop stalling..
Wall-clock time (seconds): timeout
Max memory use (bytes): 180224
Total CPU time (nanoseconds): 4942463093
Total CPU time in user mode (USER_HZ): 161
Total CPU time in kernel mode (USER_HZ): 335
\end{lstlisting}

So using SquashFS here yields a 15x increase in memory use ise and 4x increase
in CPU time, i.e. the lowest limit on memory we can set on memory is about 24
megabytes. Using a timeout there yields over 1600x increase in CPU time. This
yields timeouts unusable together with limits on the CPU time.

This is unfortunate, as we would have liked more fine grained control of how
much memory, CPU time, and wall-clock time students may use. At the same time,
dropping a timeout limit allows student programs to stall.  Using bind mounts
ahead of SquashFS disallows from disabling access to block devices in general.
We may be able to solve the latter problem using an approach similar to our I/O
file system, but we have not had time to look into this.

Recall that we already showed that we safely catch fork bombs in
\referToSection{assessment-engine:monitoring-and-limits}.

% vim: set spell:

\section{Future Work}

Clearly, we need to apply the techniques discussed above and combine them into
a working system.

Once we do, we are likely to realise needs for a range of user space tools for
the configuration of our assessment engines. For instance, tools that can
automatically figure out which shared libraries should be included in a root
file system for a particular dynamically linked program to be sandboxed.

If we ever move on to actually use such a system in a real course, we will
probably observe a lack of general tools for the testing of student programs.
This may become a large area of interest in and of itself --- to discover a
useful set of testing tools for all the conceivable courses.

\subsection{Course Content and Key Management}

Using Git for course content management may not be very user-friendly for
students in disciplines other than computer programming. Even beginning
computer programmers might find it perplexing.

Git gave us a simple way to retain multiple submissions to assignments,
multiple versions of assignments, and multiple versions of assignment feedback.
Git also gave us useful hooks which we used to provide for an encrypted data
store and a way for students to digitally sign their submissions. We also used
Git hooks to automatically trigger the assessment of a submission when it
arrives.

An interesting direction of future work would be to provide a more
user-friendly interface to all this, or find a way to achieve similar
objectives with e.g. a web-based platform. The major incumbent to this is our
idea of students encrypting and signing their submissions. One source of
inspiration, might be the new HTML5 \texttt{<keygen>} element\footnote{See also
\url{http://www.w3.org/html/wg/drafts/html/master/forms.html}.}.

